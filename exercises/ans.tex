\begin{Solution}{1}
\textbf{Quick overview of R}

R, also called GNU S, is a strongly functional language and environment. It provides a wide variety of statistical (linear and nonlinear modeling, classical statistical tests, time-series analysis, classification, clustering, simulation, optimization...) and graphical techniques, and is highly extensible. To get a flavor of the possibilities just try the
following commands:

\begin{itemize}
\item \texttt{example(lm)}
\item \texttt{demo(graphics)}
\item \texttt{demo(lm.glm)}
\item \texttt{library(tcltk); demo(tkcanvas)}
\item \texttt{RSiteSearch("GMM\textquotedblright)}
\end{itemize}

It is open-source and can be run under Windows and any UNIX System (MAC, Linux). It is a programming language, so we type commands (\quotedblbase we ask\textquotedblright ) and R executes them (\quotedblbase R answers\textquotedblright ).

R also has a rudimental GUI (Graphical User Interface), which can be used to organize the workspace, load packages and list and manipulate some objects. For instance look at \texttt{Verschiedenes}. There are buttons for the following functions:
\begin{itemize}
  \item \texttt{objects()} or \texttt{ls()} lists all variables and
      objects
  \item \texttt{search()} lists all packages that are currently in use
  \item \texttt{rm(x,y,z)} removes the variables x,y and z
  \item \texttt{rm(list=ls())} removes everything (be careful!)
\end{itemize}

There are better GUIs like SciViewsR, JGR or RCommander. We, however, won't use them, but will instead type everything in the command window or even better in a script editor. TinnR, Notepad++ (in combination with NPPtoR) or Eclipse (in combination with StatET) are highly recommendable. In our opinion, one of the best editors is RStudio, so we will make use of its functionality! If you need assistance installing those editors, please contact us.

There are a couple of good books and manuals for R, for instance:
\begin{itemize}
    \item Behr, Andreas (2011) - Einf\"{u}hrung in die Statistik mit R
    \item Crawley, Michael J. (2007) - The R Book
    \item Farnsworth, Grant V. (2008) - Econometrics in R
    \item Verzani, John (2005) - Using R for Introductory Statistics
    \item The R Development Core Team (2010): An Introduction to R
    \item The command RSiteSearch("whatever-you-want-to-find") helps to find functions and references on the internet.
\end{itemize}
\end{Solution}
\begin{Solution}{1.1}
\textbf{Starting and quitting}

During the start R searches for two files in the current working directory: .RData and .Rhistory. Those files contain your workspace and your history. You can create them by saving your workspace and saving your history (simply use the menu bar: \texttt{Datei}). The basic mathematical operators are \texttt{+,-,*,/,\symbol{94}.} Please try the following code:

\begin{verbatim}
#############################
### Starting and quitting ###
#############################
1+1
2-1
3/2
2*4
2^10
q()
# In order to save your workspace, you can also use
save.image(file.choose())
#In order to save your history, i.e. the commands you've executed, use
savehistory(file.choose())
#To load your workspace and history use
load(file.choose())
loadhistory(file.choose())
#better: use the GUI!!!
\end{verbatim}

There are two very handy key shortcuts:
\begin{itemize}
    \item $\uparrow $ and $\downarrow $ scroll through the history,i.e.
        all the commands you have typed so far.
    \item Tab ($\leftrightarrows $) completes commands, functions and
        variable names
\end{itemize}
\end{Solution}
\begin{Solution}{1.2}
\textbf{Scripts}

Scripts are a great way to organize your code, since it can be very impractical to scroll through all the commands and execute them one by one. So please use an editor!

\emph{Variables:} Variables are placeholders for any content you can imagine: numbers, vectors, matrices, text, dates, times, tables, data, etc. In order to assign a value or a content into a variable, use \texttt{x <- 5.6}. Note: \texttt{x <-5,6} produces an error. Try the following code:

\begin{verbatim}
###############
### Scripts ###
###############
a <- 3
b <- 4
c <- a+b
#there are three ways to print the contents of the variable
(c <- a+b) #Parentheses around a command prints its output
print(c)   	#The command print()
c	#Just type the name of the variable
pi
Pi
PI
\end{verbatim}

\emph{Upper and lower cases, dot and comma:} R differentiates between upper and lower cases, try: \texttt{pi}, \texttt{Pi} und \texttt{PI}. Keep that in mind when calling functions and commands, and also when naming variables. The decimal point is the dot.
\end{Solution}
\begin{Solution}{1.3}
\textbf{Working directory}

To change the working directory you can also use the GUI: \texttt{(Datei -- Verzeichnis wechseln)}.

\emph{Text and strings:} You have to use quotation marks and put your text in between those. Missing quotation marks are the most common error you get. Try the following code:
\begin{verbatim}
#########################
### Working directory ###
#########################
getwd()
setwd("c:/") #important: Windows uses the backslash \, R uses the slash /
getwd()
\end{verbatim}
\end{Solution}
\begin{Solution}{1.4}
\textbf{Help and comments}

\emph{Help}
\begin{itemize}
\item It is very important to get used to the extensive help functions in R.
\item The most important one is the question mark: \texttt{?function} (example: \texttt{?mean}).
\item If you can't remember the exact name of the function, try
   \begin{itemize}
     \item The TAB-button $\leftrightarrows$.
     \item \texttt{??mean} searches for aliases, concepts or titles that include \texttt{mean}.
     \item \texttt{apropos("\emph{mean}")} lists all objects that contain \texttt{mean}.
     \item \texttt{example("\emph{mean}")} gives you an example.
     \item \texttt{find("\emph{mean}")} gives you the name of the package containing the function mean.
   \end{itemize}
\end{itemize}

\emph{Comments:} You can put comments in your script using \#. This is extremely important in order to keep an organized and understandable code. Please get used to comment what you are doing. Please have a look at the following code:
\begin{verbatim}
#########################
### Help and comments ###
#########################
?mean
??mean
apropos("mean")
example("mean")
find("mean")
\end{verbatim}
\end{Solution}
\begin{Solution}{1.5}
\textbf{Packages}

There are several methods to install new packages:
\begin{enumerate}
  \item Using the GUI:
    \begin{itemize}
     \item \texttt{Pakete -- Installiere Pakete}, choose a close-by mirror.
     \item Highlight AER and xlsx (using the CTRL button you can highlight several items).
     \end{itemize}
  \item Using the command window:
     \begin{itemize}
       \item \texttt{install.packages("xlsx")}
     \end{itemize}
\end{enumerate}
     After you downloaded the packages you can load them either by typing: \newline \texttt{library(xlsx)} and \texttt{library(AER)} or using the GUI: \texttt{Pakete -- Lade Paket}.\newline Please have a look at the help files: \texttt{library(help=xlsx)} and \texttt{library(help=AER)}. Here's the code:
\begin{verbatim}
################
### Packages ###
################
#If you want to specify a folder for the installation files of the package,
#use: .libPaths("X:/Your Folder")
install.packages("xlsx")
#very important: JAVA needs to be installed on your computer otherwise you get an error
library(xlsx)
library(help=xlsx)
library(AER)
\end{verbatim}
\end{Solution}
\begin{Solution}{2.1}
\textbf{Reading text files}

The output should look like this:
\begin{verbatim}
   Groesse Alter Tore Gehalt
1      168    21    0    1.9
2      186    20    0    1.6
3      158    21    4    3.3
4      170    20    6    1.6
(...)
\end{verbatim}
CSV (\emph{Comma-Separated Values}) is a format to describe structured data in a textfile (\texttt{.txt} or \texttt{.csv}). Each item is separated by a comma, semicolon or a tabstop. Often you can find the names of the variables in the first line (\texttt{header=TRUE} oder \texttt{header=FALSE}). The command \texttt{read.csv()} is used to import the data. You can change the separation symbol with \texttt{sep=}. The default value is the comma: \texttt{bsp1 <- read.csv("bsp1.txt",sep=",")}.

If you don't want to write the exact name of the file, simply use \texttt{file=file.choose()}. The code looks like this:
\begin{verbatim}
##########################
### Reading text files ###
##########################
bsp1 <- read.csv(file.choose())
\end{verbatim}
\end{Solution}
\begin{Solution}{2.2}
\textbf{Reading excel files}

\begin{verbatim}
###########################
### Reading excel files ###
###########################
#specify the symbol for the decimal point and separator
bsp2 <- read.csv(file.choose(), dec=",",sep=";")
bsp2
bsp2 <- read.csv(file.choose()) # if you use the "English" decimal point
bsp2 <- read.csv2(file.choose()) # if you use the "German" decimal point
bsp2

#use the package xlsx
library(xlsx)
?read.xlsx
bsp2 <- read.xlsx(file.choose(),sheetIndex=1)
bsp2
\end{verbatim}
\end{Solution}
\begin{Solution}{2.3}
\textbf{Other data formats}

Please see the following code.
\begin{verbatim}
##########################
### Other data formats ###
##########################
install.packages("foreign")
library(foreign)
library(help=foreign)
bsp3 <- read.dta(file.choose())
bsp3

#R's data format
save(bsp3, file=file.choose())
rm(bsp3)
bsp3
load(file.choose())
bsp3

#Editing data
                    # you can input as many things as you like,
scandat <- scan()   # copy&paste works as well (very handy)
data.entry(scandat) # edits your data
edit(scandat)       # edits your data
\end{verbatim}
\end{Solution}
\begin{Solution}{2.4}
\textbf{Missing values and trimming}

Please see the following code.
\begin{verbatim}
###################################
### Missing values and trimming ###
###################################
0/0
y <- c(1:3,NA,NA,4:2)
y
mean(y)
is.na(y)        # a query to get NAs
which(is.na(y)) # another way to get the positions of the NAs
y[-4]           # removes the 4th entry

y <- c(1:3,NA,NA,4:2)
y[which(is.na(y))]=0 ;y # overwrites NA with 0, note: y changes!

y <- c(1:3,NA,NA,4:2)
y[-which(is.na(y))]; y # removes the NA, note: y has not changed!
\end{verbatim}
\end{Solution}
\begin{Solution}{3.1}
\textbf{Head and tail}

Please have a look at the following code:
\begin{verbatim}
#####################
### Head and tail ###
#####################
indices <- read.csv2(file.choose())
head(indices)
tail(indices)
names(indices)
str(indices)        # gives you the structure and an overview of the object
class(indices)      # gives you the type of the object
attributes(indices) # gives you a very good overview of the attributes of the object
\end{verbatim}
\end{Solution}
\begin{Solution}{3.2}
\textbf{Attaching dataframes}

Please have a look at the following code:
\begin{verbatim}
############################
### Attaching dataframes ###
############################
#To access data there are at least 3 ways to do it
#1) With the dollar sign
dax <- indices$dax; ftse <- indices$ftse
print(dax)
#2) with braskets
dax <- indices[,1]; ftse <- indices[,2]
print(dax)
#3) with the attach command
attach(indices)
print(dax)
\end{verbatim}
\end{Solution}
\begin{Solution}{3.3}
\textbf{Simple plots }

\begin{verbatim}
####################
### Simple plots ###
####################
plot(dax)
plot(log(dax)) #compare the y-axis!
\end{verbatim}
The graphs look the same, the y-axis, however, has changed. The range is smaller using logs.
\end{Solution}
\begin{Solution}{3.4}
\textbf{Stock returns}
Please see the following code:
\begin{verbatim}
#####################
### Stock returns ###
#####################
dim(indices)
n <- dim(indices)[1]; n
head(dax)
head(dax[2:n])     # vector containing all elements of dax except the first one

tail(dax)
tail(dax[1:(n-1)]) # vector containing all elements of dax except the last one
rdax <- log(dax[2:n]/dax[1:(n-1)])
head(rdax)
length(rdax)
rftse <- log(ftse[2:n]/ftse[1:(n-1)])

plot(rdax)
library(MASS)
truehist(rdax)

mean(rdax)
var(rdax)
sd(rdax)
median(rdax)
quantile(rdax,probs=c(0.01,0.99))
range(rdax)

boxplot(rdax)
boxplot(rdax,rftse)

plot(rdax,rftse)

cor(rdax,rftse)

m <- length(rdax)
#Attention dim(rdax) doesn't work (dim(rdax)=NULL), so we have to use the length of the vector.
#Note: A vector has the dimension of NULL!
cor(rdax[2:m],rdax[1:(m-1)])
\end{verbatim}

  The plots, the histogram as well as the boxplot show the well-known stylized facts about stock returns:
  \begin{itemize}
  \item The mean is around 0, but positive (positive expected return).
  \item The standard deviation is a measure of risk.
  \item Compared to a normal distribution, one can see that the daily returns are not perfectly symmetric around the mean (weak asymmetry).
  \begin{itemize}
  \item Large negative returns are more often than large positive ones.
  \item Large positive returns are in absolute terms greater than large negative returns.
  \end{itemize}
  \item There's more mass in the tails of the distribution (fat tails).
  \item The center is, compared to a normal distribution, higher (peakedness).
  \end{itemize}
  \item The computed statistics support the evidence for those stylized facts.
\end{Solution}
\begin{Solution}{4}
\textbf{More on graphics }

\texttt{plot()} is a very powerful command. Among other things it creates a graphical window, a Cartesian coordinate system and it plots the data. Most graphic functions expect x- and y-coordinates. You can load these from variables, enter them manually or use the function \texttt{locator(n)}, where you can simply click in the plot. The parameter \texttt{n} indicates the number of times you have to click for coordinates. Thus, \texttt{points(locator(4))} expects four clicks and after that it puts four points in the graph.

You can add other graphics like \texttt{points()}, \texttt{lines()},\texttt{legend()},... Please note that in order to use those functions you first have to call the \texttt{plot()} function. Thus, when plotting several graphics, write a script and execute all commands each time you change something.

There are several parameters that are pretty common for all graphical functions. Among others: \texttt{xlab=, ylab=, main=, col=, pch=, type=, ylim=, xlim=, lty=, lwd=} \dots

\texttt{par()} creates a new windows for several graphics. It doesn't draw a coordinate system, however, so you have to use \texttt{plot()} again.
\end{Solution}
\begin{Solution}{4.1}
\textbf{School data }

Please see the following code:
\begin{verbatim}
###################
### School data ###
###################
#1)
caschool  <- read.csv(file.choose())
head(caschool)
tail(caschool)
names(caschool)
str(caschool)
str <- caschool$str #Note that you have now overwritten the str() function!
testscr <- caschool$testscr
plot(str,testscr)

#2)
plot(str,testscr,xlab="Student Teacher Ratio", ylab="Test Score")

#3)
plot(str,testscr,xlab="Student Teacher Ratio", ylab="Test Score",main="CA Test Score Data")

#4)
print(colors())
plot(str,testscr,xlab="Student Teacher Ratio", ylab="Test Score",main="CA Test Score Data",
                                                                                col="violet")

#5)
?points #read the pch settings
plot(str,testscr,xlab="Student Teacher Ratio", ylab="Test Score",main="CA Test Score Data")
points(mean(str),mean(testscr), col= "red", pch=23)

#6)
#either
plot(str,testscr,xlab="Student Teacher Ratio", ylab="Test Score",main="CA Test Score Data")
points(mean(str),mean(testscr), col= "red", pch=19)
text(mean(str)-1,mean(testscr),"MEAN", col="red")
#or
plot(str,testscr,xlab="Student Teacher Ratio", ylab="Test Score",main="CA Test Score Data")
points(mean(str),mean(testscr), col= "red", pch=19)
text(locator(1),"MEAN", col="red") #locator(n) asks for n positions

#7)
el_pct <- caschool$el_pct
meal_pct <- caschool$meal_pct
calw_pct <- caschool$calw_pct
par(mfrow=c(2,2))
plot(testscr, str, xlab = "Test Score", ylab="Student-teacher-ratio")
plot(testscr, el_pct,xlab = "Test Score", ylab="Percentage English learners")
plot(testscr, meal_pct,xlab = "Test Score", ylab="Percentage reduced price lunch")
plot(testscr, calw_pct,xlab = "Test Score", ylab="Percentage income assistance")
\end{verbatim}
\end{Solution}
\begin{Solution}{4.2}
\textbf{Index returns}

Please see the following code:
\begin{verbatim}
#####################
### Index returns ###
#####################
#1)
indices <- read.csv2(file.choose())
head(indices)
tail(indices)
names(indices)
str(indices)

dax <- indices$dax
dax_norm <- 100*dax/dax[1]
par(mfrow=c(3,1))
plot(dax)
plot(dax_norm)
plot(dax,dax_norm) #Perfect linear correlation
cor(dax,dax_norm)  #Perfect linear correlation

#2)
par(mfrow=c(1,1))
ftse <- indices$ftse
ftse_norm <- 100*ftse/ftse[1]
plot(dax_norm,type="l", col="blue")
lines(ftse_norm, col="red")

#3)
plot(dax_norm,type="l", col="blue")
lines(ftse_norm, col="red")
legend(locator(1), legend=c("Normalized Dax", "Normalized FTSE"), fill = c("blue","red"))
\end{verbatim}
\end{Solution}
\begin{Solution}{5.1}
\textbf{Using functions}

Have a look at the following code:
\begin{verbatim}
#######################
### Using functions ###
#######################
#1)
log(10,10)
log(10,base=10)

sqrt(2); sin(pi); exp(1); log(10)
log(10,10);log(10,base=10)
sqrt(2

#2)
simpsons <- c("Homer","Marge","Bart","Lisa","Maggie")
x <- c(1,2,3,4,5,6,7,8,9,10)
x <- c(1:10)
length(simpsons); sum(x); mean(x)
simpsons[3]
simpsons[-3]

#3)
x <- 0:10
sum(x<5)
x<5

x*(x<5)
sum(x*(x<5)) #simple

x[x<5]
sum(x[x<5]) # more elegant
\end{verbatim}
\texttt{sum(x<5)} is equal to 5, because \texttt{x<5} gives you a vector consisting of 1 and 0. These numbers indicate if the value in \texttt{x} is smaller than 5, i.e. TRUE(1) if it's smaller and FALSE(0) if not. \texttt{sum(x<5)} counts these 0's and 1's. The right solution is \texttt{sum(x*(x<5))} or \texttt{sum(x[x<5])}. Try to understand why!

\end{Solution}
\begin{Solution}{5.2}
\textbf{Sequences and other vectors}

\begin{verbatim}
###################################
### Sequences and other vectors ###
###################################
?seq
?rep

# x=(1,2,...,100)
x <- 1:100; x

# x=(2,4,...,1000)
x <- 2*(1:500)
x <- seq(2,1000,by=2); x

# equi-spaced grid from -4 to 4 with 500 grid points
x <- seq(-4,4,length.out=500); x
length(x)
x[2]-x[1]
x[2]-x[1] == x[300]-x[299]

# 100 missing values
x <- rep(NA,100); x

# x=(0,1,2,0,1,2,...,0,1,2) with length 300
x <- rep(c(0,1,2),100); x
length(x)

# vector with 0 except a 1 at position 40, length 100
# several ways to do this
x <- c(rep(0,39),1,rep(0,60)); x ;length(x)
x <- rep(0,100); x[40] <- 1; x; length(x)
\end{verbatim}
\end{Solution}
\begin{Solution}{5.3}
\textbf{Random numbers }

\begin{verbatim}
######################
### Random numbers ###
######################
library(MASS)
#1)
?rnorm
x <- rnorm(10000); x
truehist(x)

#2)
?rt
r <- rt(500,3); r
plot(r)
mean(r) # expectation of a student t-distribution is E(r) = 0
var(r)  # variance of a student t-distribution is Var(r) = (k)/(k-2)
        # with k degrees of freedom. Here: Var(r)=3
truehist(r)

#3)
x <- 1:10
x; cumsum(x)
plot(cumsum(r)) # we se a random walk!
\end{verbatim}
\end{Solution}
\begin{Solution}{5.4}
\textbf{Loops}

\begin{verbatim}
#############
### Loops ###
#############
#1)
?Control
r <- rt(500,3)
Z <- rep(NA,length(r))

for (i in 10:length(r)) {
    Z[i] <- mean(r[-10:10 + i])
}

plot(r)
lines(Z, col="red")
abline(h=mean(r), col="blue")

#2)
??"Log normal"
?Lognormal
Z <- rep(NA,10000)
for (r in 1:10000) {
	Z[r] <- max(rlnorm(100))
}

library(MASS)
truehist(Z)
truehist(rlnorm(10000))
\end{verbatim}
\end{Solution}
\begin{Solution}{5.5}
\textbf{Functions}

\begin{verbatim}
#################
### Functions ###
#################
#1)
fexmpl1 <- function(x) {
    z <- x^2 + sin(x)
    return(z)
}
str(fexmpl1)

x <- seq(-3,3,length.out=500)
plot(x,fexmpl1(x))

#2)
mp <- function(x,p) {
	n <- length(x)
	m <- 1/n * sum(x^p)
	return(m)
}

x <- c(1,2,3)
mp(x,1)
mp(x,2)
\end{verbatim}
\end{Solution}
\begin{Solution}{5.6}
\textbf{Numerical optimization}

\begin{verbatim}
##############################
### Numerical optimization ###
##############################
?optimize
?optim
#1)
optimize(fexmpl1, upper=0, lower=-1)

#2)
f <- function(x,y) x^2+sin(x)+y^2-2*cos(y)
x <- seq(-5,5,by=.2)
y <- seq(-5,5,by=.2)
z <- outer(x,y,f)
persp(x,y,z,phi=-45,theta=45,col="yellow",shade=.65 ,ticktype="detailed")
#the Minimum appears to be at (-.5,0)

#note: When using optim for multidimensional optimization,
#the argument in your definition of the function must be a single vector
fx <- function(x) x[1]^2+sin(x[1])+x[2]^2-2*cos(x[2])
optim(c(-.5,0),fx) # does work!
optim(c(-.5,0),f) #does not work
\end{verbatim}
\end{Solution}
\begin{Solution}{6.1}
\textbf{Moments}

\begin{enumerate}
\item Since the density function of $N(0,1)$ is symmetric around 0, all
    odd moments are obviously 0. Proof:
    \begin{equation*}
      E[X^r] = E[(-X)^r] = E[(-1)^r X^r] = E[-(X^r)]=-E[X^r]\Leftrightarrow E[X^r] = 0
    \end{equation*}

    We use without proof $\mu _{2}=1$ (variance). As to moments of even
    orders $r\geq 4$,
\begin{equation*}
\mu _{r}=\int_{-\infty }^{\infty }x^{r}\frac{1}{\sqrt{2\pi }}e^{-x^{2}/2}dx
\end{equation*}%
which can be integrated by parts to
\begin{equation*}
\mu _{r}=\left. \frac{1}{r+1}x^{r+1}\frac{1}{\sqrt{2\pi }}%
e^{-x^{2}/2}\right\vert _{-\infty }^{\infty }-\int_{-\infty }^{\infty }\frac{%
1}{r+1}x^{r+1}\frac{1}{\sqrt{2\pi }}e^{-x^{2}/2}\left( -x\right) dx.
\end{equation*}%
Since the exponential function goes to zero faster than any power of $x$
goes to infinity, the first summand vanishes, and%
\begin{equation*}
\mu _{r}=\frac{1}{r+1}\underbrace{\int_{-\infty }^{\infty }x^{r+2}\frac{1}{\sqrt{2\pi }}e^{-x^{2}/2}dx}_{\mu_{r+2}}
\end{equation*}%
or%
\begin{equation*}
\mu _{r+2}=\left( r+1\right) \mu _{r}.
\end{equation*}%
Odd moments are thus the product of odd numbers. Another way to write
this product is given by $\mu _{r}=\prod_{i=1}^{r/2}\left( 2i-1\right) $.

\item The expectation is%
\begin{eqnarray*}
E(Y) &=&E\left( \exp \left( X\right) \right)  \\
&=&\int_{-\infty }^{\infty }\exp \left( x\right) \frac{1}{\sqrt{2\pi }\sigma
}\exp \left( -\frac{1}{2}\left( \frac{x-\mu }{\sigma }\right) ^{2}\right) dx
\\
&=&\int_{-\infty }^{\infty }\frac{1}{\sqrt{2\pi }\sigma }\exp \left( -\frac{1%
}{2}\left( \frac{x-\mu }{\sigma }\right) ^{2}-\frac{1}{2}\left( -2x\right)
\right) dx.
\end{eqnarray*}%
The term inside the exponential function can be written as%
\begin{eqnarray*}
&&-\frac{1}{2}\left[ \left( \frac{x-\mu }{\sigma }\right) ^{2}-\frac{%
2x\sigma ^{2}}{\sigma ^{2}}\right]  \\
&=&-\frac{1}{2}\left( \frac{x^{2}-2\left( \mu +\sigma ^{2}\right) x+\mu ^{2}%
}{\sigma ^{2}}\right)  \\
&=&-\frac{1}{2}\left( \frac{\left[ x^{2}-2\left( \mu +\sigma ^{2}\right)
x+\left( \mu +\sigma ^{2}\right) ^{2}\right] -\left( \mu +\sigma ^{2}\right)
^{2}+\mu ^{2}}{\sigma ^{2}}\right)  \\
&=&-\frac{1}{2}\left( \frac{\left[ x-\left( \mu +\sigma ^{2}\right)
\right]^2 -\mu^2 -2\mu \sigma ^{2}-\sigma ^{4}+\mu^2}{%
\sigma ^{2}}\right)  \\
&=&-\frac{1}{2}\left(\frac{\left[x-(\mu+\sigma^2)\right]^2}{\sigma^2} - 2\mu - \sigma^2 \right)
\end{eqnarray*}%
Hence,%
\begin{eqnarray*}
E(Y) &=&\int_{-\infty }^{\infty }\frac{1}{\sqrt{2\pi }\sigma }\exp \left( -%
\frac{1}{2}\left(\frac{\left[x-(\mu+\sigma^2)\right]^2}{\sigma^2}\right) + \mu +\frac{\sigma^2}{2} \right) dx \\
&=&e^{\mu +\sigma ^{2}/2}\int_{-\infty }^{\infty }\frac{1}{\sqrt{2\pi }%
\sigma }\exp \left( -\frac{1}{2}\left(\frac{\left[x-(\mu+\sigma^2)\right]^2}{\sigma^2}\right)\right)dx
\end{eqnarray*}%
The integrand is simply the density of a normal distribution with mean
$\mu +\sigma ^{2}$ and variance $\sigma ^{2}$. The integral over the
density is unity, and thus%
\begin{equation*}
E(Y)=e^{\mu +\sigma ^{2}/2}.
\end{equation*}

\item The density of the Pareto distribution is%
\begin{equation*}
f_{X}(x)=\frac{dF(x)}{dx}=\alpha K^\alpha x^{-\alpha -1}
\end{equation*}%
for $x\geq K$ (and zero elsewhere). The moment of order $%
p<\alpha $ is%
\begin{eqnarray*}
E\left( X^{p}\right)  &=&\int_{-\infty }^{\infty }x^{p}f_{X}(x)dx \\
&=&\int_{K}^{\infty }x^{p}\alpha K^\alpha x^{-\alpha -1}dx \quad\text{since } x\geq K\\
&=&\alpha K^\alpha\int_{K}^{\infty }x^{p-\alpha -1}dx \\
&=&\alpha K^\alpha\cdot \left[ \frac{1}{p-\alpha }x^{p-\alpha }\right]
_{K}^{\infty }.
\end{eqnarray*}%
Since $p<\alpha $ the expression within the square brackets goes to zero as $%
x\rightarrow \infty $, and therefore%
\begin{eqnarray*}
E\left( X^{p}\right)  &=&\alpha K^\alpha\cdot \frac{1}{\alpha -p} K^{p-\alpha } \\
&=&\frac{\alpha }{\alpha -p}K^{p}.
\end{eqnarray*}%
Moments of order $p\geq \alpha $ do not exist as the integral does not
converge.
\end{enumerate}
\end{Solution}
\begin{Solution}{7.1}
\textbf{Student teacher ratio (I)}

\begin{verbatim}
###################################
#### Student teacher ratio (I) ####
###################################
#1)
caschool <- read.csv(file.choose())
View(caschool)
testscr <-caschool$testscr
str <- caschool$str

regr <- lm(testscr~str)
print(regr)
str(regr) #a very complex object! You can access those things using the $ sign.

plot(str,testscr, xlab="Student-Teacher-Ratio", ylab="Testscore", main="Scatterplot", pch=20)
abline(regr, col="red",lwd=3, lty=2)

#2)
prediction <- predict(regr, newdata=data.frame(str=19.33))
plot(str,testscr, xlab="Student-Teacher-Ratio", ylab="Testscore", main="Scatterplot", pch=20)
abline(regr, col="red",lwd=3, lty=2)
#add a point
points(19.33, prediction, col="blue", pch=20)
#lines connects two points with a straight line.
#Note: abline needs a slope and an intercept to draw a straight line
lines(c(19.33, 19.33), c(0,prediction),lty=3, col="blue")
lines(c(0, 19.33), c(prediction,prediction),lty=3, col="blue")

#3)
#Either use the function residuals()
sum(residuals(regr))
#Or extract them from the object regr with the dollar sign
sum(regr$residuals)

#4)
plot(residuals(regr))
abline(h=0)

#5)
plot(regr$residuals,str)

#6)
print(summary(regr)) # with this command you assume homoscedasticity

library(AER)
print(coeftest(regr,vcov=vcovHC)) #with this command you assume heteroscedasticity

#7)
# Access the variables
BETA <- coeftest(regr,vcov=vcovHC)[2,1]
SDBETA <- coeftest(regr,vcov=vcovHC)[2,2]
#t-Test
H0 <- -1
t <- (BETA- (H0))/SDBETA
t
#the critical values are 1.96 (5%) and 2.58 (1%)
abs(t) > 1.96; abs(t) > 2.58
#H0 can be rejected with a significance level of 5%!
#The estimate is significantly different from -1.

#There is also a function for linear hypothesis testing  (F-Test)
library(car)
linearHypothesis(regr,c("str=-1"))

#8)
lower_limit <- BETA - 1.96*SDBETA
upper_limit <- BETA + 1.96*SDBETA

names(lower_limit)="lower limit"; names(upper_limit)="upper limit"
print(c(lower_limit, upper_limit))
\end{verbatim}
\begin{enumerate}
  \item In the scatterplot you can see that the points scatter
      heterogeneously around the regression line, since there are several
      outliers. The assumption of homoscedasticity does not hold.
  \item Note: \texttt{lines} connects points with a straight line.
      \texttt{abline}, however, needs a slope and an intercept to draw a
      straight line.
  \item The \$ sign is very handy to access data and variables from
      complex objects. Get used to using the \$.
  \item The plot shows that the residuals are heteroscedastic and
      most likely autocorrelated (not i.i.d.).
  \item According to the plot, there is no obvious linear correlation
      between the exogenous variable and the residuals.
  \item This is the standard output of a regression. In order to control
      for heteroscedasticity, use \texttt{coeftest()} and specify the
      variance-covariance matrix appropriately (e.g.
      \texttt{vcov=vcovHC}).
  \item This is a standard t-Test. You can either compute it by hand or
      use \texttt{linearHypothesis()}.
  \item The critical value for a 95\% confidence interval is 1.96.
\end{enumerate}
\end{Solution}
\begin{Solution}{7.2}
\textbf{Capital asset pricing model}

Please see the following code:
\begin{verbatim}
#####################################
#### Capital asset pricing model ####
#####################################
#1)
capm <- read.csv2(file.choose())
head(capm)
rdai <- capm$rdai
rdax <- capm$rdax
regr <- lm(rdai~rdax)
regr
library(car)
linearHypothesis(regr,c("(Intercept) = 0")) #cannot be rejected

#2)
summary(regr)
BETA <- summary(regr)$coefficients[2,1]
SDBETA <- summary(regr)$coefficients[2,2]
lower_limit <- BETA - 1.96*SDBETA
upper_limit <- BETA + 1.96*SDBETA
names(lower_limit)="lower limit"; names(upper_limit)="upper limit"
print(c(lower_limit, upper_limit))
\end{verbatim}
\end{Solution}
\begin{Solution}{7.3}
\textbf{Student teacher ratio (II)}

\begin{verbatim}
####################################
#### Student teacher ratio (II) ####
####################################
library(AER)
#1)
caschool <- read.csv(file.choose())
head(caschool)
testscr <- caschool$testscr
str <- caschool$str
el_pct <- caschool$el_pct
expn_stu <- caschool$expn_stu
regr <- lm(testscr~str+el_pct)
regr

#2)
simple <- lm(testscr~str)
r1 <- residuals(simple)
plot(r1)

multiple <- lm(testscr~str + el_pct + expn_stu)
r2 <- residuals(multiple)
plot(r2)

plot(r1,ylab="")
points(r2,col="red")

sum(r1^2)
sum(r2^2)
sum(r1^2) > sum(r2^2)

#3)
multiple <- lm(testscr~str + el_pct + expn_stu)
predict(multiple, newdata=data.frame(str=25, el_pct=0.6, expn_stu=4000))
predict(multiple, newdata=data.frame(str=17, el_pct=0.6, expn_stu=4000))

#4)
regr <- lm(testscr~str + el_pct + expn_stu)
summary(regr)
library(AER)
coeftest(regr, vcov=vcovHC)

#5)
library(car)
linearHypothesis(regr,c("str=0","expn_stu=0","el_pct=-.7"))
\end{verbatim}
\end{Solution}
\begin{Solution}{7.4}
\textbf{Omitted variable bias}

Please see the following code:
\begin{verbatim}
###############################
#### Omitted variable bias ####
###############################
library(car)
omitted <- read.csv2(file.choose())
head(omitted)
y <- omitted$y
x1 <- omitted$x1
x2 <- omitted$x2
x3 <- omitted$x3
x4 <- omitted$x4

cor(omitted)

#1
regra <-lm(y ~ x1+x2+x3+x4)
summary(regra)
linearHypothesis(regra,c("x1=2","x2=3","x3=4","x4=5"))
#2
regrb <- lm(y~x2+x3+x4)
summary(regrb)
linearHypothesis(regrb,c("x2=3","x3=4","x4=5"))
#3
regrc <- lm(y~x1+x2+x3)
summary(regrc)
linearHypothesis(regrc,c("x1=2","x2=3","x3=4"))
\end{verbatim}

Multicollinearity: Strong correlation between the exogenous
      variables. The coefficients, however, are unbiased if the model is
      specified correctly, i.e. if no variables are omitted. Otherwise you get an omitted variable bias.
      Furthermore the estimators are not efficient, they have high standard errors. Also a ceteribus
      paribus interpretation of a single coefficient is not valid.
\end{Solution}
\begin{Solution}{7.5}
\textbf{Asymptotic normality}
The code might look like this:
\begin{verbatim}
##############################
#### Asymptotic Normality ####
##############################
library(MASS)
beta_true <- c(3,2,-1)
TT <- 100
R <- 10000
set.seed(123)
X_fix <- cbind(1,mvrnorm(n=TT,c(5,10),matrix(c(1,0.9,0.9,1),2,2)))

V1 <- rep(NA,R); V2 <- rep(NA,R);
for (i in 1:R) {
  u <- runif(TT,-1,1)
  X_rand <- cbind(1,mvrnorm(n=TT,c(5,10),matrix(c(1,0.9,0.9,1),2,2)))
  y1 <- X_fix%*%beta_true + u
  y2 <- X_rand%*%beta_true + u
  beta_hat1 <- solve(t(X_fix)%*%X_fix)%*%t(X_fix)%*%y1
  beta_hat2 <- solve(t(X_rand)%*%X_rand)%*%t(X_rand)%*%y2
  V1[i] <- beta_hat1[2]
  V2[i] <- beta_hat2[2]
}
truehist(V1)
m1 <- mean(V1)
s1 <- sd(V1)
curve(dnorm(x,mean=m1,sd=s1),add=T)

truehist(V2)
m2 <- mean(V2)
s2 <- sd(V2)
curve(dnorm(x,mean=m2,sd=s2),add=T)
\end{verbatim}
\end{Solution}
\begin{Solution}{7.6}
\textbf{Pitfalls in the linear regression model (I)}

Please have a look at the following code:
\begin{verbatim}
#####################################################
#### Pitfalls in the linear regression model (I) ####
#####################################################
gehaelter <- read.csv(file.choose())
names(gehaelter)
head(gehaelter)
str(x)

dauer <- gehaelter$dauer
gehalt <- gehaelter$gehalt
fach <- gehaelter$fach

#1
plot(dauer, gehalt, xlab="Duration", ylab="Salary", main = "Scatterplot")

#2
model <- lm(gehalt~dauer)
model
plot(dauer, gehalt, xlab="Duration", ylab="Salary", main = "Scatterplot")
abline(model)

#3
chem <- lm(gehalt~dauer, data=x[fach==1,])
chem
econ <- lm(gehalt~dauer, data=x[fach==2,])
econ

plot(dauer, gehalt, xlab="Duration", ylab="Salary", main = "Scatter")
points(x[fach==1,2],x[fach==1,1],col="blue") #mark chem students
points(x[fach==2,2],x[fach==2,1], col="red") #mark econ students
abline(model)
abline(chem, col="blue")
abline(econ, col="red")
legend("topright", legend=c("Total","Chemistry", "Economics"), fill = c("black","blue","red"))
\end{verbatim}
\end{Solution}
\begin{Solution}{7.7}
\textbf{Pitfalls in the linear regression model (II)}

\begin{verbatim}
######################################################
#### Pitfalls in the linear regression model (II) ####
######################################################
storch <- read.csv2(file.choose())
head(storch)
names(storch)
str(storch)

Horstpaare <- storch$Horstpaare
Geburten <- storch$Geburten
nichtehelich<- storch$nichtehelich

#1)
plot(Horstpaare, Geburten, xlab="Number storks", ylab="Number of births")
regr <- lm(Geburten ~ Horstpaare)
regr
abline(regr)

#2)
regr1 <- lm(nichtehelich ~ Horstpaare)
regr1
plot(Horstpaare, nichtehelich, xlab="Number storks", ylab="Number of out-of-wedlock births")
abline(regr1)
\end{verbatim}
\end{Solution}
\begin{Solution}{7.8}
\textbf{Pitfalls in the linear regression model (III)}

Please have a look at the following code:
\begin{verbatim}
#######################################################
#### Pitfalls in the linear regression model (III) ####
#######################################################
indices <- read.csv2(file.choose())
head(indices)
names(indices)
str(indices)
dax <- indices$dax
ftse <- indices$ftse

n <- dim(indices)[1]
kdax <- dax[6:n]
kftselag <- ftse[1:(n-5)]

plot(kftselag,kdax)
obj <- lm(kdax ~ kftselag)
obj
summary(obj)
library(AER)
coeftest(obj, vcov=vcovHC)
\end{verbatim}
Be careful with time series data! The assumption of iid in the linear regression model is not valid anymore!
\end{Solution}
\begin{Solution}{8.1}
\textbf{Joint distributions}

\begin{enumerate}
\item If $f(x,y)$ is a density function, then the double integral over
    the support must be equal to 1. So, first integrate for x and then
    for y:
\begin{align*}
\int_0^1 f(x,y) dx = \frac{10}{3} y^3 (2-y)\\
\int_0^1 \frac{10}{3} y^3 (2-y) dy = 1
\end{align*}
\item The marginal densities are given by
\begin{align*}
f_Y(y) = \int_0^1 f(x,y) dx = \frac{10}{3} y^3 (2-y)\\
f_X(x) = \int_0^1 f(x,y) dy = -20 x^3 + 42x^2-27x+55
\end{align*}
\item The conditional density of $X$ given $Y=y$ is given by
\begin{align*}
f_X(x|y) =\frac{f(x,y)}{f_Y(y)} = 12(x-0.5)^2 \frac{3-2x-y}{2-y}
\end{align*}
\item X and Y are dependent, because:
\begin{align*}
f(x,y) \neq f_X(x)\cdot f_Y(y)
\end{align*}
\end{enumerate}
\end{Solution}
\begin{Solution}{8.2}
\textbf{Gaussianity or else?}

The program might look like this:
\begin{verbatim}
##############################
#### Gaussianity or else? ####
##############################
#1)
gaussian <- read.csv(file.choose())
View(gaussian)

library(MASS)
x <- seq(-4, 4, length=100)
par(mfrow=c(2,2))

truehist(gaussian$V1); lines(x,dnorm(x))
truehist(gaussian$V2); lines(x,dnorm(x))
truehist(gaussian$V3); lines(x,dnorm(x))
truehist(gaussian$V4); lines(x,dnorm(x))

#2)
cor(gaussian)

#3)
pairs(gaussian)

#4)
Y <- gaussian$V1+gaussian$V2+gaussian$V3+gaussian$V4
par(mfrow=c(1,1))
truehist(Y); lines(x,dnorm(x, mean=mean(Y),sd=sd(Y)))
\end{verbatim}
\begin{enumerate}
  \item Each variable seems to be normally distributed.
  \item There is no (or just a very small) correlation between the
      variables.
  \item The scatterplots show that the variables are NOT independent,
      even though they are normally distributed and uncorrelated. If one
      variable is (in absolute terms) very big, it is very likely that
      the other variables are (in absolute terms) big as well. This is
      dependence!
  \item Compared to a normal distribution the sum has more mass in the
      center of the distribution. Because of the dependence the sum is
      per se not normally distributed.
\end{enumerate}
This exercise illustrates that for the sum of normally distributed variables
being also normally distributed requires the assumption of
\textbf{independence}, not just uncorrelatedness; two separately (not
jointly) normally distributed random variables can be uncorrelated without
being independent, in which case their sum can be non-normally distributed.
\end{Solution}
\begin{Solution}{8.3}
\textbf{Gaussian and uncorrelated, but dependent}

\begin{enumerate}
  \item The distribution of Y is the same as X, because
  \begin{align*}
  &Pr(Y \leq x) = Pr(X\leq x)\cdot Pr(U=1) + Pr(-X\leq x)\cdot Pr(U=-1)\\
  &=\Phi(x) \cdot \frac{1}{2}+ \Phi(x) \cdot \frac{1}{2} = \Phi(x)
  \end{align*}
  since $X$ and $-X$ have the same distribution and $\Phi$ ist the
  distribution function of the normal distribution.
  \item X and Y are uncorrelated, since the covariance is given by
  \begin{align*}
  Cov(X,Y) = E(X\cdot Y) - \underbrace{E(X)}_{=0}\cdot E(Y) = E[E(X \cdot Y|U)] = E[X^2]\cdot \frac{1}{2} + E[-X^2]\cdot \frac{1}{2} = 0
  \end{align*}
  \item The program might look like this
\begin{verbatim}
##################################################
#### Gaussian and uncorrelated, but dependent ####
##################################################
library(MASS)
X <- rnorm(1000)
U <- sample(c(-1, 1), 1000, replace = TRUE)
Y <- U*X
truehist(X)
truehist(Y)
cov(X,Y)
plot(X,Y)
\end{verbatim}
Even though X and Y are normal and uncorrelated, they are not
independent, since it is very likely that if X is large, Y is in absolute
terms also large. In fact: $|Y|=|X|$.
\item The additional code might look like this:
\begin{verbatim}
Z <- X+Y
mean(Z)
par(mfrow=c(1,1))
truehist(Z)
\end{verbatim}
Because X and Y are dependent, the sum is not normally distributed (see
also ex. \emph{Gaussianity or else}).
\end{enumerate}

\end{Solution}
\begin{Solution}{9.1}
\textbf{Law of large numbers}

The programs might look like this:
\begin{verbatim}
##############################
#### Law of large numbers ####
##############################
#1)
z <- rep(NA,1000); u <- rep(NA,1000); g <- rep(NA,1000)
for (i in 1:1000) {
  z[i] <- mean(rnorm(i,mean=10,sd=2))
  u[i] <- mean(runif(i,min=0,max=1)) #expectation is (a+b)/2
  g[i] <- mean(rgeom(i,prob=0.2)) #expectation is (1-p)/p
}

par(mfrow=c(1,2))

truehist(z,main="Law of large numbers")
plot(z, main="for the normal distribution"); abline(h=10,lwd=2,col="red")

truehist(u,main="Law of large numbers")
plot(u,main="for the uniform distribution"); abline(h=0.5,lwd=2,col="red")

truehist(g,main="Law of large numbers")
plot(g,main="for the geometric distribution"); abline(h=4,lwd=2,col="red")

#2)
z <- rep(NA,1000)
rho=0.8
mu=2
for (i in 1:1000) {
  z[i] <- mean(filter((1-rho)*mu+rnorm(i,sd=2),rho,method="recursive",init=(1-rho)*mu))
  #try and use a different distribution
}
par(mfrow=c(1,2),pty="s")
truehist(z,main="Law of large numbers")
plot(z, main="for intertemporal dependence AR(1)"); abline(h=mu,lwd=2,col="red")
\end{verbatim}
\end{Solution}
\begin{Solution}{9.2}
\textbf{Law of large numbers for the variance}

The program might look like this:
\begin{verbatim}
###############################################
#### Law of large numbers for the variance ####
###############################################
#1)
library(MASS)
zn <- rep(NA,1000); zu <- rep(NA,1000); zg <- rep(NA,1000)
for (i in 1:1000) {
  rn <- rnorm(i,mean=10,sd=2) ; variance_n <- 2^2
  ru <- runif(i,min=0,max=6) ; variance_u <- 3
  rg <- rgeom(i,prob=0.2); variance_g <-20
  zn[i] <- sum((rn-mean(rn))^2)/i
  zu[i] <- sum((ru-mean(ru))^2)/i
  zg[i] <- sum((rg-mean(rg))^2)/i
}

par(mfrow=c(1,2))
truehist(zn,main="Law of large numbers")
plot(zn, main="for the variance (normal distrib.)"); abline(h=variance_n,lwd=2,col="red")

truehist(zn,main="Law of large numbers")
plot(zu, main="for the variance (uniform distrib.)"); abline(h=variance_u,lwd=2,col="red")

truehist(zn,main="Law of large numbers")
plot(zg, main="for the variance (geometr. distrib.)"); abline(h=variance_g,lwd=2,col="red")

#2)
zt <- rep(NA,1000)
for (i in 1:1000) {
  xt <- rt(i,df=3)
  zt[i] <- sum((xt-mean(xt))^2)/i; variance_t <-3/(3-2)
}

par(mfrow=c(1,2))

truehist(zt,main="Law of large numbers")
plot(zt, main="for the variance (t-distrib. df=3)");abline(h=variance,lwd=2,col="red")
\end{verbatim}
\end{Solution}
\begin{Solution}{9.3}
\textbf{Central limit theorem}

The program might look like this:
\begin{verbatim}
###############################
#### Central limit theorem ####
###############################
#1) Illustration of the central limit theorem for the uniform distribution
#    with sigma as the standard deviation
library(MASS)
N <- 10000 # how many times to draw individual X_i's, note that i = 1,2,..,n
expect <- 0.5
vari <- 1/12
par(mfrow = c(1,2),pty = "s")
for(n in 1:20) {
  X <- runif(N*n) #N*n random numbers
  M <- matrix(X, N, n) #matrix of N rows and n columns filled with random numbers
  Yn <- rowSums(M) # calculate the sum of n random numbers for N rows
  Zn <- (Yn/n - expect)*sqrt(n/vari) #standardization
  #display the sequence of random variables
  truehist(Zn, xlim = c(-4,4),ylim = c(0,0.5), main = paste("n =", toString(n),sep =" "))
  coord <- par("usr")
  # par("usr") gives you a vector of the form c(x1, x2, y1, y2)
  # giving the extremes of the coordinates of the plotting region
  x <- seq(coord[1], coord[2], length.out = 500)
  lines(x, dnorm(x), col = "red")
  qqnorm(Zn, ylim = c(-4,4), xlim = c(-4,4), pch = ".", col = "blue")
  abline(0, 1, col = "red")
  Sys.sleep(1)
}

#2) Illustration of the central limit theorem for the uniform distribution
#     with the estimated standard deviation
library(MASS)
N <- 10000 # how many times to draw individual X_i's, note that i = 1,2,..,n
expect <- 0.5
par(mfrow = c(1,2),pty = "s")
#hint: it needs to start with n=2, for n=1 you get vari=0 and you cannot divide by 0
for(n in 2:20) {
  X <- runif(N*n) #N*n random numbers
  M <- matrix(X, N, n) #matrix of N rows and n columns filled with random numbers
  vari <- 1/n * rowSums((M-rowMeans(M))^2)
  Yn <- rowSums(M) # calculate the sum of n random numbers for N rows
  Zn <- (Yn/n - expect)*sqrt(n/vari) #standardization
  #display the sequence of random variables
  truehist(Zn, xlim = c(-4,4),ylim = c(0,0.5), main = paste("n =", toString(n),sep =" "))
  coord <- par("usr")
  x <- seq(coord[1], coord[2], length.out = 500)
  lines(x, dnorm(x), col = "red")
  qqnorm(Zn, ylim = c(-4,4), xlim = c(-4,4), pch = ".", col = "blue")
  abline(0, 1, col = "red")
  Sys.sleep(1)
}

#3)
library(MASS)
N <- 10000 # how many times to draw individual X_i's, note that i = 1,2,..,n
expect <- 0
par(mfrow = c(1,2),pty = "s")
#hint: it needs to start with n=2, for n=1 you get vari=0 and you cannot divide by 0
for(n in 2:20) {
  X <- rt(N*n,df=1.5)
  M <- matrix(X, N, n)
  vari <- 1/n * rowSums((M-rowMeans(M))^2)
  Yn <- rowSums(M)
  Zn <- (Yn/n - expect)*sqrt(n/vari)
  truehist(Zn, xlim = c(-4,4),ylim = c(0,0.5), main = paste("n =", toString(n),sep =" "))
  coord <- par("usr")
  x <- seq(coord[1], coord[2], length.out = 500)
  lines(x, dnorm(x), col = "red")
  qqnorm(Zn, ylim = c(-4,4), xlim = c(-4,4), pch = ".", col = "blue")
  abline(0, 1, col = "red")
  Sys.sleep(1)
}
\end{verbatim}
\end{Solution}
\begin{Solution}{9.4}
\textbf{Central limit theorem for dependent data}

\begin{enumerate}
  \item First let's derive the expectation and variance of the AR(1) process with $|\rho|<1$. For this, we use recursive substitution techniques given a starting value $X_0$:
    \begin{align*}
      X_i = (1-\rho)(1+\rho+\rho^2+\dots+\rho^N)\mu + \varepsilon_i + \rho \varepsilon_{i-1} + \rho^2 \varepsilon_{i-2} + \dots + \rho^N \varepsilon_{i-N} + \rho^{N+1} X_0
    \end{align*}
    Note that $\lim_{N\rightarrow \infty} \rho^{N+1} = 0$ and $\lim_{N\rightarrow \infty} \sum_{j=0}^\infty \rho^j = \frac{1}{1-\rho}$, since $|\rho|<1$. The AR(1) process with $|\rho|<1$ can therefore be equally represented by
    \begin{align*}
    X_i = \mu + \sum_{j=1}^\infty \rho^j \varepsilon_{i-j}
    \end{align*}
    Its expectation and variance are then equal to
    \begin{align*}
    E(X_i) &= \mu + \sum_{j=1}^\infty \rho^j E(\varepsilon_{i-j}) = \mu\\
    Var(X_i) &= \sum_{j=1}^\infty (\rho^j)^2 var(\varepsilon_{i-j}) = \sum_{j=1}^\infty (\rho^2)^j \sigma_\varepsilon^2 = \frac{\sigma_\varepsilon^2}{1-\rho^2}
    \end{align*}

  \item Asymptotic distribution for mean
  \begin{enumerate}
    \item Due to our assumptions on $\varepsilon_i$, we can use the central limit theorem such that
        \begin{equation*}
            \sqrt{n} \left(\frac{1}{n} \sum_{i=1}^n \varepsilon_i \right) = \frac{1}{\sqrt{n}} \sum_{i=1}^n \varepsilon_i  \overset{d}{\rightarrow} U_\varepsilon \sim N(0,\sigma_\varepsilon^2)
        \end{equation*}
    \item Let's have a look at $\frac{1}{\sqrt{n}} \sum_{i=1}^n \varepsilon_i$:
        \begin{align*}
        \frac{1}{\sqrt{n}} \sum_{i=1}^n \varepsilon_i
        &= \frac{1}{\sqrt{n}} \sum_{i=1}^n \left[(X_i-\mu)-\rho(X_{i-1}-\mu)\right]\\
        &= \frac{1}{\sqrt{n}} \left[\sum_{i=1}^n (X_i-\mu)- \rho\sum_{i=1}^n(X_{i-1}-\mu)\right]\\
        &= \frac{1}{\sqrt{n}} \left[\sum_{i=1}^n (X_i-\mu)- \rho\left[\sum_{i=1}^n(X_{i}-\mu)-(X_n - X_0)\right]\right]\\
        &= \sqrt{n} \left[\frac{1}{n}\sum_{i=1}^n (X_i-\mu)- \rho\left[\frac{1}{n}\sum_{i=1}^n(X_{i}-\mu)-\left(\frac{X_n - X_0}{n}\right)\right]\right]\\
        &= \sqrt{n} \left[\frac{1}{n}Y_n-\mu- \rho\left[\frac{1}{n}Y_n-\mu-\left(\frac{X_n - X_0}{n}\right)\right]\right]\\
        &= \sqrt{n}\left[(1-\rho)\left(\frac{1}{n}Y_n-\mu\right) + \rho\left(\frac{X_n - X_0}{n}\right)\right]
        \end{align*}

    \item Using the definition of the probability limit, we have to show that for any $\delta>0$
        \begin{align*}
          \lim_{n\rightarrow \infty} P\left(\left|\frac{\rho}{1-\rho}\left(\frac{X_n - X_0}{\sqrt{n}}\right)\right|> \delta\right) = 0
        \end{align*}
        By Tchebychev's Inequality we have
        \begin{align*}
        P\left(\left|\frac{\rho}{1-\rho}\left(\frac{X_n - X_0}{\sqrt{n}}\right)\right|> \delta\right) \leq \frac{1}{\delta^2} var\left[\frac{\rho}{1-\rho}\left(\frac{X_n - X_0}{\sqrt{n}}\right)\right]
        \end{align*}
        Let's have a look at $var\left[\frac{\rho}{1-\rho}\left(\frac{X_n - X_0}{\sqrt{n}}\right)\right]$:
        \begin{align*}
        var\left[\frac{\rho}{1-\rho}\left(\frac{X_n - X_0}{\sqrt{n}}\right)\right]
        &= \frac{1}{n}\left(\frac{\rho}{1-\rho}\right)^2 var(X_n - X_0)\\
        &= \frac{1}{n}\left(\frac{\rho}{1-\rho}\right)^2 \left(var(X_n) + var(X_0) - 2 cov(X_n,X_0)\right]\\
        &= \frac{1}{n}\left(\frac{\rho}{1-\rho}\right)^2 \left[\frac{\sigma_\varepsilon^2}{1-\rho^2} + \frac{\sigma_\varepsilon^2}{1-\rho^2} - 2 corr(X_n,X_0)\sqrt{\frac{\sigma_\varepsilon^2}{1-\rho^2}}\sqrt{\frac{\sigma_\varepsilon^2}{1-\rho^2}} \right]\\
        &\leq \frac{1}{n}\left(\frac{\rho}{1-\rho}\right)^2 4 \left(\frac{\sigma_\varepsilon^2}{1-\rho^2}\right)
        \end{align*}
        since $corr(X_n,X_0) \geq -1$.

        Thus for any $\delta>0$, we have
        \begin{align*}
        P\left(\left|\frac{\rho}{1-\rho}\left(\frac{X_n - X_0}{\sqrt{n}}\right)\right|> \delta\right) \leq \frac{1}{\delta^2} \frac{1}{n}\left(\frac{\rho}{1-\rho}\right)^2 4 \left(\frac{\sigma_\varepsilon^2}{1-\rho^2}\right)
        \end{align*}
        In the limit
        \begin{align*}
        \lim_{n\rightarrow \infty} P\left(\left|\frac{\rho}{1-\rho}\left(\frac{X_n - X_0}{\sqrt{n}}\right)\right|> \delta\right) = 0.
        \end{align*}

    \item Now, let's go back to
            \begin{align*}
            \frac{1}{\sqrt{n}} \sum_{i=1}^n \varepsilon_i = \sqrt{n}\left[(1-\rho)\left(\frac{1}{n}Y_n-\mu\right) + \rho\left(\frac{X_n - X_0}{n}\right)\right]
            \end{align*}
            Let's divide by $(1-\rho)$
        \begin{align*}
        \frac{\frac{1}{\sqrt{n}} \sum_{i=1}^n \varepsilon_i}{1-\rho} & = \sqrt{n}\left[\frac{1}{n}Y_n-\mu\right] + \frac{\rho}{1-\rho}\left(\frac{X_n - X_0}{\sqrt{n}}\right)
        \end{align*}

        For the left-hand-side we have
        \begin{align*}
        \frac{\frac{1}{\sqrt{n}} \sum_{i=1}^n \varepsilon_i}{1-\rho} \overset{d}{\rightarrow} \tilde{U_\varepsilon} \sim N\left(0,\frac{\sigma_\varepsilon^2}{(1-\rho)^2}\right)
        \end{align*}

        Since $\textsl{plim}\left[\frac{\rho}{1-\rho}\left(\frac{X_n - X_0}{\sqrt{n}}\right)\right] = 0$, we have
        \begin{align*}
        \sqrt{n}\left[\frac{1}{n}Y_n-\mu\right] \overset{d}{\rightarrow} \tilde{U} \sim N\left(0,\frac{\sigma_\varepsilon^2}{(1-\rho)^2}\right)
        \end{align*}
        and we're done. That is, set $\sigma^2 = \frac{\sigma_\varepsilon^2}{(1-\rho)^2}$, then
        \begin{align*}
        Z_n = \sqrt{n}\frac{\left(\frac{1}{n}Y_n\right)-\mu}{\sigma} \overset{d}{\rightarrow} U \sim N(0,1)
        \end{align*}
  \end{enumerate}
  \item The program might look like this:

\begin{verbatim}
################################################
### Central limit theorem for dependent data ###
################################################
# AR(1) process
library(MASS)
n <- 10000
N <- 5000 # how many times to draw individual X_i's, note that i = 1,2,..,n
mu <- 0
rho=0.8
sigma_eps <- 0.5 #this is the standard deviation of the random term in the AR(1) process
var_x <- sigma^2/(1-rho^2) # analytical variance of an AR(1)-process
factr <- var_x*(1+rho)/(1-rho) # this is the required adjustment
X <- matrix(NA,N,n)
for (j in 1:N) {
  X[j,] <- filter((1-rho)*mu+rnorm(n,sd=sigma),rho,method="recursive",init=(1-rho)*mu)
}
Yn <- rowSums(X)
Zn <- sqrt(n)*(Yn/n - mu)/sqrt(factr) #standardization
truehist(Zn, main = paste("n =", toString(n),sep =" "))
coord <- par("usr")
x <- seq(coord[1], coord[2], length.out = 500)
lines(x, dnorm(x), col = "red")
\end{verbatim}
\end{enumerate}



\end{Solution}
\begin{Solution}{9.5}
\textbf{Delta method}

\begin{enumerate}
  \item According to a central limit theorem, we have
  $$\sqrt{n}\left(\bar{X}_n-\mu\right) \overset{d}{\rightarrow} U \sim N(0,\sigma^2).$$ The first order Taylor expansion of $f(\bar{X}_n)$ around $\mu $ is%
\begin{equation*}
f\left( \bar{X}_n \right) \approx f\left( \mu \right) +f^{\prime }\left( \mu_n  \right) \left(
\bar{X}_n-\mu \right),
\end{equation*}
where $\mu_n$ is an intermediate point between X and $\mu$. Since $|\mu_n-\mu|\leq |\bar{X}_n-\mu|$ and $X_n \overset{p}{\rightarrow} \mu$, $\mu_n \overset{p}{\rightarrow}\mu$. Since $f'$ is continuous we have $f'(\mu_n) \overset{p}{\rightarrow} f'(\mu)$.

We see that $f(\bar{X}_n)$ is just a linear transformation of $\bar{X}_n$. Since $\bar{X}_n$ is asymptotically normal, so is $f(\bar{X}_n)$. The asymptotic mean is equal to
\begin{equation*}
E\left( f(\bar{X}_n)\right) = f\left( \mu \right) +f^{\prime }\left( \mu
\right) \left( E\left(X\right)-\mu \right) = f\left( \mu \right)
\end{equation*}%
and the asymptotic variance is%
\begin{eqnarray*}
Var\left( f(\bar{X}_n)\right) &=&Var\left( f\left( \mu \right) +f^{\prime }\left(
\mu \right) \left( X-\mu \right) \right) \\
&=&\left[ f^{\prime }\left( \mu \right) \right] ^{2}Var\left( X\right)= \left[ f^{\prime }\left( \mu \right) \right] ^{2}\sigma ^{2}.
\end{eqnarray*}
So we have
$$\sqrt{n}\left(f(\bar{X}_n)-f(\mu)\right)=f'(\mu_n)\left[\sqrt{n}\left(\bar{X}_n-\mu\right)\right] \overset{d}{\rightarrow} f'(\mu)U \sim N(0,[f'(\mu)]^2\sigma^2)$$
\item The first order Taylor expansion of $f$ around $\mu $ is%
\begin{equation*}
Y \approx f\left( \mu \right) +D_f\left( \mu \right) \left(
\bar{X_n}-\mu \right)
\end{equation*}
Because Y is a linear transformation of $\bar{X_n}$, and $\bar{X_n}$ is asymptotically normal, so is $Y$. The asymptotic mean of $Y$ is%
\begin{equation*}
E\left( Y\right) = f\left( \mu \right)
\end{equation*}%
and the asymptotic variance is%
\begin{eqnarray*}
Var\left( Y\right) &=&Var\left( f\left( \mu \right) +D_f\left(\mu\right) \left( \bar{X_n}-\mu \right) \right) \\
&=&D_f\left( \mu \right) Var\left( \bar{X_n}\right)D_f\left( \mu \right)' \\
&\rightarrow & D_f\left( \mu \right)\Sigma D_f\left( \mu \right)'.
\end{eqnarray*}
\end{enumerate}

\end{Solution}
\begin{Solution}{9.6}
\textbf{Limits of maxima (I)}

The program might look like this:
\begin{verbatim}
############################
### Limits of maxima (I) ###
############################
library(MASS)
n <- 100
N <- 1000 # how many times to draw individual X_i's, note that i = 1,2,..,n
X <- matrix(rnorm(N*n),N, n) #matrix of N rows and n columns filled with random numbers
dn <- sqrt(2*log(n))-(log(4*pi)+log(log(n)))/(2*sqrt(2*log(n)))
cn <- (2*log(n))^(-1/2)
Mn <- apply(X, 1, max) # get max of each row
Rn <- (Mn - dn)/cn #standardization
#display the sequence of random variables
truehist(Rn, main = paste("n =", toString(n),sep =" "))
coord <- par("usr")
x <- seq(coord[1], coord[2], length.out = 500)
lines(x, exp(-x-exp(-x)), col = "red")
\end{verbatim}
See also the Fisher-Tippett-Galambos Theorem which states that a sample of iid random variables after proper standardization can only converge in distribution to one of 3 possible distributions: the Gumbel distribution, the Fréchet distribution, or the Weibull distribution.
\end{Solution}
\begin{Solution}{9.7}
\textbf{Limits of maxima (II)}
The program might look like this:
\begin{verbatim}
###############################
#### Limits of maxima (II) ####
###############################
library(MASS)
n <- 100
N <- 1000 # how many times to draw individual X_i's, note that i = 1,2,..,n
X <- matrix(rt(n*N,df=1.5), N, n) #matrix of N rows and n columns filled with random numbers
Mn <- apply(X, 1, max) # get max of each row
Rn <- Mn/qt(1-1/n,1.5) #standardization
#display the sequence of random variables
truehist(Rn, main = paste("n =", toString(n),sep =" "))
coord <- par("usr")
x <- seq(coord[1], coord[2], length.out = 500)
lines(x, 1.5*x^(-2.5)*exp(-x^(-1.5)), col = "red")
\end{verbatim}
See also the Fisher-Tippett-Galambos Theorem which states that a sample of iid random variables after proper standardization can only converge in distribution to one of 3 possible distributions: the Gumbel distribution, the Fréchet distribution, or the Weibull distribution.
\end{Solution}
\begin{Solution}{9.8}
\textbf{Limits of maxima (III)}

The program might look like this:
\begin{verbatim}
################################
#### Limits of maxima (III) ####
################################
library(MASS)
n <- 100
N <- 1000 # how many times to draw individual X_i's, note that i = 1,2,..,n
X <- matrix(runif(n*N), N, n) #matrix of N rows and n columns filled with random numbers
Mn <- apply(X, 1, max) # get max of each row
dn <- 1
cn <- 1/n
Rn <- (Mn-dn)/cn #standardization
#display the sequence of random variables
truehist(Rn, main = paste("n =", toString(n),sep =" "))
coord <- par("usr")
x <- seq(coord[1], coord[2], length.out = 500)
lines(x, exp(x), col = "red")
\end{verbatim}
See also the Fisher-Tippett-Galambos Theorem which states that a sample of iid random variables after proper standardization can only converge in distribution to one of 3 possible distributions: the Gumbel distribution, the Fréchet distribution, or the Weibull distribution.
\end{Solution}
\begin{Solution}{10.1}
\textbf{Counter examples}

We assume in addition that $X$ is normally distributed, $X\sim N\left( \mu
,1\right) $.

\begin{enumerate}
\item The sequence of estimators $\hat{\mu}_{n}=X_{1}$ is unbiased since $%
E\left( \hat{\mu}_{n}\right) =E\left( X_{1}\right) =\mu $. But it is
obviously not consistent.

\item The sequence of estimators $\hat{\mu}_{n}=\left( \frac{1}{n}%
\sum_{i=1}^{n}X_{i}\right) +1/n$ is biased with bias $1/n$. For given $n$
\begin{eqnarray*}
E\left( \hat{\mu}_{n}\right)  &=&\mu +\frac{1}{n} \\
Var\left( \hat{\mu}_{n}\right)  &=&\frac{1}{n},
\end{eqnarray*}%
and thus $\lim_{n\rightarrow \infty }E\left( \hat{\mu}_{n}\right) =\mu $
and $\lim_{n\rightarrow \infty }Var\left( \hat{\mu}_{n}\right) =0$.
\textbf{These are the sufficient conditions for consistency.}

\item This is the most tricky case. Consider the sequence of estimators%
\begin{equation*}
\hat{\mu}_{n}=\mathbf{1}\left( X_{1}-\frac{1}{n}\sum_{i=1}^{n}X_{i}<\Phi
^{-1}\left( 1/n\right) \right) \cdot n+\left[ 1-\mathbf{1}\left( X_{1}-\frac{%
1}{n}\sum_{i=1}^{n}X_{i}<\Phi ^{-1}\left( 1/n\right) \right) \right] \cdot
\frac{1}{n}\sum_{i=1}^{n}X_{i}
\end{equation*}%
where $\mathbf{1}\left( \cdot \right) $ is an indicator function which is 1
if the condition is true and 0 else, $\Phi ^{-1}\left( 1/n\right) $ is the $%
1/n$-quantile of the standard normal distribution. For large $n$, the term $%
X_{1}-\frac{1}{n}\sum_{i=1}^{n}X_{i}$ is approximately $N\left( 0,1\right) $
and the indicator variable will be 1 with probability $1/n$. Thus, the
expectation of the first summand of $\hat{\mu}_{n}$ is%
\begin{equation*}
E\left( \mathbf{1}\left( X_{1}-\frac{1}{n}\sum_{i=1}^{n}X_{i}<\Phi
^{-1}\left( 1/n\right) \right) \cdot n\right) =\frac{1}{n}\cdot n=1.
\end{equation*}%
The expectation of the second summand is%
\begin{equation*}
E\left( \left[ 1-\mathbf{1}\left( X_{1}-\frac{1}{n}\sum_{i=1}^{n}X_{i}<\Phi
^{-1}\left( 1/n\right) \right) \right] \cdot \frac{1}{n}\sum_{i=1}^{n}X_{i}%
\right) =\left( 1-\frac{1}{n}\right) \cdot \mu ,
\end{equation*}%
and hence the asymptotic bias of $\hat{\mu}_{n}$ is $1$. The estimator is
consistent because, as $n\rightarrow \infty $, the probability mass
concentrates on the second summand which converges to $\mu $.
\end{enumerate}
\end{Solution}
\begin{Solution}{11.1}
\textbf{Nonlinear least squares}

\item The two programs could look like this:

\begin{verbatim}
############################################
#### Nonlinear least squares estimation ####
############################################

#1)
# Load the dataset
dat <- read.csv(file.choose())

# Definition of the objective function
# The first argument must be the parameter vector
squarediffs <- function(param,dat) {
  alpha <- param[1]
  beta <- param[2]
  u <- dat$y-exp(alpha+beta*dat$x)
  return(sum(u^2))
}

# Minimize the objective function
obj <- optim(c(1,0),squarediffs,dat=dat)
estimates <- obj$par
alphahat <- estimates[1]
betahat <- estimates[2]

# Plot the data and the estimated regression curve
plot(dat$x,dat$y)
g <- seq(0,40,length=500)
lines(g,exp(alphahat+betahat*g))


#2)

# Load the dataset
dat <- read.csv(file.choose())

# Definition of the objective function
# The first argument must be the parameter vector
squarediffs <- function(param,dat) {
  beta1 <- param[1]
  beta2 <- param[2]
  u <- dat$y-(beta1+beta2*dat$x1+1/beta2*dat$x2)
  return(sum(u^2))
}

# Minimize the objective function
obj <- optim(c(1,1),squarediffs,dat=dat)
estimates <- obj$par
beta1hat <- estimates[1]
beta2hat <- estimates[2]

# Plot the data and the estimated regression plane
library(rgl)
plot3d(dat$x1,dat$x2,dat$y)
gx1 <- seq(min(dat$x1),max(dat$x1),length=60)
gx2 <- seq(min(dat$x2),max(dat$x2),length=60)
yhat <- outer(gx1,gx2,function(x1,x2) beta1hat+beta2hat*x1+1/beta2hat*x2)
surface3d(gx1,gx2,yhat,col="light green")
\end{verbatim}
\end{Solution}
\begin{Solution}{11.2}
\textbf{Method of moments for the binomial distribution}

The first step is already given in the text,%
\begin{eqnarray*}
\mu _{1} &=&n\theta  \\
\mu _{2}^{\prime } &=&n\theta \left( 1-\theta \right) .
\end{eqnarray*}%
The second step is the inversion of the two equations,%
\begin{eqnarray*}
\theta  &=&1-\frac{\mu _{2}^{\prime }}{\mu _{1}} \\
n &=&\frac{\mu _{1}}{\theta }.
\end{eqnarray*}%
Finally, the theoretical moments are replaced by their empirical
counterparts, and the moment estimators are%
\begin{eqnarray*}
\hat{\theta} &=&1-\frac{\hat{\mu}_{2}^{\prime }}{\hat{\mu}_{1}} \\
\hat{n} &=&\frac{\hat{\mu}_{1}}{\hat{\theta}}.
\end{eqnarray*}
\end{Solution}
\begin{Solution}{11.3}
\textbf{Method of moments for the geometric distribution}

\begin{enumerate}
\item From $\mu _{1}=\lambda ^{-1}$ we derive $\lambda =\mu _{1}^{-1}$, and
hence the moment estimator of $\lambda $ is $\hat{\lambda}=\hat{\mu}_{1}^{-1}
$.

\item The empirical moment $\hat{\mu}_{1}$ is an unbiased estimator of $\mu
_{1}$. The estimator $\hat{\lambda}=1/\hat{\mu}_{1}$ is a nonlinear (convex)
transformation. According to Jensen's inequality $E\left( \hat{\lambda}%
\right) =E(1/\hat{\mu}_{1})>1/E(\hat{\mu}_{1})=\lambda $.

\item The rules of calculus for the probability limit are simple,%
\begin{equation*}
\textsl{plim}\hat{\lambda}=\textsl{plim}\frac{1}{\hat{\mu}_{1}}=\frac{1}{%
\textsl{plim}\hat{\mu}_{1}}=\frac{1}{\mu _{1}}=\lambda .
\end{equation*}
\end{enumerate}
\end{Solution}
\begin{Solution}{11.4}
\textbf{Method of moments for the Gumbel distribution}

Step 1:%
\begin{eqnarray*}
\mu _{1} &=&\alpha +0.5772\cdot \beta  \\
\mu _{2}^{\prime } &=&\frac{1}{6}\beta ^{2}\pi ^{2}.
\end{eqnarray*}%
Step 2:%
\begin{eqnarray*}
\beta  &=&\sqrt{\frac{6\mu _{2}^{\prime }}{\pi ^{2}}} \\
\alpha  &=&\mu _{1}-0.5772\cdot \beta .
\end{eqnarray*}%
Step 3:%
\begin{eqnarray*}
\hat{\beta} &=&\sqrt{\frac{6\hat{\mu}_{2}^{\prime }}{\pi ^{2}}} \\
\hat{\alpha} &=&\hat{\mu}_{1}-0.5772\cdot \hat{\beta}.
\end{eqnarray*}
\end{Solution}
\begin{Solution}{11.5}
\textbf{Method of moments for the Pareto distribution}

\begin{enumerate}
  \item For a given $K$ the moment estimator of $\alpha$ is given by
      \begin{align*}
        \mu_1 = \frac{\alpha K}{\alpha - 1}\\
        \alpha = \frac{\mu_1}{\mu_1-K}\\
        \hat{\alpha} = \frac{\hat{\mu_1}}{\hat{\mu_1}-K}
      \end{align*}
      For both parameters unknown one has to solve the system of
      equations
      \begin{align*}
        \mu_1 &= \frac{\alpha K}{\alpha - 1}\\
        \mu_{2}^{\prime} &= \frac{\alpha K^2}{(\alpha-2)(\alpha-1)^2}
      \end{align*}
      for $\alpha$ and $K$. Inserting the squared first expression into
      the second yields
      \begin{align*}
        &\mu_{2}^{\prime} = \frac{\mu_1^2}{\alpha(\alpha-2)}\\
        &\Leftrightarrow\alpha^2-2\alpha-\frac{\mu_1^2}{\mu_{2}^{\prime}}=0\\
        &\Leftrightarrow\alpha = \frac{2}{2} \pm \frac{\sqrt{4(1+\frac{\mu_1^2}{\mu_{2}^{\prime}})}}{2}=1\pm \sqrt{1+\frac{\mu_1^2}{\mu_{2}^{\prime}}}
      \end{align*}
      The variance is negative for $0<\alpha<2~~ (Var(X)<0)$. Thus,
      $\alpha$ must be greater than 2 and the moment estimators are
      \begin{align*}
        \hat{\alpha} &= 1 + \sqrt{1+\frac{\hat{\mu_1}^2}{\hat{\mu_{2}}^{\prime}}}\\
        \hat{K} &= \frac{\alpha-1}{\alpha}\hat{\mu}_1
      \end{align*}
  \item The program might look like this:

\begin{verbatim}
##################################################
#### Moment estimator for Pareto distribution ####
##################################################

install.packages("VGAM") # in order to get the Pareto distribution
library(VGAM)
library(AER)

a <- 5              #shape parameter
K <- 1              #location parameter
R <- 10000          # index for the loop
n <- 100            # number of random variables
ahat <- rep(NA,R)   # initialize estimator
ahat1 <- rep(NA,R)  # initialize estimator

for (i in 1:R) {
  x <- rpareto(n,location=K,shape=a)     #generate n random variables
  ahat[i] <- mean(x)/(mean(x)-1)         #moment estimator of alpha if K is known
  ahat1[i] <- 1+sqrt(1+mean(x)^2/var(x)) #moment estimator of alpha if K is unknown
}
#graphical output
par(mfrow=c(1,2))
truehist(ahat)
g <- seq(min(ahat),max(ahat),length=300)
lines(g,dnorm(g,mean(ahat),sd(ahat)))
truehist(ahat1)
h <- seq(min(ahat1),max(ahat1),length=300)
lines(h,dnorm(h,mean(ahat1),sd(ahat1)))
\end{verbatim}
For $\alpha <2$ the second estimator is not asymptotically normal
distributed, since the moment of order $2p$ does not exist (see the
lecture for the necessary conditions regarding moment estimators).
\end{enumerate}
\end{Solution}
\begin{Solution}{11.6}
\textbf{Method of moments for the uniform distribution}

\begin{enumerate}
\item The theoretical moments $\mu _{1}$ and $\mu _{2}^{\prime }$ are
already given as functions of the unknown parameters. Solving the system%
\begin{eqnarray*}
2\mu _{1} &=&a+b \\
12\mu _{2}^{\prime } &=&a^{2}-2ab+b^{2}
\end{eqnarray*}%
for $a$ and $b$ yields two solutions. Since $b>a$ only one solution is valid,%
\begin{eqnarray*}
a &=&\mu _{1}-\sqrt{3\mu _{2}^{\prime }} \\
b &=&\mu _{1}+\sqrt{3\mu _{2}^{\prime }}
\end{eqnarray*}%
and the moment estimators are%
\begin{eqnarray*}
\hat{a} &=&\hat{\mu}_{1}-\sqrt{3\hat{\mu}_{2}^{\prime }} \\
\hat{b} &=&\hat{\mu}_{1}+\sqrt{3\hat{\mu}_{2}^{\prime }}.
\end{eqnarray*}

\item The program might look like this:

\begin{verbatim}
###################################################
#### Moment estimator for uniform distribution ####
###################################################
a <- 0
b <- 1
R <- 10000
ahat <- rep(NA,R)
bhat <- rep(NA,R)
minx <- rep(NA,R)
for(r in 1:R) {
  x <- runif(n=40,a,b)
  ahat[r] <- mean(x)-sqrt(3*var(x))
  bhat[r] <- mean(x)+sqrt(3*var(x))
  minx[r] <- min(x)
  }
par(mfrow=c(2,1))
truehist(ahat)
g <- seq(min(ahat),max(ahat),length=300)
lines(g,dnorm(g,mean(ahat),sd(ahat)))
truehist(bhat)
g <- seq(min(bhat),max(bhat),length=300)
lines(g,dnorm(g,mean(bhat),sd(bhat)))
print("Proportion of impossible estimates:")
print(sum(minx<ahat)/R)
\end{verbatim}
\end{enumerate}
\end{Solution}
\begin{Solution}{11.7}
\textbf{Method of moments for the linear regression model}

First have a look at $X'y$:
\begin{align*}
  \underset{k\times T}{X}'\underset{T \times 1}{y} =
  \begin{pmatrix}
    \sum_{t=1}^T X_{t1}\cdot y_t\\
    \vdots\\
    \sum_{t=1}^T X_{tk}\cdot y_t
  \end{pmatrix}
\end{align*}
Taking the expectation yields for each row $j$: $\sum_{t=1}^T E[X_{tj}\cdot
y_t] = T\cdot E(X_{1j} \cdot y_1)$ or any other vector of $y$ and item
$X_{tj}$.

Now, left-multiplication of $y=X\beta +u$ with $X^{\prime }$ and taking expectations yields%
\begin{equation*}
E\left( X^{\prime }y\right) =E\left( X^{\prime }X\beta \right) +E\left(
X^{\prime }u\right) = T
\begin{pmatrix}
    E(X_{11}\cdot y_1)\\
    \vdots\\
    E(X_{1k}\cdot y_1)
\end{pmatrix}.
\end{equation*}%


Step 1: We relax the standard assumptions and assume that $X$ may be
non-stochastic but independent of $u$, i.e. $E(u|X)=0$. This implies that
$E\left( X^{\prime }X\beta \right) =E\left( X^{\prime }X\right) \beta $ and
$E\left( X^{\prime
}u\right) =E\left( X^{\prime }\right) E(u)$. Of course, $E(u)=0$, and thus%
\begin{eqnarray*}
E\left( X^{\prime }y\right)  &=&E\left( X^{\prime }X\beta \right) +E\left(
X^{\prime }u\right)  \\
E\left( X^{\prime }y\right)  &=&E\left( X^{\prime }X\right) \beta .
\end{eqnarray*}%
Step 2: The system is solved for the unknown parameters $\beta $,%
\begin{align*}
\beta &= \underbrace{\left[E\left( X^{\prime }X\right)\right]^{-1}}_{\text{Matrix of expectations}} \cdot \underbrace{E\left( X^{\prime}y\right)}_\text{Vector of expectations}\\
&= \frac{1}{T}\cdot \begin{bmatrix}
                    E(X_{11}^2) & \dots & E(X_{11}X_{1k})\\
                    \vdots & \ddots &\vdots\\
                    E(X_{11}X_{1k}) & \dots & E(X_{1k}^2)
                  \end{bmatrix}^{-1}
\cdot T
\begin{pmatrix}
    E(X_{11}\cdot y_1)\\
    \vdots\\
    E(X_{1k}\cdot y_1)
\end{pmatrix}.
\end{align*}%
Step 3: Replace the theoretical moments (expectations) by their empirical
counterparts,%
\begin{equation*}
\hat{\beta}=
\begin{bmatrix}
    \frac{1}{T}\sum_{t=1}^T X_{t1}^2 & \dots & \frac{1}{T}\sum_{t=1}^T X_{t1}X_{tk}\\
    \vdots & \ddots &\vdots\\
    \frac{1}{T}\sum_{t=1}^T X_{t1}X_{tk} & \dots & \frac{1}{T}\sum_{t=1}^T X_{tk}^2
\end{bmatrix}^{-1}
\cdot
\begin{pmatrix}
    \frac{1}{T}\sum_{t=1}^T X_{t1}\cdot y_t\\
    \vdots\\
    \frac{1}{T}\sum_{t=1}^T X_{tk}\cdot y_t
\end{pmatrix}=(X^{\prime }X)^{-1}X^{\prime }y.
\end{equation*}%
This is equal to the OLS estimator of $\beta $ (and also identical to the
maximum likelihood estimators, as we will see later).
\end{Solution}
\begin{Solution}{12.1}
\textbf{Extreme values}

\begin{enumerate}
\item The maximum likelihood estimator of $\alpha$ is given by maximizing
    the loglikelihood function:
\begin{align*}
  L(\alpha;x,K) &= \prod_{i=1}^n \alpha K^\alpha x_i^{-\alpha-1}\\
  log(L(\alpha;x,K))&= \sum_{i=1}^n log(\alpha K^\alpha) + \sum_{i=1}^n log(x_i^{-\alpha-1})\\
    & = n(log(\alpha)+ \alpha log(K)) - (\alpha+1) \sum_{i=1}^n log(x_i)\\
  \frac{\partial log(L)}{\partial \alpha} &= \frac{n}{\alpha} + n ~ log(K) - \sum_{i=1}^n log(x_i) = 0\\
  \Rightarrow \hat{\alpha} &= \frac{n}{\sum_{i=1}^n (log(x_i)-log(K))}
\end{align*}
\item[2./3.] The code might look like this. The ML-estimator is also estimated by numerically optimizing the likelihood.
\begin{verbatim}
########################
#### Extreme values ####
########################
library(MASS)
library(VGAM)

daxreturns <- read.csv(file.choose(), sep=";", dec=",")
View(daxreturns)
daxret <- daxreturns$daxret
plot(daxret)
tmp <- daxret*(-1)
daxlosses <- tmp[tmp>=2]
alphahat <- length(daxlosses)/(sum(log(daxlosses) - log(2)))
plot(daxlosses)
truehist(daxlosses)
coord <- par("usr")
x <- seq(coord[1], coord[2], length.out = length(daxlosses))
lines(x,dpareto(x,scale=2,shape=alphahat),col="red",lwd=2)

#Likelihood of the observations as a function of alpha
loglikelihood <- function(alpha,K,x) {
  n <- length(x)
  z <- n*(log(alpha)+alpha*log(K)) - (alpha+1)*sum(log(x))
  return(z)
}
alpha <- seq(0,10,length.out=100)
plot(alpha,loglikelihood(alpha,2,daxlosses))
abline(v=alphahat)

#Just for comparison, let's do it numerically
likelihood <- function(alpha,K,x) {
  z <- prod(dpareto(x,K,alpha))
  return(z)
}

f <- function(x) -likelihood(x,2,daxlosses)
optimize(f,lower=0,upper=3)
\end{verbatim}
\end{enumerate}
\end{Solution}
\begin{Solution}{12.2}
\textbf{Parameters of the uniform distribution}
\begin{enumerate}
  \item Consider a sample $x_1,\dots,x_n$ and let the order statistics
      be: $x_{(1)} \leq x_{(2)} \leq \dots \leq x_{(n)}$. For $\alpha
      \leq x_{(1)}$ and $b \geq x_{(n)}$ the likelihood equals:
      \begin{align*}
        L(a,b;x) &= \prod_{i=1}^n f_X(x_i|a,b) = (b-a)^{-n}\\
        log(L(a,b;x)) &= -n\cdot log(b-a)\\
        \frac{\partial log(L)}{\partial a} &= \frac{n}{b-a} >0\\
        \frac{\partial log(L)}{\partial b} &= -\frac{n}{b-a} <0
      \end{align*}
      Thus the loglikelihood is a strictly increasing function in a and a
      strictly decreasing function in b. Hence the ML-estimator is given
      by:
      \begin{equation*}
        \hat{a} = x_{(1)} \text{ and }\hat{b} = x_{(n)}
      \end{equation*}
  \item The program might look like this:
\begin{verbatim}
################################################
#### Parameters of the uniform distribution ####
################################################
R <- 10000
n <- 100
a <- 3
b <- 6
ahat <- numeric(R)
bhat <- numeric(R)
for (r in 1:R) {
  x <- runif(n,min=a,max=b)
  ahat[r] <- min(x)
  bhat[r] <- max(x)
}
par(mfrow=c(1,2))
hist(ahat,prob=T)
hist(bhat,prob=T)
\end{verbatim}
Clearly, the ML estimators are not asymptotically normal distributed. See also execise 'Limits of maxima (III)'.
\end{enumerate}
\end{Solution}
\begin{Solution}{12.3}
\textbf{Censored lognormal distribution}

First, let's consider the probability of $Y_i=c$:
\begin{equation*}
  Pr(Y_i = c) = Pr(X_i \geq c) = 1 - Pr(X_i \leq c) = 1- F_X(c)
\end{equation*}
The likelihood function is now a mixture between the product of all densities $f_{X}(y_{i}),$ for observations with $Y_{i}<c$%
, times the product of all probabilities that $Y_{i}=c$ for observations with
$Y_{i}=c$:
\begin{equation*}
L(\mu,\sigma; y) = \prod_{i=1}^n\{ f_X(y_i;\mu,\sigma)\}^{\delta_i}\{1-F_X(c;\mu,\sigma)\}^{1-\delta_i}
\end{equation*}
with $\delta_i=1$ for exact observations and $\delta_i=0$ for a censored
observation. The loglikelihood is thus the sum of those two components (let
$n_1$ be the number of non-censored observations and $n_2$ the number of
censored observations, $n_1+n_2=n$):
\begin{equation*}
\log L(\mu,\sigma; y) = \sum_{i=1}^{n_1} \log f_X(y_i;\mu,\sigma) +
\sum_{i=1}^{n_2} \log(1-F_X(c;\mu,\sigma))
\end{equation*}
The code might look like this:
\begin{verbatim}
#########################################
#### Censored lognormal distribution ####
#########################################
# Definition of loglikelihood
neglogl <- function(param,dat,cens) {
  mu <- param[1]
  sigma <- param[2]
  y <- dat
  c <- cens
  z <- sum(log(dlnorm(y[y<c],meanlog=mu,sdlog=sigma)))
  + sum(log(1-plnorm(y[y==c],meanlog=mu,sdlog=sigma)))
  return(-z)
}
# Get data
censoredln <- read.table(file.choose(), header=T, quote="\"")
x <- censoredln$x
# Optimization
obj <- optim(c(0,1),neglogl,dat=x,cens=12,hessian=T)
print(obj$par)  # Point estimates
print(solve(obj$hessian)) # Numerical covariance matrix
\end{verbatim}
\end{Solution}
\begin{Solution}{12.4}
\textbf{Exponential model}

For the log-likelihood you have to consider the distribution of the error
term:
\begin{align*}
  u_i = y_i - exp\{\alpha + \beta x_i\} \sim N(0,\sigma^2)
\end{align*}
\begin{itemize}
\item[1)/2)/3)] The distributional assumption for the likelihood is thus
    the density of the normal distribution ($f_{u_i}$) and the
    log-likelihood is given by
\begin{align*}
  \log L = \sum_{i=1}^n f_u(y_i - exp\{\alpha + \beta x_i\})
\end{align*}
\item[4)] Now the log-likelihood is given by
\begin{align*}
  \log L = -n \log(2) - \sum_{i=1}^n |\alpha + \beta x_i|
\end{align*}
\end{itemize}
The code might look like this:
\begin{verbatim}
###########################
#### Exponential model ####
###########################
# Get data
expgrowth <- read.csv(file.choose())
View(expgrowth)

#1: Definition of negative loglikelihood
neglogl <- function(param,dat){
  a <- param[1]
  b <- param[2]
  sigma <- param[3]
  x <- dat[,2]
  y <- dat[,1]
  u <- y-exp(a+b*x)
  z <- sum(log(dnorm(u,mean=0,sd=sigma)))
  zz <- sum(log(dnorm(y,mean=exp(a+b*x),sd=sigma))) #works as well
  return(-z)
}

#2: Numerical Optimization
# Optimization, for start values one has to consider that the exponential function is very
# sensitive to the parameters a and b -> use small ones
obj <- optim(c(0,0.1,1),neglogl,dat=expgrowth,hessian=T)
print(obj$par)  # Point estimates

#3: Asymptotic numerical covariance matrix
print(solve(obj$hessian))

#4: An exponential density function for the error terms
#Definition of negative loglikelihood
neglogl2 <- function(param,dat){
  a <- param[1]
  b <- param[2]
  x <- dat[,2]
  y <- dat[,1]
  u <- y-exp(a+b*x)
  z <- -n*log(2) - sum(abs(u))     #analytically
  zz <- sum(log(0.5*exp(-abs(u)))) #alternatively
  return(-z) #or return(-zz)
}
obj2 <- optim(c(0,0.1,1),neglogl2,dat=expgrowth,hessian=T)
print(obj2$par)  # Point estimates
\end{verbatim}
\end{Solution}
\begin{Solution}{12.5}
\textbf{Tobit model}

First, let's consider censored observations, for $y_t^*\leq 0$:
\begin{equation*}
  Pr(y_t = 0) = Pr(y_t^* \leq 0) = Pr(u_t\leq -x_t'\beta) = Pr\left(\frac{u_t}{\sigma}\leq \frac{-x_t'\beta}{\sigma}\right) = \Phi\left(\frac{-x_t'\beta}{\sigma}\right)
\end{equation*}
with $\Phi$ being the cdf of the standard normal distribution. For uncensored observations we have a linear model, i.e. the likelihood for $y_t^*> 0$:
\begin{align*}
  f(u_t) = \frac{1}{\sigma\sqrt{2\pi}}e^{\frac{-(y_t^*-x_t'\beta)^2}{2\sigma^2}} = \frac{1}{\sigma} \frac{1}{\sqrt{2\pi}}e^{\frac{-1}{2}\left(\frac{y_t^*-x_t'\beta}{\sigma}\right)} = \frac{1}{\sigma}\phi\left(\frac{y_t^*-x_t'\beta}{\sigma}\right)
\end{align*}
with $\phi$ being the pdf of the standard normal distribution. The likelihood function is now a mixture:
\begin{equation*}
L(\mu,\sigma; y) = \prod_{t=1}^n\left\{ \frac{1}{\sigma}\phi\left(\frac{y_t^*-x_t'\beta}{\sigma}\right)\right\}^{\delta_t}
\left\{\Phi\left(\frac{-x_t'\beta}{\sigma}\right)\right\}^{1-\delta_t}
\end{equation*}
with $\delta_t=1$ for exact observations and $\delta_t=0$ for a censored observation. The loglikelihood is thus the sum of those two components:
\begin{equation*}
\log L(\mu,\sigma; y) = \sum_{y_t>0} \log\left(\frac{1}{\sigma}\phi\left(\frac{y_t^*-x_t'\beta}{\sigma}\right)\right) +
\sum_{y_t=0} \log\left(\Phi\left(\frac{-x_t'\beta}{\sigma}\right)\right)
\end{equation*}
The code might look like this:
\begin{verbatim}
#####################
#### Tobit model ####
#####################
# Definition of negative loglikelihood
neglogl <- function(param,dat) {
  y <- dat$y
  x1 <- dat$x1
  x2 <- dat$x2
  x3 <- dat$x3
  beta1 <- param[1]
  beta2 <- param[2]
  beta3 <- param[3]
  sigm   <- param[4]
  idx_cens <- which(y==0)
  idx_uncens <- which(y>0)
  uncens <- sum(log(1/sigm*dnorm((y[idx_uncens]-x1[idx_uncens]*beta1
                            -x2[idx_uncens]*beta2-x3[idx_uncens]*beta3)/sigm)))
  cens   <- sum(log(pnorm((-x1[idx_cens]*beta1-x2[idx_cens]*beta2-x3[idx_cens]*beta3)/sigm)))
  z <- uncens+cens
  return(-z)
}
# Get data
tobitbsp <- read.csv(file.choose())
# Note that if tobitbsp.csv has negative y values,
# set all negative values to zero, and proceed with the exercise
View(tobitbsp)
tobitbsp$y[tobitbsp$y<0] <- 0
# Optimization
obj <- optim(c(0.2,-0.02,0.005,0.125),neglogl,dat=tobitbsp,hessian=T)
print(obj$par)  # Point estimates
print(diag(solve(obj$hessian))) # Numerical covariance matrix
#OLS
X <- as.matrix(cbind(tobitbsp$x1,tobitbsp$x2,tobitbsp$x3))
y <- as.matrix(tobitbsp$y)
beta_ols <- solve(t(X)%*%X)%*%t(X)%*%y
\end{verbatim}

\begin{itemize}
  \item Compared to Logit or Probit, we do not throw away information in y.
  \item Note that OLS is not a consistent estimator, however Maximum Likelihood is given some regularity conditions always consistent.

\end{itemize}
\end{Solution}
\begin{Solution}{12.6}
\textbf{Probit model}

Since the probability that $y_t=1$ is $\Phi(x_t'\beta)$, the contribution to the loglikelihood function for observation $t$ when $y_t=1$ is $log(\Phi(x_t'\beta))$, that is the log-likelihood function is equal to
\begin{align*}
\sum_{i=1}^{n} \left(y_t log(\Phi(x_t'\beta)) + (1-y_t)log(1-\Phi(x_t'\beta))\right)
\end{align*}

The code might look like this:
\begin{verbatim}
######################
#### Probit model ####
######################
rm(list=ls())
dev.off()
cat("\014")

mroz <- read.csv(file.choose())
inlf <- mroz$inlf
nwifeinc <- mroz$nwifeinc
educ <- mroz$educ
exper <- mroz$exper
exper2 <- exper^2
age <- mroz$age
kidslt6 <- mroz$kidslt6
kidsge6 <- mroz$kidsge6


numTimeVals <- length(nwifeinc)
zuerkl <- inlf
numVar <- 8
erkl <- matrix(0,numTimeVals,numVar)
erkl[,1] <- nwifeinc
erkl[,2] <- educ
erkl[,3] <- exper
erkl[,4] <- exper2
erkl[,5] <- age
erkl[,6] <- kidslt6
erkl[,7] <- kidsge6
erkl[,8] <- rep(1,numTimeVals)

probitLL2 <- function(betas,x,y){
  vals <- x %*% betas
  TT <- length(y)
  summanden <- y*log(pnorm(vals, mean=0, sd=1)) + (1-y)*log(1-pnorm(vals, mean=0, sd=1))
  return (-sum(summanden))
}

beta_anf <- 0

obj2 <- optim(par=rep(beta_anf,numVar),probitLL2,x=erkl, y=zuerkl, hessian=T)
#print(obj)
obj2$par
solve(obj2$hessian)

plot(zuerkl)
points(seq(1,numTimeVals),pnorm(erkl%*%obj2$par),col = 2)

pred <- obj2$par%*%c(30,14,10,10^2,44,0,3,1)
pnorm(pred)

#t value
abs(obj2$par[2]/obj2$hessian[2,2])
# -> significantly different from 0

#simulate data
#datenpunkte
N <- 250

beta1_wahr <- 2
beta0_wahr <- -1.5

testen <- function(N, norm = T){
  #Daten produzieren
  if(norm == T){x <- rnorm(N)
  p <- pnorm(x*beta1_wahr + beta0_wahr  + rnorm(N,0,10^-1))
  }
  else{x <- rnorm(N)
  p <- pnorm(x*beta1_wahr + beta0_wahr  + runif(N,-1,1))
  }
  y <- runif(N)
  y2 <- runif(N)
  y[y2<=p] = 1
  y[y2>p] = 0
  #zu x konstante hinzufügen
  x <- cbind(x,rep(1,N))

  #Parameter schaetzen
  obj2 <- optim(par=c(0,0),probitLL2, x=x, y=y, hessian=T)
  return(obj2$par)
}

sampleSizes <- seq(100,10000,100)
#Inkonsistenz veranschaulichen bei Gleichverteilung
param = NULL
for(numbers in sampleSizes){
  param = rbind(param,testen(numbers,F))
}

par(mfrow=c(2,2))
plot(sampleSizes, param[,1], xlab="Stichprobengröße", ylab="beta1", main="Gleichverteilung",
                                                                                ylim=c(1,3))
 abline(h = beta1_wahr, col=2)

plot(sampleSizes, param[,2],, xlab="Stichprobengröße", ylab="beta0", main="Gleichverteilung",
                                                                                ylim=c(-2,-1))
abline(h = beta0_wahr, col=2)

#Konsistenz veranschaulichen
param = NULL
for(numbers in sampleSizes){
  param = rbind(param,testen(numbers,T))
}

plot(sampleSizes, param[,1], xlab="Stichprobengröße", ylab="beta1", main="Normalverteilung",
                                                                                ylim=c(1,3))
abline(h = beta1_wahr, col=2)

plot(sampleSizes, param[,2],, xlab="Stichprobengröße", ylab="beta0", main="Normalverteilung",
                                                                                ylim=c(-2,-1))
abline(h = beta0_wahr, col=2)
\end{verbatim}
\begin{itemize}
  \item Probit models are used to model binary outcomes, such as consumption decisions, labor decisions, agricultural decisions,\dots, and are among the most used models in applied econometrics.
  \item The probit model forecasts probabilities, which are in between 0 and 1: $Pr(Y=1|x)=F(x'\beta)$
  \item Interpretation of coefficients:
  \begin{itemize}
    \item Increase in $x$ makes outcome of 1 more (or less) likely, we therefore interpret the sign of the coefficients.
    \item Marginal effects at a specific $x$ (most commonly at mean $\bar{x}$): $\frac{\partial p}{\partial x_j} = \phi(x'\beta)\beta_j$. Note that coefficients and marginal effects have the same sign.
    \item Average marginal effects are computed over all $x$.
  \end{itemize}
  \item Probit and Logit have usually very similar marginal effects, which model depends on an evaluation of the data generating process. Most of the times it doesn't matter though.
\end{itemize}
\end{Solution}
\begin{Solution}{12.7}
\textbf{Logit model}

Since the probability that $y_t=1$ is $\Lambda \left( x_{t}^{\prime }\beta \right)$, the contribution to the loglikelihood function for observation $t$ when $y_t=1$ is $log(\Lambda \left( x_{t}^{\prime }\beta \right))$, that is the log-likelihood function is equal to
\begin{align*}
\sum_{i=1}^{n} \left(y_t log(\Lambda \left( x_{t}^{\prime }\beta \right)) + (1-y_t)log(1-\Lambda \left( x_{t}^{\prime }\beta \right)))\right)
\end{align*}

The code might look like this:
\begin{verbatim}
#####################
#### Logit model ####
#####################
rm(list=ls())
dev.off()
cat("\014")

mroz <- read.csv(file.choose())
inlf <- mroz$inlf
nwifeinc <- mroz$nwifeinc
educ <- mroz$educ
exper <- mroz$exper
exper2 <- exper^2
age <- mroz$age
kidslt6 <- mroz$kidslt6
kidsge6 <- mroz$kidsge6


numTimeVals <- length(nwifeinc)
zuerkl <- inlf
numVar <- 8
erkl <- matrix(0,numTimeVals,numVar)
erkl[,1] <- nwifeinc
erkl[,2] <- educ
erkl[,3] <- exper
erkl[,4] <- exper2
erkl[,5] <- age
erkl[,6] <- kidslt6
erkl[,7] <- kidsge6
erkl[,8] <- rep(1,numTimeVals)

neglogitloglik <- function(betas,x,y){
  vals <- x %*% betas
  TT <- length(y)
  summanden <- y*log(plogis(vals)) + (1-y)*log(1-plogis(vals))
  return (-sum(summanden))
}

beta_anf <- 0

obj2 <- optim(par=rep(beta_anf,numVar),neglogitloglik,x=erkl, y=zuerkl, hessian=T)
print(obj2$par)
solve(obj2$hessian)

pred <- obj2$par%*%c(30,14,10,10^2,44,0,3,1)
plogis(pred)

#t value
abs(obj2$par[2]/obj2$hessian[2,2])
# -> significantly different from 0
\end{verbatim}
\begin{itemize}
  \item Logit models are used to model binary outcomes, such as consumption decisions, labor decisions, agricultural decisions,\dots, and are among the most used models in applied econometrics.
  \item The logit model forecasts probabilities, which are in between 0 and 1: $Pr(Y=1|x)=F(x'\beta)$
  \item Interpretation of coefficients:
  \begin{itemize}
    \item Increase in $x$ makes outcome of 1 more (or less) likely, we therefore interpret the sign of the coefficients.
    \item Marginal effects at a specific $x$ (most commonly at mean $\bar{x}$): $\frac{\partial p}{\partial x_j} = \frac{e^{x'\beta}}{(1+e^{x'\beta})^2}\beta_j$. Note that coefficients and marginal effects have the same sign.
    \item Average marginal effects are computed over all $x$.
  \end{itemize}
  \item Probit and Logit have usually very similar marginal effects, which model depends on an evaluation of the data generating process. Most of the times it doesn't matter though.
\end{itemize}
\end{Solution}
\begin{Solution}{12.8}
\textbf{Heckman regression}

\begin{enumerate}
\item Each observation contributes a factor to the likelihood function:
\begin{align*}
I(z_t=0)Pr(z_t=0) + I(z_t=1)Pr(z_t=1)f(y_t^*|z_t=1)
\end{align*}
where $I(z_t=0)$ is the indicator function and $f(y_t^*|z_t=1)$ denotes the density of $y_t^*$ conditional on $z_t=1$. The loglikelihood function is
\begin{align*}
\sum_{z_t=0}log Pr(z_t=0) + \sum_{z_t=1}log Pr(z_t=1)f(y_t^*|z_t=1) = \sum_{z_t=0}log Pr(z_t=0) + \sum_{z_t=1}log\left(Pr(z_t=1|y_t^*)f(y_t^*)\right)
\end{align*}
where $f(y_t^*)$ is the normal density with mean $X_t\beta$ and variance $\sigma^2$. The first term, where $z_t=0$, is the same as in a probit model.

Lets calculate $Pr(z_t=1|y_t^*)$. Since $u_t$ and $v_t$ are bivariate normal, we can write $v_t=\rho u_t/\sigma + \varepsilon_t$, where $\varepsilon_t$ is a normally distributed random variable with mean 0 and variance $(1-\rho^2)$. Thus
\begin{align*}
z_t^* = W_t\gamma+\rho(y_t^*-X_t\beta)/\sigma + \varepsilon_t
\end{align*}
Because $y_t=y_t^*$ for $z_t=1$, it follows that
\begin{align*}
Pr(z_t=1|y_t^*) = \Phi\left(\frac{W_t\gamma + \rho(y_t-X_t\beta)/\sigma}{(1-\rho^2)^{1/2}}\right)
\end{align*}
Combining everything the loglikelihood is given by
\begin{align*}
\sum_{z_t=0}log\Phi(-W_t\gamma) + \sum_{z_t=1}log\left(\frac{1}{\sigma}\phi\left((y_t-X_t\beta)/\sigma\right)\right) + \sum_{z_t=1} log \Phi\left(\frac{W_t\gamma + \rho(y_t-X_t\beta)/\sigma}{(1-\rho^2)^{1/2}}\right)
\end{align*}

\item Make variable in-labor-force if wage is observed:
\begin{verbatim}
library(foreign); womanwk <- read.dta(file.choose());
womanwk$inlf <- rep(1,dim(womanwk)[1]); womanwk$inlf[is.na(womanwk$wage)] <- 0
\end{verbatim}

\item In the underlying model women choose whether to work and if they do, we observe their wages. If women made this decision randomly, we could ignore that not all wages are observed and use ordinary regression to fit a wage model. Such an assumption of random participation, however, is unlikely to be true; women who would have low wages may be unlikely to choose to work, and thus the sample of observed wages is biased upward. In the jargon of economics, women choose not to work when their personal reservation wage is greater than the wage offered by employers. Thus women who choose not to work might have even higher offer wages than those who do work—they may have high offer wages, but they have even higher reservation wages. We could tell a story that competency is related to wages, but competency is rewarded more at home than in the labor force In any case, in this problem—which is the paradigm for most such problems—a solution can be found if there are some variables that strongly affect the chances for observation (the reservation wage) but not the outcome under study (the offer wage). Such a variable might be the number of children in the home. (Theoretically, we do not need such identifying variables, but without them, we depend on functional form to identify the model. It would be difficult for anyone to take such results seriously because the functional form assumptions have no firm basis in theory.)

\item The code might look like this
\begin{verbatim}
################################################
#### Sample selectivity: Heckman regression ####
################################################
rm(list=ls())
dev.off()
cat("\014")
library(foreign)
womanwk <- read.dta(file.choose())
# make variable in-labor-force if wage is observed
womanwk$inlf <- rep(1,dim(womanwk)[1])
womanwk$inlf[is.na(womanwk$wage)] <- 0

negloglik <- function(param,dat){
  y <- as.matrix(dat$wage)
  X <- as.matrix(cbind(1,dat$educ, dat$age))
  z <- as.matrix(dat$inlf)
  W <- as.matrix(cbind(1, dat$married, dat$children, dat$age, dat$educ))

  bet <- param[1:dim(X)[2]]
  gam <- param[(dim(X)[2]+1):(dim(X)[2]+dim(W)[2])]
  rho_aux <- param[dim(X)[2]+dim(W)[2]+1] #rho_aux is unconstrained
  #rho <- -1 + 0.5*(1+1)*(1+tanh(rho_aux)) #this constraints rho to be between -1 and 1
  rho <- (exp(2*rho_aux)-1)/(1+exp(2*rho_aux)) #this constraints rho to be between -1 and 1
  sigm_aux <- param[dim(X)[2]+dim(W)[2]+2]#sigm_aux is unconstrained
  sigm <- exp(sigm_aux) #sigm is positive
  X_bet <- X%*%as.matrix(bet)
  W_gam <- W%*%as.matrix(gam)
  sum1 <- sum(log(pnorm(-W_gam[z==0])))
  sum2 <- sum(log(1/sigm*dnorm((y[z==1]-X_bet[z==1])/sigm)))
  sum3 <- sum(log(pnorm( (W_gam[z==1]+rho*(y[z==1]-X_bet[z==1])/sigm)/(sqrt(1-rho^2)) )))
  return (-(sum1+sum2+sum3))
}

#Finding good starting values is hard, an alternative method is Heckman's two-step method
#Step1: Probit regression of selection equation
y <- as.matrix(womanwk$wage)
X <- as.matrix(cbind(1,womanwk$educ, womanwk$age))
z <- as.matrix(womanwk$inlf)
W <- as.matrix(cbind(1, womanwk$married, womanwk$children, womanwk$age, womanwk$educ))
step1 <- glm(z ~ -1 + W, family=binomial(link="probit"))
gam_anf <- coef(step1)
#Step 2: OLS regression with inverse Mills ratio as additional regressor
aux <- dnorm(W%*%as.matrix(gam_anf))/pnorm(W%*%as.matrix(gam_anf))
step2 <- lm(y ~ -1 + X + aux)
bet_anf <- coef(step2)[1:dim(X)[2]]
rho_anf <- c(0.7)
sigm_anf <- coef(step2)[dim(X)[2]+1]/rho_anf
par_anf <- c(bet_anf,gam_anf,rho_anf,sigm_anf); names(par_anf) <- NULL;

#Maximum Likelihood with constraints: rho in (-1;1) and sigm>0
#par_anf[9] <- tanh( 2*(par_anf[9]+1)/(1+1)-1) #rescale rho
par_anf[9] <- 1/2*log((1+par_anf[9])/(1-par_anf[9])) #rescale rho
par_anf[10] <- log(par_anf[10]) #rescale sigm
obj <- optim(par=par_anf,negloglik,dat=womanwk, hessian=T)
obj$value

#Show parameters, rescale rho and sigm
estimates <- obj$par
#estimates[9] <- -1 + 0.5*(1+1)*(1+tanh(estimates[9]))
estimates[9] <- (exp(2*estimates[9])-1)/(1+exp(2*estimates[9]))
estimates[10] <- exp(estimates[10])
print(estimates)
diag(solve(obj$hessian))
\end{verbatim}

\end{enumerate}
\end{Solution}
\begin{Solution}{12.9}
\textbf{Count data}

The loglikelihood function is
\begin{align*}
\sum_{i=1}^{n}\left(-exp(x_i\beta) + y_i X_i\beta - log(y_i!)\right)
\end{align*}
The code might look like this:
\begin{verbatim}
####################
#### Count data ####
####################
rm(list=ls()); dev.off(); cat("\014");

players <- read.csv(choose.files())

## Negative Log-likelihood
neg_ll <- function(bet,dat){
  position <- dat$position
  age <- dat$age
  training <- dat$training
  salary <- dat$salary
  bonus <- dat$bonus
  y <- dat$goals
  X <- cbind(1,position,age,age^2,training,salary,bonus)
  ll <- sum(-exp(X%*%bet)+y*X%*%bet-log(factorial(y)))
  return(-ll)
}
#Log-likelihood maximieren
out <- optim(par=c(-1,0.01,0.01,-0.01,0.01,0.01,0.01), fn=neg_ll,dat=players,hessian=T)
par <- out$par

## Standardabweichung der einzelnen Parameter
sdbeta <- round(sqrt(diag(solve(out$hessian))),8)
sdbeta

#Wahrscheinlichkeit, über drei Tore
mu <- (1*par[1]+1*par[2]+25*par[3]+25^2 * par[4] + 15*par[5]+ 700 *
         par[6] + 0)

1-ppois(3,mu)
#0.001064403
# Sehr unwahrscheinlich unter diesen Gegebenheiten über drei Tore zu schießen.
\end{verbatim}
\end{Solution}
\begin{Solution}{12.10}
\textbf{Stochastic frontier analysis}

\begin{enumerate}
\item The density of the merged error term $\varepsilon $ is%
\begin{eqnarray}
f_{\varepsilon }\left( x\right)  &=&f_{v-u}\left( x\right)   \notag \\
&=&\int_{0}^{\infty }f_{v}\left( u+x\right) \cdot f_{u}\left( u\right) du
\notag \\
&=&\int_{0}^{\infty }\phi \left( \frac{u+x}{\sigma }\right) \cdot \lambda
e^{-\lambda u}du  \notag \\
&=&\int_{0}^{\infty }\frac{1}{\sqrt{2\pi }}e^{-\frac{1}{2}\left( \frac{u+x}{%
\sigma }\right) ^{2}}\lambda e^{-\lambda u}du  \notag \\
&=&\frac{\lambda }{\sqrt{2\pi }}\int_{0}^{\infty }\exp \left( -\frac{1}{2}%
\left( \frac{u+x}{\sigma }\right) ^{2}-\lambda u\right) du.  \label{eq1}
\end{eqnarray}%
The integral in (\ref{eq1}) can be manipulated in a way similar to exercise %
\ref{moments}.2,%
\begin{eqnarray*}
&&\int_{0}^{\infty }\exp \left( -\frac{1}{2}\left( \frac{u+x}{\sigma }%
\right) ^{2}-\lambda u\right) du \\
&=&\int_{0}^{\infty }\exp \left( -\frac{u^{2}+2u\left( x+\lambda \sigma
^{2}\right) +x^{2}}{2\sigma ^{2}}\right) du \\
&=&\int_{0}^{\infty }\exp \left( -\frac{u^{2}+2u\left( x+\lambda \sigma
^{2}\right) +\left( x+\lambda \sigma ^{2}\right) ^{2}+x^{2}-\left( x+\lambda
\sigma ^{2}\right) ^{2}}{2\sigma ^{2}}\right) du \\
&=&\int_{0}^{\infty }\exp \left( -\frac{\left( u+\left( x+\lambda \sigma
^{2}\right) \right) ^{2}+x^{2}-\left( x+\lambda \sigma ^{2}\right) ^{2}}{%
2\sigma ^{2}}\right) du \\
&=&\exp \left( -\frac{x^{2}-\left( x+\lambda \sigma ^{2}\right) ^{2}}{%
2\sigma ^{2}}\right) \int_{0}^{\infty }\exp \left( -\frac{1}{2}\left( \frac{%
u-\left( -x-\lambda \sigma ^{2}\right) }{\sigma }\right) ^{2}\right) du.
\end{eqnarray*}%
Substituting the integral in (\ref{eq1}) we arrive at%
\begin{eqnarray*}
f_{\varepsilon }\left( x\right)  &=&\lambda \exp \left( -\frac{x^{2}-\left(
x+\lambda \sigma ^{2}\right) ^{2}}{2\sigma ^{2}}\right)  \\
&&\times \int_{0}^{\infty }\frac{1}{\sqrt{2\pi }}\exp \left( -\frac{1}{2}%
\left( \frac{u-\left( -x-\lambda \sigma ^{2}\right) }{\sigma }\right)
^{2}\right) du.
\end{eqnarray*}%
The first factor can be simplified to%
\begin{equation*}
\lambda \exp \left( -\frac{x^{2}-\left( x+\lambda \sigma ^{2}\right) ^{2}}{%
2\sigma ^{2}}\right) =\lambda \exp \left( \lambda x+\frac{\sigma ^{2}\lambda
^{2}}{2}\right) ,
\end{equation*}%
and the integrand in the second factor is simply the density of a normal
distribution with mean $\left( -x-\lambda \sigma ^{2}\right) $ and variance $%
\sigma ^{2}$. Thus the value of the integral can be derived from the cdf of $%
N(0,1)$ as follows,%
\begin{equation*}
\int_{0}^{\infty }\frac{1}{\sqrt{2\pi }}\exp \left( -\frac{1}{2}\left( \frac{%
u-\left( -x-\lambda \sigma ^{2}\right) }{\sigma }\right) ^{2}\right) du=\Phi
\left( \frac{-x-\lambda \sigma ^{2}}{\sigma }\right) .
\end{equation*}%
In sum, the density of $\varepsilon $ is%
\begin{equation*}
f_{\varepsilon }\left( x\right) =\lambda \exp \left( \lambda x+\frac{\sigma
^{2}\lambda ^{2}}{2}\right) \Phi \left( \frac{-x-\lambda \sigma ^{2}}{\sigma
}\right)
\end{equation*}%
and its logarithm is%
\begin{equation*}
\ln f_{\varepsilon }\left( x\right) =\ln \lambda +\lambda x+\frac{\sigma
^{2}\lambda ^{2}}{2}+\ln \Phi \left( \frac{-x-\lambda \sigma ^{2}}{\sigma }%
\right) .
\end{equation*}

\item The code might look like this:
\begin{verbatim}
######################################
#### Stochastic Frontier Analysis ####
######################################

##Log-Likelihood Funktion, die maximiert werden soll
neg_log_likeli <- function(theta, daten){
  A <- theta[1]
  alpha <- theta[2]
  beta <- theta[3]
  lambda <- theta[4]
  sigma <- theta[5]

  x1 <- daten[,3]
  x2 <- daten[,4]
  y <- daten[,2]

  logA <- log(A)
  logy <- log(y)
  logx1 <- log(x1)
  logx2 <- log(x2)

  eps <- logy - logA - alpha*logx1 - beta*logx2

  log_likeli <- sum(log(lambda*exp(lambda*eps+lambda^2/2*sigma^2)*pnorm(
                                  -eps/sigma-lambda*sigma,mean=0,sd=1)))
  return(-log_likeli)
}

daten <- read.csv(file.choose())
head(daten)

##Schätzung der Parameter
opt <- optim(c(7, 0.3, 0.7, 10, 3), neg_log_likeli, daten=daten, hessian=T)
opt

A <- opt$par[1]
alpha <- opt$par[2]
beta <- opt$par[3]
lambda <- opt$par[4]
sigma <- opt$par[5]

##Standardfehler: Problem mit negativen Werten, hier kleine Stichprobe das Problem
diag(solve(opt$hessian))

##Geschätzte Ineffizienzen
epsilon <- log(daten$ValueAdd)-log(A)-alpha*log(daten$Capital)-beta*log(daten$Labor)
plot(epsilon)
hist(epsilon, freq=F)
curve(lambda*exp(lambda*x+lambda^2/2*sigma^2)*pnorm(-x/sigma-lambda*sigma, mean=0, sd=1),
                                                        from=-1, to=1, add=T, col="red")
\end{verbatim}
\end{enumerate}

\begin{itemize}
  \item SFA are also called \emph{composed error models}. In a production context efficiency loss is modelled as $-u$, in a cost context as $+u$.
  \item SFA models are also used to measure efficiency of the banking system, level of competitiveness of a market, quality of inputs and outputs, regulation, management evaluation,\dots
  \item Basic idea is that the ratio of observed output to maximum possible output is less or equal to 1.
  \item How to estimate inefficiency?
  \begin{itemize}
    \item We can compute residuals $\hat{\varepsilon_t} = \widehat{u_t+v_t}$. Since $E(v_t)=0$ we can conclude that if the residual is high, then so is the inefficiency.
    \item Jondrow, Lovell, Materov, Schmidt (JLMS) approach: Look at the mean (or mode) of the conditional distribution.
  \end{itemize}
\end{itemize}
\end{Solution}
\begin{Solution}{12.11}
\textbf{ARCH models}

For ARCH models we have:
\begin{align*}
E(X_t|X_{t-1}) = E(\varepsilon_t \sigma_t|X_{t-1}) = E(\varepsilon_t\sqrt{\omega+\alpha X_{t-1}^2}|X_{t-1}) =\sqrt{\omega+\alpha X_{t-1}^2} E(\varepsilon_t|X_{t-1}) = 0
\end{align*}
and
\begin{align*}
Var(X_t|X_{t-1}) = Var(\varepsilon_t \sigma_t|X_{t-1}) = Var(\varepsilon_t\sqrt{\omega+\alpha X_{t-1}^2}|X_{t-1}) = (\omega+\alpha X_{t-1}^2)\underbrace{E(\varepsilon_t^2|X_{t-1})}_{=1} = \omega+\alpha X_{t-1}^2
\end{align*}
since $\varepsilon_t$ is independent of $X_{t-1}$. Given that $\varepsilon_t \sim N(0,1)$, the conditional density of $X_t$ given $X_{t-1}$ is
\begin{align*}
X_t|X_{t-1} \sim N(0,\omega+\alpha X_{t-1}^2)
\end{align*}
In general we can factorize a joint likelihood function:
\begin{align*}
  f_{X_1,\dots,X_T}(x_1,\dots,x_T) = \prod_{t=1}^{T}f_{X_t|X_{t-1},\dots,X_1}(x_t|X_{t-1,\dots,x_1})
\end{align*}
In our case:
\begin{align*}
  f_{X_1,\dots,X_T}(x_1,\dots,x_T) = f_{X_1}(x_1)\prod_{t=2}^{T}\frac{1}{\sqrt{2\pi(\omega+\alpha x_{t-1}^2)}}\exp\left(-\frac{1}{2}\left(\frac{x_t}{\omega+\alpha x_{t-1}^2}\right)^2\right)
\end{align*}
We can ignore $f_{X_1}(x_1)$ for large sample sizes, since it contribution is relatively small the larger the sample size. The log-likelihood is then given by
\begin{align*}
  -\frac{T-2}{2}log(2\pi)-\frac{1}{2}\sum_{t=2}^{T}log(\omega+\alpha x_{t-1}^2) - \frac{1}{2} \sum_{t=2}^{T} \left(\frac{x_t}{\omega+\alpha x_{t-1}^2}\right)^2
\end{align*}
The code might look like this
\begin{verbatim}
#####################
#### ARCH Models ####
#####################

arch1_ll <- function(theta, tseries) {
  omega <- theta[1]
  alpha <- theta[2]
  Te <- length(tseries)
  return((Te - 1) / 2 * log(2 * pi)
         + 0.5 * sum(log(omega + alpha * tseries[1:(Te - 1)]^2))
         + 0.5 * sum((tseries[2:Te]^2 /
                        (omega + alpha * tseries[1:(Te - 1)]^2))))
}

arch1bsp <- read.csv(file.choose(), header = TRUE)

estimate <- optim(par = c(0.2, 0.9), fn = arch1_ll, tseries = arch1bsp$x,
                  hessian = TRUE)
estimate$par
solve(estimate$hessian)

gitter <- seq(0.05, 0.95, by = 0.01)
value_omega <- rep(NA, length(gitter))
value_alpha <- rep(NA, length(gitter))
for (i in 1:length(gitter)) {
  value_omega[i] <- -arch1_ll(c(gitter[i], estimate$par[2]),
                              tseries = arch1bsp$x)
  value_alpha[i] <- -arch1_ll(c(estimate$par[1], gitter[i]),
                              tseries = arch1bsp$x)
}
plot(gitter, value_omega, type = "l", xlab = expression(omega),
     ylab = "",
     main = expression(paste("log-Likelihood in Abhängigkeit von ", omega,
                             ", ", alpha, " = 0.9376")), lwd = 2,
     cex.axis = 1.5, cex.lab = 2, cex.main = 2)
mtext(text = expression(paste("logL(", omega, ")")), side = 2, cex = 2,
      line = 2.2)
abline(h = -estimate$value, col = "red")
abline(v = estimate$par[1], col = "red")

plot(gitter, value_alpha, type = "l", xlab = expression(alpha),
     ylab = "",
     main = expression(paste("log-Likelihood in Abhängigkeit von ", alpha,
                             ", ", omega, " = 0.2056")), lwd = 2,
     cex.axis = 1.5, cex.lab = 2, cex.main = 2)
mtext(text = expression(paste("logL(", alpha, ")")), side = 2, cex = 2,
      line = 2.2)
abline(h = -estimate$value, col = "red")
abline(v = estimate$par[2], col = "red")

# Unbedingte Varianz
estimate$par[1] / (1 - estimate$par[2])
\end{verbatim}
\begin{itemize}
\item Volatility is not observable. To measure volatility by the empirical standard deviation is only valid if volatility is relatively stable, but usually we have time-variation in volatility.
\item Idea of ARCH is that volatility is dependent on past observations (conditional distribution is time-varying), however, the unconditional distribution is still stationary.
\end{itemize}
\end{Solution}
\begin{Solution}{12.12}
\textbf{Duration models}

\begin{enumerate}
\item Obtaining the survivor function is very easy:
\begin{align*}
  S(t) \equiv 1-F(t) = exp(-(\theta t)^\alpha)
\end{align*}
For the pdf
\begin{align*}
  f(t) = \frac{\partial F(t)}{\partial t} = \alpha \theta^\alpha t^{\alpha-1} exp(-(\theta t)^\alpha)
\end{align*}
The hazard function is then given by
\begin{align*}
h(t) \equiv \frac{f(t)}{S(t)} = \frac{ \alpha \theta^\alpha t^{\alpha-1} exp(-(\theta t)^\alpha)}{exp(-(\theta t)^\alpha)} = \alpha \theta^\alpha t^{\alpha-1}
\end{align*}
When $\alpha=1$, the Weibull distribution collapses to the exponential and the hazard is just a constant (duration independent). For $\alpha<1$, the hazard is decreasing over time (negative duration dependence) and for $\alpha>1$ it is increasing (positive duration dependence).
\item Taking the log of $f(t_i)$ yields
\begin{align*}
  log(f(t_i)) = log\left(h(t_i)\right) + log\left(S(t_i)\right) = log(\alpha_i) +\alpha_i log(\theta_i) + (\alpha_i-1)log(t_i) -(\theta_i t_i)^\alpha_i
\end{align*}
Since $\theta_i = exp(X_i\beta)$ and $\alpha_i=\alpha$ for all i, we have
\begin{align*}
  log(f(t_i,X_i,\alpha,\beta)) = log(\alpha) +\alpha X_i \beta + (\alpha-1)log(t_i) - t_i^\alpha \cdot exp(\alpha X_i \beta)
\end{align*}
Summing over all n independent observations gives the log-likelihood
\begin{align*}
log(f(t,X,\alpha,\beta)) = n log(\alpha) + \alpha \sum_{i=1}^{n} X_i\beta + (\alpha-1) \sum_{i=1}^{n}log(t_i) - \sum_{i=1}^{n} t_i^\alpha \cdot exp(\alpha X_i \beta)
\end{align*}
Since data sets contain observations for which $t_i$ is not actually observed (e.g. sample of people who entered unemployment at various points in time, then it is extremely likely that some peaple in the sample were still unemployed when data collection ended). We can deal with this censoring, i.e. the observed $t_i$ is the duration of an incomplete spell, it is the logarithm of the probability of censoring, which is the probability that the duration exceed $t_i$, that is, the log of the survivor function. Denote U as the set of $n_u$ uncensored observations, the loglikelihood function for the entire sample is then
\begin{align*}
log(f(t,X,\alpha,\beta)) = n_u log(\alpha) + \alpha \sum_{i\in U} X_i\beta + (\alpha-1) \sum_{i\in U}log(t_i)  - \sum_{i=1}^{n} t_i^\alpha \cdot exp(\alpha X_i \beta)
\end{align*}
Uncensored observations contribute to both terms, while censored observations contribute only to the Survivor function.
\item The code might look like this
\begin{verbatim}
#########################
#### Duration models ####
#########################

## Load data
spells <- read.csv(file.choose())
View(spells)

##Negative log-likelihood
neg_log_likeli <- function(param, dat) {
  beta <- as.matrix(param[1:3])
  alpha <- param[4]

  t <- dat$duration
  const <- dat$const
  x1 <- dat$X1
  x2 <- dat$X2
  X <- as.matrix(cbind(const,x1,x2))
  idx_uncens <- which(t<0.5)
  nu <- length(idx_uncens)
  log_likeli <- nu*log(alpha) + alpha*sum(X[idx_uncens,]%*%beta)
            + (alpha-1)*sum(log(t[idx_uncens])) - sum((t^alpha)*exp(alpha*X%*%beta))
  return(-log_likeli)
}

##Schätzung der Parameter
par_anf <- c(1,0.7,0.3, 1.3)
neg_log_likeli(par_anf,spells)
opt <- optim(par=par_anf, neg_log_likeli, dat=spells, hessian=T)
print(opt$par)

##Standardfehler
diag(solve(opt$hessian))
\end{verbatim}
\end{enumerate}
\begin{itemize}
\item The terminology like survival or hazard is due to evolutionary concepts, however, duration models in economics are quite often found in the labor market literature. Also: strike duration, state of being single (until marriage), etc.
\end{itemize}
\end{Solution}
\begin{Solution}{12.13}
\textbf{Ultra-high-frequency data}
\end{Solution}
\begin{Solution}{12.14}
\textbf{Spatial dependence }

\begin{enumerate}
\item The model can be written as%
\begin{eqnarray*}
\left( I-\rho W\right) y &=&\alpha +\delta z+u \\
y &=&\left( I-\rho W\right) ^{-1}\left( \alpha +\delta z\right) +\left(
I-\rho W\right) ^{-1}u.
\end{eqnarray*}%
To keep the notation short, define $D=\left( I-\rho W\right) ^{-1}$. Since $u
$ is multivariate normal, $u\sim N(0,\sigma ^{2}I_{n})$, we find that%
\begin{equation*}
y\sim N\left( \mu ,\Sigma \right)
\end{equation*}%
with%
\begin{eqnarray*}
\mu  &=&D\left( \alpha +\delta z\right)  \\
\Sigma  &=&\sigma ^{2}DD^{\prime }
\end{eqnarray*}

\item The joint density of $y$, i.e. the likelihood function, is%
\begin{eqnarray*}
L\left( \alpha ,\delta ,\rho ,\sigma \right)  &=&(2\pi )^{-n/2}\left[ \det (%
\mathbf{\Sigma })\right] ^{-1/2}\cdot \exp \left( -\frac{1}{2}\left( \mathbf{%
x}-\mathbf{\mu }\right) ^{\prime }\mathbf{\Sigma }^{-1}\left( \mathbf{x}-%
\mathbf{\mu }\right) \right)  \\
&=&(2\pi )^{-n/2}\left[ \det (\sigma ^{2}DD^{\prime })\right] ^{-1/2} \\
&&\times \exp \left( -\frac{1}{2}\left( y-D\left( \alpha +\delta z\right)
\right) ^{\prime }\left( \sigma ^{2}DD^{\prime }\right) ^{-1}\left(
y-D\left( \alpha +\delta z\right) \right) \right) .
\end{eqnarray*}%
Consider the terms in turn,%
\begin{eqnarray*}
\left[ \det (\sigma ^{2}DD^{\prime })\right] ^{-1/2} &=&\left[ \sigma
^{2n}\det \left( DD^{\prime }\right) \right] ^{-1/2} \\
&=&\left( \sigma ^{2}\right) ^{-n/2}\det \left( D\right) ^{-1} \\
&=&\left( \sigma ^{2}\right) ^{-n/2}\det (I-\rho W)
\end{eqnarray*}%
since $\det (D)=\det (D^{\prime })$ and $\det (D)^{-1}=\det (I-\rho W)$.
Next, the term inside the exponential function is%
\begin{eqnarray*}
&&-\frac{1}{2}\left( y-D\left( \alpha +\delta z\right) \right) ^{\prime
}\left( \sigma ^{2}DD^{\prime }\right) ^{-1}\left( y-D\left( \alpha +\delta
z\right) \right)  \\
&=&-\frac{1}{2\sigma ^{2}}\left( y-D\left( \alpha +\delta z\right) \right)
^{\prime }D^{-1\prime }D^{-1}\left( y-D\left( \alpha +\delta z\right)
\right)  \\
&=&-\frac{1}{2\sigma ^{2}}\left[ D^{-1}\left( y-D\left( \alpha +\delta
z\right) \right) \right] ^{\prime }\left[ D^{-1}\left( y-D\left( \alpha
+\delta z\right) \right) \right]  \\
&=&-\frac{1}{2\sigma ^{2}}\left[ D^{-1}y-\alpha -\delta z\right] ^{\prime }%
\left[ D^{-1}y-\alpha -\delta z\right]  \\
&=&-\frac{\left( y-\rho Wy-\alpha -\delta z\right) ^{\prime }\left( y-\rho
Wy-\alpha -\delta z\right) }{2\sigma ^{2}}.
\end{eqnarray*}%
Hence, the log-likelihood function is%
\begin{eqnarray*}
\ln L\left( \alpha ,\delta ,\rho ,\sigma \right)  &=&-\frac{n}{2}\ln \left(
2\pi \sigma ^{2}\right) +\ln \left( \det \left( I-\rho W\right) \right)  \\
&&-\frac{\left( y-\rho Wy-\alpha -\delta z\right) ^{\prime }\left( y-\rho
Wy-\alpha -\delta z\right) }{2\sigma ^{2}}.
\end{eqnarray*}

\item (...)
\end{enumerate}

The code might look like this:
\begin{verbatim}
############################
#### Spatial dependence ####
############################
# 3)
# Einlesen der Datensätze
spatial = read.csv(file.choose(), header = T, sep = "")
neighbour = read.csv(file.choose(), header = T, sep = "")

# 4)
# Normalisieren der Nachbarschaftsmatrix
W.norm = neighbour / rowSums(neighbour)

# 5)
# Log-Likelihood-Funktion bestimmen und optimieren
ll_Spatial = function(y, z, W, theta){
  W = as.matrix(W)
  n = length(y)
  rho = theta[1]
  alpha = theta[2]
  delta = theta[3]
  sigma = theta[4] # hier sigma^2
  In = diag(n)
  A = t(y - rho*W%*%y - alpha - delta*z)%*%(y - rho*W%*%y - alpha - delta*z)
  loglik = -n/2*log(2*pi*sigma) + log(det(In - rho*W)) - A/(2*sigma)
  return(loglik)
}

optModel = optim(c(0.5 ,1 ,2 ,3) , ll_Spatial , y = spatial$HPrice07 ,
                 z =spatial$HHIncome07 , W = W.norm, hessian = T,
                 method = "L-BFGS-B",lower = c(-1,-Inf ,-Inf ,0),
                 upper = c(1, Inf , Inf , Inf), control = list(fnscale = -1))

# optimale Parameter des Modells
(param <- optModel$par)

# 6)
# Test auf signifikante spatiale Autokorrelation
# Hessematrix
(H = optModel$hessian)

# Geschätzte Kovarianzmatrix
(covmat <- -solve(H))

# Berechnen der Standardfehler
(errors = sqrt(diag(covmat)))

# t-Test auf signifikante Autokorrelation
param[1]/errors[1]
# Eindeutig größer als das 1-alpha/2 Quantil der N(0,1)-Verteilung (1.96)
# --> zum Niveau 5% ein von 0 verschiedene räumliche Autokorrelation
\end{verbatim}
\end{Solution}
\begin{Solution}{13.1}
\textbf{The miracle of the instruments}
The code might look like this:
\begin{verbatim}
########################################
#### The miracle of the instruments ####
########################################
  #1)
  library(MASS)
  library(AER)

  # Generate data: both exogenous variables are correlated with u
  Omega <- matrix(c(
  2,0.3,0.5,0.7,
  0.3,1,0.5,0.7,
  0.5,0.5,1,0,
  0.7,0.7,0,1),nrow=4,ncol=4)

  n <- 100
  dat <- mvrnorm(n,c(5,5,0,5),Omega)
  x1 <- dat[,1]
  x2 <- dat[,2]
  u <- dat[,3]
  w <- dat[,4] # instrument
  y <- 1+2*x1+3*x2+u
  obj <- ivreg(y~x1+x2|w) #does not work

  #2)
  R <- 1000
  Z <- matrix(NA,nrow=R,ncol=3)
  ZZ <- matrix(NA,nrow=R,ncol=3)
  for(r in 1:R) {
    dat <- mvrnorm(n,c(5,5,0,5),Omega)
    x1 <- dat[,1]
    x2 <- dat[,2]
    u <- dat[,3]
    w <- dat[,4] # instrument
    y <- 1+2*x1+3*x2+u

    #OLS is inconsistent
    ols <- lm(y~x1+x2)

    # IV from AER-package
    w2 <- w^2
    w3 <- w^3
    obj <- ivreg(y~x1+x2|w2+w3)

    #store estimates
    Z[r,] <- coefficients(obj)
    ZZ[r,] <- coefficients(ols)
    }

  print(apply(Z,2,median))
  print(apply(Z,2,mean))
  print(apply(Z,2,sd))

  print(apply(ZZ,2,median))
  print(apply(ZZ,2,mean))
  print(apply(ZZ,2,sd))

  par(mfrow=c(3,1))
  truehist(Z[,1])
  truehist(Z[,2])
  truehist(Z[,3])
\end{verbatim}
\end{Solution}
\begin{Solution}{13.2}
\textbf{Linear combinations of instruments}

\end{Solution}
\begin{Solution}{13.3}
\textbf{Compulsory School Attendance}

The code might look like this:
\begin{verbatim}
######################################
#### Compulsory School Attendance ####
######################################
  library(foreign)
  library(AER)
  graphics.off()

  #1) Replicate Figure I and II
    x <- read.dta(file.choose())
    View(x)
    dob <- x$yob+(x$qob-1)*0.25

    # Figure I
    Z <- matrix(NA,40,2)
    Z[,1] <- seq(1930,1939.75,by=0.25)
    for(i in 1:dim(Z)[1]) {
      Z[i,2] <- mean(x$educ[dob==Z[i,1]])
    }
    plot(Z,t="o",main="Figure I",xlab="Year of Birth",
                                        ylab="Years of Completed Education",ylim=c(12.2,13.2))

    # Figure II
    Z <- matrix(NA,40,2)
    Z[,1] <- seq(1940,1949.75,by=0.25)
    for(i in 1:dim(Z)[1]) {
      Z[i,2] <- mean(x$educ[dob==Z[i,1]])
    }
    plot(Z,t="o",main="Figure II",xlab="Year of Birth"
                                        ,ylab="Years of Completed Education",ylim=c(13,13.9))

  #2) Replicate Figure V
    Z <- matrix(NA,80,2)
    Z[,1] <- seq(1930,1949.75,by=0.25)
    for(i in 1:dim(Z)[1]) {
      Z[i,2] <- mean(x$lwklywge[dob==Z[i,1]])
    }
    plot(Z,t="o",main="Figure V",xlab="Year of Birth",ylab="Log Weekly Earnings")

  #3) Replicate column 1 of table IV
    # Backup data
    backup <- x
    # Drop all persons born after 1929Q4
    x <- x[x$yob<1930,]
    dob <- x$yob+(x$qob-1)*0.25

    # Create yob dummies
    Dyear <- factor(x$yob)

    # Column (1)
    regr <- lm(x$lwklywge ~ x$educ + Dyear)
    summary(regr)

  #4) Replicate column 3 of table IV
    age <- 1970-dob
    # Column (3)
    regr <- lm(x$lwklywge~x$educ+Dyear+age+I(age^2))
    summary(regr)

  #5) Replicate column 2 of table IV
    Dq <- dob
    Dq[Dq-floor(Dq)==0.75] <- 0
    Dq <- factor(Dq)
    # Column (2)
    regr <- ivreg(x$lwklywge~x$educ+Dyear|Dq+Dyear)
    summary(regr)

  #6) Replicate column 4 of table IV
    # Column (4)
    regr <- ivreg(x$lwklywge~x$educ+age+I(age^2)+Dyear|Dq+Dyear)
    summary(regr)
\end{verbatim}
\end{Solution}
\begin{Solution}{13.4}
\textbf{A simple example}
\end{Solution}
\begin{Solution}{13.5}
\textbf{Money demand}
\begin{enumerate}
  \item Durbin-Wu-Hausman test: \\
  $H_0:$ $r_t$ can be treated as exogenous (OLS is better $E(X'u)=0$) vs.
      $H_1:$ $r_t$ cannot be treated as exogenous (OLS is not consistent,
      IV model is better $E(W'u)=0$), with $r_{t-1}$ and $r_{t-2}$ as
      additional instruments.\\
      Idea: compare $\widehat{\beta_{IV}} -
      \widehat{\beta_{OLS}}$. To test if this difference is significantly
      different from zero, perform a Wald test of $\delta=0$ in the
      Wu-regression: $y=X\beta + P_W \widetilde{X}\delta$ with
      $\widetilde{X}$ including all possible endogenous regressors (here
      $r_t$) and $P_W = W(W'W)^{-1}W'$
  \item $\widehat{\beta_{GIV}} = (X'P_WX)^{-1}X'P_Wy$
  \item Test of overidentifying restrictions: \\
  Idea: Test if IV residuals can be explained by the full set of
      instruments $W$.\\ $H_0:$ Instruments are valid and uncorrelated
      with the residuals. \\Testregression: $u_i=W_i'\gamma +
      \varepsilon_i$ with $i=1,\dots,n$. \\Teststatistics:$n~R^2 \sim
      \chi^2(m)$ with $m:$ degrees of overidentification. \\If
      instruments pass the test (that is $H_0$ is not rejected), they are
      valid by this criterion.
\end{enumerate}
The code might look like this:
\begin{verbatim}
######################
#### Money demand ####
######################
  library(AER)
  # read data, define vectors and matrices
  money <- read.csv(file.choose())
  View(money)
  m <- money$m
  r <- money$r
  y <- money$y
  TT <- length(m)
  # Matrix of regressors
  X <- cbind(1,r[5:TT],y[5:TT],m[4:(TT-1)],m[3:(TT-2)])
  # Matrix of Instruments
  W <- cbind(1,r[4:(TT-1)], r[3:(TT-2)],y[5:TT], m[4:(TT-1)], m[3:(TT-2)])
  # Projection-matrix
  Pw <- W %*% solve( t(W)%*%W ) %*% t(W)

  #1)
  OLS <- lm(m[5:TT] ~ r[5:TT] + y[5:TT] + m[4:(TT-1)] + m[3:(TT-2)])
  summary(OLS)

  # Durbin-Wu-Hausman test
  WuRegr <- lm(m[5:TT] ~ Pw%*%r[5:TT] + y[5:TT] + m[4:(TT-1)] + m[3:(TT-2)])
  summary(WuRegr)
  # H0 can be rejected.
  # If r_t-1 and r_t-2 are appropriate instruments for r_t,
  # then the OLS estimator is not consistent, but the IV-estimator is!
  # Hence, r_t should not be treated as exogenous!

  #2)
  IV <- ivreg(m[5:TT] ~ r[5:TT] + y[5:TT] + m[4:(TT-1)] + m[3:(TT-2)]
                        |r[4:(TT-1)]+ r[3:(TT-2)] + y[5:TT] + m[4:(TT-1)] + m[3:(TT-2)])
  summary(IV)
  #You can get the same coefficients, if you simply use the Formula
  betaGIV <- solve(t(X)%*%Pw%*%X)%*%t(X)%*%Pw%*%m[5:TT]
  betaGIV

  # compare to OLS coefficients:
  OLS$coefficients
  IV$coefficients

  # Test if the coefficient is the same
  covmat <- IV$sigma^2*IV$cov
  linearHypothesis(IV,paste("r[5:TT]=",OLS$coefficients[2]),V=covmat) # H0 can not be rejected

  #3)
  u <- residuals(IV)
  n <- length(u)
  #perform OLS and store the results
  obj <- summary(lm(u ~ r[4:(TT-1)] + r[3:(TT-2)] + y[5:TT] + m[4:(TT-1)] + m[3:(TT-2)]))
  teststat <- n*obj$r.squared
  1-pchisq(teststat,df=1) # p-value
  #Nullhypothesis can be rejected, the model is thus not overidentified.
\end{verbatim}
\end{Solution}
\begin{Solution}{13.6}
\textbf{Tests for the IV model}

The code might look like this:
\begin{verbatim}
################################
#### Tests for the IV model ####
################################
  library(AER)
  fertility <- read.csv2(file.choose())
  View(fertility)
  attach(fertility)
  #1)
    n <- dim(fertility)[1] #number of families
    x <- fertility[morekids==1,] # only families that had another child
    m <- dim(x)[1] #number of families that had another child
    dim(x[x$samesex==1,])[1]/m
    dim(x[x$samesex==0,])[1]/m

  #2)
    regr <- lm(weeksm1 ~ morekids + boy1st + boy2nd + agem1 + black + hispan + othrace)
    summary(regr)

  #3)
    #Relevance of instrument
    relevInst <- lm(morekids ~ samesex)
    summary(relevInst) # F-Statistic is high!

  #4)
    ivmod <- ivreg(weeksm1 ~ morekids+boy1st + boy2nd + agem1 + black
                            + hispan + othrace|samesex + boy1st + boy2nd
                                                + agem1 + black + hispan + othrace)
    summary(ivmod)

  #5) t-test that two coefficients are equal
    covmat <- ivmod$sigma^2*ivmod$cov
    betaHisp <- ivmod$coefficients[7]
    betaOthr <- ivmod$coefficients[6]
    varHisp <- covmat[7,7]
    varOthr <- covmat[6,6]
    ttest <- (betaHisp -betaOthr)/sqrt(varHisp + varOthr)
    abs(ttest) > 2.56 # reject H0 at least on a 1% level!
    1-pt(abs(ttest),n-7) #p-value is zero!

  #6)
  betahat <- ivmod$coefficients[c(3,4,7)]
  beta0 <- c(0,0,0)
  ivmod0 <- ivreg(weeksm1 ~ morekids + agem1 + black + hispan
                                + othrace|samesex + boy1st + boy2nd
                                            + agem1 + black + hispan + othrace)
  # Wald test formula and p-value
  Wald <- (betahat-beta0)%*%solve(covmat[c(3,4,7),c(3,4,7)])%*%(betahat - beta0)
  1-pchisq(Wald,df=3)
  #or simply use the linearHypothesis command
  linearHypothesis(ivmod,c("boy1st=0","boy2nd=0","hispan=0"), V=covmat)
\end{verbatim}
Additional remarks:
\begin{itemize}
  \item[2)] The coefficient is -6.23209. This indicates that women with
      more than 2 children work 6.23209 fewer weeks per year than women
      with 2 or fewer children. However, both fertility (morekids) and
      laborsupply (weeks worked) are choice variables. A woman who works
      more than average (positive regression error) may also be a woman
      who is less likely to have an additional child. This would imply
      that morekids is positively correlated with the regression error.
      OLS estimator is thus positively biased.
  \item[3)] Samesex is random and unrelated to any of the other variables
      including the error term in the labor supply equation. Thus, the
      instrument is exogenous. The instrument is also relevant (see part
      1). You can also test for relevance: compute the F-Statistic in the
      regression: $morekids_i = \beta_0+\beta_1 samesex_i +
      \varepsilon_i$.
  \item[5)] $H_0: \beta_4-\beta_5 = 0,$ $ t=\frac{\widehat{\beta_4} -
      \widehat{\beta_5}}{\sqrt{\widehat{var(\beta_4)}+\widehat{var(\beta_5)}}}$
      under the assumption that $\beta_4$ and $\beta_5$ are independent.
  \end{itemize}
\end{Solution}
\begin{Solution}{14.1}
\textbf{The R packagae gmm}

Basic scheme to use gmm:
\begin{enumerate}
  \item Set up matices with data $Y$ and/or instruments $W$.
  \item Specify the moment conditions
  \begin{verbatim} g <- function(param,dat) {moment conditions} \end{verbatim}.
  For the gmm package this means that you have to program the formula
      inbetween the expectation $E[g(\theta,Y)]=0$. Here: \begin{align*}
      X-\mu\\
      (X-\mu)^2 - \sigma^2\\
      X^3-\mu(\mu^2-3\sigma^2)
      \end{align*}
  \item Call
      $gmm(\underbrace{g=g}_\text{function},\underbrace{x=Y}_\text{data},\underbrace{t0=c(0,\dots,0)}_{\text{starting
      values}},\dots)$. Use appropriate arguments for numerical
      optimization, weights, gradient (to improve precision), etc. Most
      of the times just use the standard setting.
\end{enumerate}

The code might look like this:
\begin{verbatim}
############################
#### The R packagae gmm ####
############################
install.packages("gmm")
library(gmm)
##generate data
set.seed(123)
n <- 200
dat <- rnorm(n,mean=4,sd=2)

##moment conditions
g <- function(param,x) {
  m1 <- param[1]-x
  m2 <- param[2]^2-(x-param[1])^2
  m3 <- x^3 - param[1]*(param[1]^2+3*param[2]^2)
  f <- cbind(m1,m2,m3)
  return(f)
}
g(c(3,1),dat) #gives you a n-by-L matrix with n observations and L moments

##gmm estimation with standard settings
t0 <- c(mu=3, sigm=1)
res1 <- gmm(g,dat,t0)
print(res1)
summary(res1)
coef(res1)
vcov(res1)
confint(res1)
specTest(res1)

##gmm estimation with gradient
#gradient of gbar=1/n*sum(g)
Dgbar <- function(param,x){
  xbar <- mean(x)
  Dgbar <- matrix(c(1, 2*(xbar-param[1]), -3*(param[1]^2+param[2]^2), 0, 2*param[2], -6*param[1]*param[2]),nrow=3)
  return(Dgbar)
}
Dgbar(c(3,1),dat)
t0 <- c(mu=3, sigm=1)
res2 <- gmm(g,dat,t0,grad=Dgbar)
print(res2);summary(res2);coef(res2);vcov(res2);confint(res2);specTest(res2)

##gmm estimation with restrictions on parameter space (with nlminb)
t0 <- c(mu=3, sigm=1)
res3 <- gmm(g,dat,t0,optfct="nlminb",lower=c(-5,0),upper=c(5,5))
print(res3);summary(res3);coef(res3);vcov(res3);confint(res3);specTest(res3)

##gmm estimation: Iterative GMM (ITGMM) and continuous updated GMM (CUE)
t0 <- c(mu=3, sigm=1)
res4 <- gmm(g,dat,t0,type="iterative",crit=1e-5,itermax=200)
print(res4);summary(res4);coef(res4);vcov(res4);confint(res4);specTest(res4)
res5 <- gmm(g,dat,res4$coef,type="cue")#use ITGMM as starting values
print(res5);summary(res5);coef(res5);vcov(res5);confint(res5);specTest(res5)
\end{verbatim}
\end{Solution}
\begin{Solution}{14.2}
\textbf{Nonlinear least squares estimation and GMM}

\begin{enumerate}
  \item As elementary zero functions we can use $E(u_t)=0$ and $E(u_t^2-\sigma^2)=0$. As Instruments we can use the elementary zero function itself and $x_t(\beta)$, since $E(x_t(\beta)\cdot u_t)=E(x_t(\beta)\cdot (y_t-x_t(\beta)))=0$. That is we have three moment conditions:
      \begin{align*}
      E(u_t) = E(y_t-x_t(\beta)) &= 0\\
      E(x_t(\beta) u_t) = E(x_t(\beta)\cdot (y_t-x_t(\beta)))&= 0\\
      E(u_t^2-\sigma^2) = E((y_t-x_t(\beta)^2-\sigma^2) &=0
      \end{align*}
  \item Here $x_t(\beta)=exp(\alpha+\beta x_t)$. This yields the same results as in exercise \ref{nls1}.
\end{enumerate}
The code might look like this:
\begin{verbatim}
####################################
#### Nonlinear regression model ####
####################################
expgrowth <- read.csv(file.choose())
View(expgrowth)
x <- expgrowth$x
y <- expgrowth$y

g <- function(param,dat){
  y <- dat[,1]
  x <- dat[,2]
  u <- y - exp(param[1]+param[2]*x)
  f1 <- u
  f2 <- u^2
  m1 <- 1*f1
  m2 <- x*f1
  m3 <- f2-param[3]^2
  return(cbind(m1,m2,m3))
}
g(c(1,0.1,1),cbind(y,x)) # g gives you a nxq matrix with n observations and q moments

gmmmod <- gmm(g,x=cbind(y,x),c(alpha=0,beta=0.01,sigma=1))
print(gmmmod)
gmmmod <- gmm(g,x=cbind(y,x),c(alpha=2,beta=0.01,sigma=2))
print(gmmmod$coefficients)
#depends on the start-values, in order to be more precise one could specify the gradient
summary(gmmmod)
\end{verbatim}
\end{Solution}
\begin{Solution}{14.3}
\textbf{Ordinary least squares estimation and GMM}

The code might look like this
\begin{verbatim}
###################################################
#### Ordinary least squares estimation and GMM ####
###################################################
#Get data
olsgmm <- read.csv(file.choose())
View(olsgmm)
y <- olsgmm$y
x1 <- olsgmm$x1
x2 <- olsgmm$x2

#OLS
ols <- lm(y~x1+x2)
summary(ols)

#GMM estimation explicitly taking into account linearity
gmm_lin <- gmm(y~x1+x2,cbind(x1,x2))
gmm_lin$coef - ols$coef #numerically the same

#GMM estimation without linearity
g <- function(param,dat){
  y <- dat$y
  x1 <- dat$x1
  x2 <- dat$x2
  u <- y -param[1] -param[2]*x1 -param[3]*x2
  f1 <- u
  f2 <- u*x1
  f3 <- u*x2
  return(cbind(f1,f2,f3))
}
g(c(1,0.5,1.2),olsgmm) # g gives you a nxq matrix with n observations and q moments

t0<-c(alpha=1,beta1=0.5,beta2=1.2)
gmm_nonlin <- gmm(g,olsgmm,t0)
print(gmm_nonlin$coefficients)
print(gmm_lin$coefficients)
\end{verbatim}
\end{Solution}
\begin{Solution}{14.4}
Maximum likelihood estimation and GMM

The code might look like this
\begin{verbatim}
## Maximum Likelihood of Normal Distribution
mue = 1; sigm =1;
X <- rnorm(1000,mean=mue,sd=sigm)

g <- function(param,x){
  mue <- param[1]
  sigm <- param[2]
  dmue <- 1/(sigm^2)*(x-mue)
  dsig <- -1/sigm +1/(sigm^3)*(x-mue)^2
  cbind(dmue,dsig)
}

Dg <- function(param,x){
  mue <- param[1]
  sigm <- param[2]
  z <- matrix(c( -1/sigm^2, (2*mue - 2*mean(x))/sigm^3, (2*(mue - mean(x))/sigm^3), 1/sigm^2 - (3*(mue - mean(x))^2)/sigm^4),nrow=2)
  return(z)
}

Dg(c(1,0.1),X)
gmm(g,X,c(2,.01))
\end{verbatim}
\end{Solution}
\begin{Solution}{14.5}
\textbf{Instrumental variables estimation and GMM }

The gmm framework is given by
\begin{align*}
  E[m_t -X_t\beta] = 0\\
  E[y_t(m_t -X_t\beta)] = 0\\
  E[m_{t-1}(m_t -X_t\beta)] = 0\\
  E[m_{t-2}(m_t -X_t\beta)] = 0\\
  E[r_{t-1}(m_t -X_t\beta)] = 0\\
  E[r_{t-2}(m_t -X_t\beta)] = 0
\end{align*}

 The code might look like this:
\begin{verbatim}
###################################################
#### Instrumental variables estimation and GMM ####
###################################################
library(gmm)
library(AER)
# read data, define vectors and matrices
money <- read.csv(file.choose())
View(money)
m <- money$m
r <- money$r
y <- money$y
TT <- length(m)

yt <- y[5:TT]
mt <- m[5:TT]; mt1 <- m[4:(TT-1)]; mt2 <- m[3:(TT-2)]
rt <- r[5:TT]; rt1 <- r[4:(TT-1)]; rt2 <- r[3:(TT-2)]

# Gmm estimation in linear model notation
obj <- gmm(mt~rt+yt+mt1+mt2, ~rt1+rt2+yt+mt1+mt2)
obj$coefficients

# Generalized IV estimation
IV <- ivreg(mt~rt+yt+mt1+mt2|rt1+rt2+yt+mt1+mt2)
IV$coefficients
\end{verbatim}
\end{Solution}
\begin{Solution}{14.6}
\textbf{Moment conditions and moment existence }

The code might look like this:
\begin{verbatim}
################################################
#### Moment conditions and moment existence ####
################################################
library(gmm)
library(MASS)

n <- 100
x <- rt(n,df=3)/sqrt(3)
u <- rt(n,df=3)/sqrt(3)
y <- 0.9*x+u

g <- function(theta,dat) {
  y <- dat[,1]
  x <- dat[,2]
  bet <- theta[1]
  sigm <- theta[2]
  u <- y-bet*x
  m1 <- u
  m2 <- u*x
  m3 <- u^2-sigm^2
  return(cbind(m1,m2,m3))
}
# GMM estimates for beta and sigma
obj <- gmm(g,cbind(y,x),t0=c(0.9,1),wmatrix="optimal")
obj1 <- gmm(g,cbind(y,x),t0=c(0.9,1),wmatrix="ident")
obj
obj1

# Whole thing within a loop
R <- 1000
n <- 100 # or try n=1000
nu <- 3
Z1 <- matrix(NA,R,2)
Z2 <- matrix(NA,R,2)
for(r in 1:R) {
  x <- rt(n,df=nu)/sqrt(nu/(nu-2))
  u <- rt(n,df=nu)/sqrt(nu/(nu-2))
  y <- 0.9*x+u
  obj1 <- gmm(g,x=cbind(y,x),t0=c(0.9,1),wmatrix="optimal")
  Z1[r,] <- (coefficients(obj1)-c(0.9,1))/sqrt(diag(vcov(obj1)))
  obj2 <- gmm(g,x=cbind(y,x),t0=c(0.9,1),wmatrix="ident")
  Z2[r,] <- (coefficients(obj2)-c(0.9,1))/sqrt(diag(vcov(obj2)))
}
# mean of the estimates for beta and sigma
apply(Z1,2,mean)
apply(Z2,2,mean)
#histogram of sigma compared to a normal distribution
truehist(Z1[,2])
xx <- seq(min(Z1[,2]),max(Z1[,2]),length=500)
lines(xx,dnorm(xx,mean(Z1[,2]),sd(Z1[,2])))

truehist(Z2[,2])
xx <- seq(min(Z2[,2]),max(Z2[,2]),length=500)
lines(xx,dnorm(xx,mean(Z2[,2]),sd(Z2[,2])))

###Now with the normal distribution
R <- 1000 # or try R=1000
n <- 100 # or try n=1000
Z1 <- matrix(NA,R,2)
Z2 <- matrix(NA,R,2)
for(r in 1:R) {
  x <- rnorm(n)
  u <- rnorm(n)
  y <- 0.9*x+u
  obj1 <- gmm(g,cbind(y,x),t0=c(0.9,1),wmatrix="optimal")
  obj2 <- gmm(g,cbind(y,x),t0=c(0.9,1),wmatrix="ident")
  Z1[r,] <- (coefficients(obj1)-c(0.9,1))/sqrt(diag(vcov(obj1)))
  Z2[r,] <- (coefficients(obj2)-c(0.9,1))/sqrt(diag(vcov(obj2)))
}
# mean of the estimates for beta and sigma
apply(Z1,2,mean)
apply(Z2,2,mean)

#histogram of sigma compared to a normal distribution
truehist(Z1[,2])
xx <- seq(min(Z1[,2]),max(Z1[,2]),length=500)
lines(xx,dnorm(xx,mean(Z1[,2]),sd(Z1[,2])))

truehist(Z2[,2])
xx <- seq(min(Z2[,2]),max(Z2[,2]),length=500)
lines(xx,dnorm(xx,mean(Z2[,2]),sd(Z2[,2])))
\end{verbatim}
\begin{itemize}
  \item The GMM estimator is asymptotically normally distributed.
      However, the estimator for the t-distribution with 3 degrees of
      freedom is not asymptotically normally distributed. This is because
      for convergence higher moments have to exist, which they don't for
      the t-distribution. If you use the normal distribution or another
      distribution, then the distribution is well approximated by a
      normal distribution.
  \item The weighting scheme does not influence the approximate
      distribution.
\end{itemize}
\end{Solution}
\begin{Solution}{14.7}
\textbf{Standard CAPM }

The moment conditions are given by
\begin{equation*}
  E\left[(\theta_1+\theta_2 R_{mt})R_{it}-1\right]=0
\end{equation*}
The code might look like this:
\begin{verbatim}
#######################
#### Standard CAPM ####
#######################
library(gmm)
#Get data
data(Finance)
?Finance
View(Finance)
#let's only take the first 500 observations
r <- Finance[,c("WMK","UIS","ORB","MAT","ABAX")]
rm <- Finance[,"rm"]

#Define moment conditions
g <- function(param,dat){
  R <- 1+dat[,1:5]
  Rm <- 1+ dat[,6]
  m <- (param[1]+param[2]*Rm)*R-1
  return(m)
}

#estimate GMM
obj <- gmm(g,x=cbind(r,rm),c(0,0)) #this gives you an error
mode(r) # problem is in the data: r is a list and not numeric. gmm needs numeric data
mode(rm) # rm is numeric
mode(cbind(r,rm)) #if we combine the data, we still get a list
X <- as.matrix(cbind(r,rm)) #so let's convert it into a numeric structure
mode(X)
obj <- gmm(g,x=X,c(0,0))
obj
summary(obj)

# Test of overidentifying restrictions
specTest(obj) #confirms the non-rejection of the theory
\end{verbatim}
We can think of a return as a payoff with price one. If you pay one dollar
today, the return is how many dollars or units of consumption you get
tomorrow.

The asset pricing model says that, although expected returns can vary
across time and assets, expected discounted returns should always be the
same, 1.
\end{Solution}
\begin{Solution}{14.8}
\textbf{Consumption-based CAPM}

The code might look like this:
\begin{verbatim}
####################################
###### Consumption-based CAPM ######
####################################
#Load Data
consumptiondata <- read.csv(file.choose(), sep=";", dec=",")
dax30ann <- read.csv(file.choose())
View(consumptiondata)
View(dax30ann)
V7 <- consumptiondata$V7
dax30 <- dax30ann$dax30

#growth rates for consumption
gr1 <- V7[2:22]/V7[1:21]
gr2 <- V7[24:42]/V7[23:41]
gr <- c(gr1,gr2)
#gross return of DAX
R <- dax30[3:42]/dax30[2:41]
#Matrix of data
X <- cbind(gr,R)

g <- function(param,dat){
  TT <- dim(dat)[1]
  gr <- dat[,1]
  R <- dat[,2]
  f <- param[1]*(gr[-1]^(-param[2]))*R[-1]-1
  z <- cbind(1,gr[-TT],R[-TT])
  m <- f*z
  return(m)
}
g(c(1,1),X)
obj <- gmm(g,x=X,c(1,1))
obj
\end{verbatim}
\end{Solution}
\begin{Solution}{14.9}
\textbf{Minimum distance estimation}
\end{Solution}
\begin{Solution}{15.1}
\textbf{AR(1) processes}

The code might look like this:
\begin{verbatim}
############################################
#### Indirect Inference - AR(1) process ####
############################################
library(MASS)
#Define function that estimates the ar process for different values of rho,n and R
simestim <- function(rho,n,R) {
  Z <- matrix(NA,R,3)
  for (r in 1:R) {
    x <- filter(rnorm(n),rho,method="r",init=rnorm(1))
    Z[r,1]<- ar(x,aic=F,order = 1,method = "yw")$ar # method: yule-walker
    Z[r,2]<- ar(x,aic=F,order = 1, method = "ols")$ar #method: ols
    Z[r,3]<- ar(x,aic=F,order = 1, method = "mle")$ar #method: mle
  }
  return(Z)
}

#1) Stationary AR(1)
stationary <- simestim(rho=0.9,n=100,R=1000)
truehist(stationary[,1],main="Yule-Walker")
truehist(stationary[,2],main="OLS")
truehist(stationary[,3],main="ML")

#2) Random Walk, methods: yule-walker, ols, mle
randomwalk <- simestim(rho=1,n=100,R=1000)
truehist(randomwalk[,1],main="Yule-Walker")
truehist(randomwalk[,2],main="OLS")
truehist(randomwalk[,3],main="ML")

#3) Explosive AR(1)
explosive <- simestim(rho=1.01,n=100,R=1000)
truehist(explosive[,1],main="Yule-Walker")
truehist(explosive[,2],main="OLS")
truehist(explosive[,3],main="ML")


#4) Estimation by indirect inference, auxiliary model is an AR(1)-process
H <- 100; n <- 100; W <- diag(1)

rhohat <- function(truedata,H,n,W){
  thetahat <- ar(truedata,aic=F,order = 1)$ar #auxiliary model with true data
  thetahat <- as.matrix(thetahat)             #as matrix so you can compute Q

  f <- function(rho){
    thetahatsim <- rep(NA,H)
    set.seed(123)
    for (h in 1:H){
      simdata <- filter(rnorm(n),rho,method="r",init=rnorm(1)) #simulated data depending on beta
      thetahatsim[h] <- ar(simdata,aic=F,order = 1)$ar          #store estimator
    }
    thetatilde <- mean(thetahatsim)
    Q <- t(thetahat - thetatilde) %*% W %*% (thetahat - thetatilde)
    return(Q)
  }
  # indirect inference estimator for rho
  rhohat <- optimize(f,lower=0.7,upper=1.5)
  return(rhohat$minimum)
}

truedata <- filter(rnorm(n),0.9,method="r",init=rnorm(1)) #true model with rho=0.8

rhohat(truedata,H=10,n=100,W=diag(1))
ar(truedata,aic=F,order = 1,method="ols")$ar #method: ols
ar(truedata,aic=F,order = 1,method="yw")$ar #method: yw
ar(truedata,aic=F,order = 1,method="mle")$ar #method: mle
\end{verbatim}
\end{Solution}
\begin{Solution}{15.2}
\textbf{Filter models }
\end{Solution}
\begin{Solution}{15.4}
\textbf{Ornstein-Uhlenbeck process}

The code might look like this
\begin{verbatim}
  #########################################################
  #### Indirect Inference - Ornstein-Uhlenbeck process ####
  #########################################################

  #1) Single path of an Ornstein-Uhlenbeck process
  install.packages("sde")
  library(sde)
  ?sde.sim #look at the example
  # Ornstein-Uhlenbeck
  drift <- expression(0.9*0 - 0.9 * x)
  sigma <- expression(1)
  X <- sde.sim(X0=2,drift=drift, sigma=sigma,N=100,T=100)
  plot(X,main="Ornstein-Uhlenbeck")
  # or specify model to generate data
  mu <- 0;  lambda <- 0.9; sigma <- 1
  X <- sde.sim(t0=0,T=100,X0=2, N=100,theta=c(lambda*mu,lambda,sigma),model="OU")
  plot(X,main="Ornstein-Uhlenbeck")

  #2) Estimation of discrete model
  library(AER)
  R <- 1000
  Z <- rep(NA,R)
  mu <- 0;  lambda <- 0.9; sigma <- 1

  for (r in 1:R){
    X <- sde.sim(t0=0,T=100,X0=2, N=100,theta=c(lambda*mu,lambda,sigma),model="OU")
    estim <- arima(X,order=c(1,0,0))
    Z[r] <- estim$coef[1]
  }

  truehist(Z) #histogram of 1-lambda, totally wrong
  truehist(1-Z) #histogram of lambda, totally wrong

  #3) Indirect inference estimation
  oupath <- read.table(file.choose(), header=T)
  truedata <- oupath$x
  plot(truedata,type="l")
  #Hint: The data generating process uses lambda=1.3; mu=9; sigma=3; T=N<-100

  OUIndirect <- function(truedata,H,W,startval){
    estim <- arima(truedata,order=c(1,0,0)) #auxiliary model with true data
    lambdahat <- 1-estim$coef[1]
    muhat <- estim$coef[2]/lambdahat
    sigmahat <- sqrt(estim$sigma2)
    thetahat <- rbind(lambdahat,muhat,sigmahat)

    f <- function(theta){
      thetahatsim <- matrix(NA,H,3)
      set.seed(123)
      for (h in 1:H){
        lambda <- theta[1]; mu <- theta[2]; sigma <- theta[3]
        #simulated data depending on theta
        simdata <- sde.sim(t0=0,T=100,X0=mu, N=100,theta=c(lambda*mu,lambda,sigma),model="OU")
        estimsim <- arima(simdata,order=c(1,0,0))         #store estimator
        lambdahatsim <- 1-estimsim$coef[1]
        muhatsim <- estimsim$coef[2]/lambdahatsim
        sigmahatsim <- estimsim$sigma2
        thetahatsim[h,1] <- lambdahatsim
        thetahatsim[h,2] <- muhatsim
        thetahatsim[h,3] <- sigmahatsim
      }
      thetatilde <- colMeans(thetahatsim)
      Q <- t(thetahat - thetatilde) %*% W %*% (thetahat - thetatilde)
      return(Q)
    }
    # optimize ove all thetas
    res <- optim(startval,f)
    return(res$par)
  }

  OUIndirect(truedata=truedata,H=10,W=diag(3),startval=c(1.3,9,3))
\end{verbatim}
\end{Solution}
\begin{Solution}{15.5}
\textbf{Time-aggregated observations}
\end{Solution}
\begin{Solution}{16.1}
\textbf{Omitted variables bias does not go away }

The code might look like this:
\begin{verbatim}
  #################################################
  #### Omitted variables bias does not go away ####
  #################################################
  #1)
  #Input data and make it accessible
  omitted <- read.csv(file.choose(), sep=";", dec=",")
  x1 <- omitted$x1
  x2 <- omitted$x2
  x3 <- omitted$x3
  x4 <- omitted$x4
  y <- omitted$y
  n <- length(y)
  #Estimate by OLS without x4
  #Reminder: the true model is Y = 1 + 2X1 + 3X2 + 4X3 + 5X4
  # X1 is uncorrelated, X2&X3 are correlated and X3&X4, X2&X4 are uncorrelated
  mod <- lm(y~x1+x2+x3)
  summary(mod) #only the uncorrelated variable x1 is estimated well

  #2)
  ### 1.Alternative: Bootstrap the residuals
  uhat <- mod$residuals
  ahat <- coefficients(mod)[1]
  beta1hat <- coefficients(mod)[2]
  beta2hat <- coefficients(mod)[3]
  beta3hat <- coefficients(mod)[4]

  B <- 1000
  betastar <- matrix(NA,B,2)
  SEbetastar <- matrix(NA,B,2)
  for(b in 1:B) {
    ustar <- sample(uhat,n,replace=TRUE)
    ystar <- ahat+beta1hat*x1+beta2hat*x2+beta3hat*x3+ustar
    bootmod <- lm(ystar~x1+x2+x3)
    betastar[b,1] <- coefficients(bootmod)[3] #coef for x2
    betastar[b,2] <- coefficients(bootmod)[4] #coef for x3
    SEbetastar[b,1] <- sqrt(vcov(bootmod)[3,3]) #SE for x2
    SEbetastar[b,2] <- sqrt(vcov(bootmod)[4,4]) #SE for x3
  }

  # Calculate the bias
  print(colMeans(betastar)-c(beta2hat,beta3hat)) #omitted variable bias does not go away
  colMeans(SEbetastar)

  ### 2. Alternative: Bootstrap of the observations
  B <- 1000
  betastar <- matrix(NA,B,2)
  SEbetastar <- matrix(NA,B,2)
  for(b in 1:B) {
    indices <- sample(1:n,n,replace=TRUE)
    x1star <- x1[indices]
    x2star <- x2[indices]
    x3star <- x3[indices]
    ystar <- y[indices]
    bootmod <- lm(ystar~x1star+x2star+x3star)
    betastar[b,1] <- coefficients(bootmod)[3] #coef for x2
    betastar[b,2] <- coefficients(bootmod)[4] #coef for x3
    SEbetastar[b,1] <- sqrt(vcov(bootmod)[3,3]) #SE for x2
    SEbetastar[b,2] <- sqrt(vcov(bootmod)[4,4]) #SE for x3
  }

  # Calculate the bias
  print(colMeans(betastar)-c(beta2hat,beta3hat)) #omitted variable bias does not go away
  colMeans(SEbetastar)
\end{verbatim}
\end{Solution}
\begin{Solution}{16.2}
\textbf{Confidence intervals for the Gini index}

The program could look like this:
\begin{verbatim}
  #################################################
  #### Confidence intervals for the Gini index ####
  #################################################

  #1)
  install.packages("ineq")
  library(ineq)
  earnings <- read.csv(file.choose(), header=T)
  x <- earnings$x
  ?ineq
  ineq(x,type="Gini")
  Ginihat <- Gini(x)
  print(Ginihat)
  #2)
  n <- length(x)
  B <- 1000
  Ginistar <- rep(NA,B)

  for(b in 1:B) {
    # Draw a resample
    xx <- sample(x,n,replace=TRUE)
    # Compute and save the Gini-coefficient for the resample
    Ginistar[b] <- Gini(xx)
    }

  # Compute the standard error
  print(sd(Ginistar))

  #3)
  # Sort Ginistar
  Ginistar <- sort(Ginistar)

  print(paste("Lower limit = ",2*Ginihat-Ginistar[0.975*B]))
  print(paste("Upper limit = ",2*Ginihat-Ginistar[0.025*B]))
\end{verbatim}
\end{Solution}
\begin{Solution}{16.3}
\textbf{Confidence intervals for correlation coefficients }

The code might look like this:
\begin{verbatim}
  ###########################################################
  #### Confidence intervals for correlation coefficients ####
  ###########################################################
  #1)
  install.packages("copula")
  library(copula)
  x <- qexp(rcopula(gumbelCopula(1.3),1000))
  plot(x)
  cor(x)

  #2)
  library(AER)
  n <- 50
  x <- qexp(rcopula(gumbelCopula(1.3),n))
  corhat <- cor(x)[1,2]
  B <- 5000
  corstar <- rep(NA,B)

  for(b in 1:B) {
    #Draw a resample
    indices <- sample(1:n,n,replace=T)
    xx <- x[indices,]
    corstar[b] <- cor(xx)[1,2]
  }
  truehist(corstar)
  curve(dnorm(x,mean=mean(corstar),sd=sd(corstar)),add=T)
  # the normal distribution does noch fit

  #3)
  corstar <- sort(corstar)

  print(paste("Lower limit = ",2*corhat-corstar[0.975*B]))
  print(paste("Upper limit = ",2*corhat-corstar[0.025*B]))
  #poor confidence intervals!
\end{verbatim}
\end{Solution}
\begin{Solution}{16.4}
\textbf{The t-test}

\begin{verbatim}
  #####################
  #### The t-test ####
  ####################
  ttestboot <- read.csv(file.choose())
  x <- ttestboot$x
  y <- ttestboot$y
  n <- length(y)
  #Reminder: the data is generated by a=1, b=0, sigma=2

  #1)
  ols <- lm(y~x)
  obj <- summary(ols)
  tols <- obj$coefficients[2,3]
  pols <- obj$coefficients[2,4]

  #2)
  alphahat <- coefficients(ols)[1]
  betahat <- coefficients(ols)[2]
  uhat <- residuals(ols)
  sigma2hat <- 1/(n-2)*sum(uhat^2)

  #3)
  R <- 5000
  Z <- rep(NA,R)

  for(r in 1:R){
    ustar <- rnorm(n,sd=sqrt(sigma2hat))
    ystar <- alphahat + betahat*x + ustar
    olsstar <- lm(ystar~x)
    betahatstar <- coefficients(olsstar)[2]
    SEbetahatstar <- summary(olsstar)$coefficients[2,2]
    Z[r] <- (betahatstar - betahat)/(SEbetahatstar)
  }

  #4)
  library(AER)
  truehist(Z)
  curve(dt(x,df=7),add=T) #the approximation fits perfectly

  #5)
  length(Z[abs(Z)>abs(tols)])/length(abs(Z))
  pols
  #p-value is almost the same!
\end{verbatim}
\end{Solution}
\begin{Solution}{16.5}
\textbf{The percentile-t-method }

The code might look like this:
\begin{verbatim}
  #################################
  #### The percentile-t-method ####
  #################################
  library(AER)
  # read data
  money <- read.csv(file.choose())
  m <- money$m
  r <- money$r
  y <- money$y
  TT <- length(m)

  yt <- y[5:TT]
  mt <- m[5:TT]; mt1 <- m[4:(TT-1)]; mt2 <- m[3:(TT-2)]
  rt <- r[5:TT]; rt1 <- r[4:(TT-1)]; rt2 <- r[3:(TT-2)]

  # Generalized IV estimation with confidence interval
  IV <- ivreg(mt~rt+yt+mt1+mt2|rt1+rt2+yt+mt1+mt2)
  summary(IV)
  confint(IV,parm="rt",level=.95)

  # Bootstrap 0.95-percentile-t condfidence intervals
  # estimate for original data
  rhat <- coefficients(IV)[2]
  SErhat <- sqrt(vcov(IV)[2,2])

  # Draw resamples
  B <- 1000
  Taustar <- rep(NA,B)
  for(b in 1:B) {
    indices <- sample(1:TT,TT,replace=TRUE)
    mtstar <- mt[indices]
    rtstar <- rt[indices]
    ytstar <- yt[indices]
    mt1star <- mt1[indices]
    mt2star <- mt2[indices]
    rt1star <- rt1[indices]
    rt2star <- rt2[indices]
    bootmod <- ivreg(mtstar~rtstar+ytstar+mt1star+mt2star|rt1star+rt2star+ytstar+mt1star+mt2star)
    SErstar <- sqrt(vcov(bootmod)[2,2])
    rstar <- bootmod$coefficients[2]
    Taustar[b] <- (rstar-rhat)/SErstar
  }

  # Sort
  Taustar <- sort(Taustar)

  #Compute the interval
  low <- rhat-Taustar[0.975*B]*SErhat
  high <- rhat-Taustar[0.025*B]*SErhat
  names(low)="2.5%"; names(high)="97.5%"
  CI <- c(low,high)

  print(CI) #CI using bootstrap
  confint(IV,parm="rt",level=.95) #CI using GIV estimation
\end{verbatim}
\end{Solution}
\begin{Solution}{16.8}
\textbf{Bootstrap test for the Zipf index of city size distributions}

The code might look like this:
\begin{verbatim}
  ######################################################################
  #### Bootstrap test for the Zipf index of city size distributions ####
  ######################################################################
  #1)
  library(AER)
  n <- 20
  R <- 1000
  y <- 1:n
  Z <- rep(NA,R)

  for (r in 1:R) {
    x <- sort(exp(rexp(n)),decreasing=TRUE)
    obj <- lm(log(y) ~ log(x))
    Z[r] <- coefficients(obj)[2]
  }
  truehist(Z)

  #2)
  n <- 20
  R <- 1000
  y <- 1:n
  TT <- rep(NA,R)
  for (r in 1:R) {
    x <- sort(exp(rexp(n)),decreasing=TRUE)
    obj <- lm(log(y) ~ log(x))
   TT[r] <- (coefficients(obj)[2]-1)/ sqrt(vcov(obj)[2,2])
  }
  truehist(TT)
  curve(dt(x,df=n-2),add=T)

  #3)
  # True data
  n <- 20
  alpha <- 1 #true value
  x <- sort(exp(rexp(n,rate=alpha)),decreasing=TRUE)
  y <- 1:n

  # Hypothetical value (here it is also the true value), Nullhypothesis
  alpha0 <- 1

  # Compute the test statistics for the true data
  obj <- lm(log(y)~log(x))
  alphahat <- coefficients(obj)[2]
  SEalphahat <- sqrt(vcov(obj)[2,2])
  Tstat <- (alphahat-alpha0)/SEalphahat #test statistic for true data

  # Approximate distribution through bootstrap
  B <- 1000
  Tsharp <- rep(NA,B)

  for(b in 1:B) {
    #draw a resample taking into account the nullhypothesis, alpha=alpha0=1
    #here it is also the true value
    xx <- sort(exp(rexp(n,rate=alpha0)),decreasing=TRUE)
    obj <- lm(log(y)~log(xx))
    alphasharp <- coefficients(obj)[2]
    SEalphasharp <- sqrt(vcov(obj)[2,2])
    Tsharp[b] <- (alphasharp-alpha0)/SEalphasharp
    }

  # Sort the Tsharp values
  Tsharp <- sort(Tsharp)
  truehist(Tsharp)
  critlow <- Tsharp[0.025*B]
  crithigh <- Tsharp[0.975*B]

  if(Tstat<critlow | Tstat>crithigh)
   print("Reject H0") else
   print("Don't reject H0")
\end{verbatim}

Please note that there are better ways to remedy the failure of standard OLS than using the bootstrap. In particular, extreme value theory provides many useful tools for estimation and testing.
\end{Solution}
