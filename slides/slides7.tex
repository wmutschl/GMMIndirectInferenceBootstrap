\documentclass[notes=show]{beamer}
\usepackage{amsmath,amsfonts}
\usetheme{Madrid}
\usecolortheme{seagull}
\setbeamertemplate{navigation symbols}{}

\begin{document}

\title{GMM, Indirect Inference and Bootstrap}
\subtitle{The three classical tests}
\author[Willi Mutschler]{Willi Mutschler}
\date{Winter 2015/2016}
\institute{TU Dortmund}
\maketitle

\begin{frame}\frametitle{The three classical tests}
\begin{itemize}
    \item Wald test, Lagrange multiplier test and likelihood ratio test \newline
    (W, LM, LR)
    \item Hypotheses
    \begin{equation*}
    H_{0}:r(\theta )=0\mathbf{\quad }\text{vs\quad }H_{1}:r(\theta )\neq 0
    \end{equation*}
    \item Often, $r$ is a scalar-valued function and $\theta $ is a scalar
    \item The function $r$ may be non-linear!
\end{itemize}
\end{frame}


\begin{frame}\frametitle{The three classical tests}
Basic test ideas:
\begin{itemize}
    \item Wald test:
    If $r(\theta )=0$ is true, then $r(\hat{\theta}_{ML})$ will be close to 0
    \item Likelihood ratio test:
    If $r(\theta )=0$ is true, then $\ln L(\hat{\theta}_{R})$ will not be far below $\ln L(\hat{\theta}_{ML})$

    \item Lagrange multiplier test:
    If $r(\theta )=0$ is true, the score function $g(\hat{\theta}_{R})=\partial \ln L(\hat{\theta}_{R})/\partial \theta $ will be close to 0
\end{itemize}
\end{frame}


\begin{frame}\frametitle{The three classical tests}
Example:
\begin{itemize}
    \item Let $X_{1},\ldots ,X_{n}$ be a random sample from $X\sim Exp(\lambda )$
    \item Test $H_{0}:\lambda =4$ against $H_{1}:\lambda \neq 4$
    \item Different notation:
    \begin{equation*}
        H_{0}:r(\lambda )=0
    \end{equation*}%
    where $r(\lambda )=\lambda -4$
    \item See \texttt{threetests.R}
\end{itemize}
\end{frame}


\begin{frame}\frametitle{The three classical tests}\framesubtitle{Wald test}
Wald test
\begin{itemize}
    \item Hypotheses%
    \begin{eqnarray*}
        H_{0} &:&r(\theta )=0 \\
        H_{1} &:&r(\theta )\neq 0
    \end{eqnarray*}%
    with functions $r\mathbf{=(}r_{1},\ldots ,r_{m})$
    \item $m$ is the number of restrictions
    \item Wald test: If $r(\theta )=0$ is true, then $r(\hat{\theta}_{ML})$ will be close to 0
\end{itemize}
\end{frame}


\begin{frame}\frametitle{The three classical tests}\framesubtitle{Wald test}
\begin{itemize}
    \item Asymptotically, under $H_{0}$ (by delta method!)
    \begin{equation*}
        r(\hat{\theta}_{ML})\sim N\left( 0,Cov(r(\hat{\theta}_{ML}))\right)
    \end{equation*}
    with
    \begin{equation*}
        Cov(r(\hat{\theta}_{ML}))=\frac{\partial r(\hat{\theta}_{ML})}{\partial\theta ^{\prime }}\cdot Cov(\hat{\theta}_{ML})\cdot \frac{\partial r(\hat{\theta}_{ML})}{\partial \theta }
    \end{equation*}
    \item Remember: If $X\sim N(\mu ,\Sigma )$, then $\left( X-\mu \right)^{\prime }\Sigma ^{-1}\left( X-\mu \right) \sim \chi _{m}^{2}$
    \item Wald test statistic
    \begin{equation*}
    W=r(\hat{\theta}_{ML})^{\prime }\left[ Cov(r(\hat{\theta}_{ML}))\right]^{-1}r(\hat{\theta}_{ML})\overset{asy}{\sim }\chi _{m}^{2}
    \end{equation*}
\end{itemize}
\end{frame}


\begin{frame}\frametitle{The three classical tests}\framesubtitle{Wald test}
Remarks:
\begin{itemize}
    \item Reject $H_{0}$ if $W$ is larger than the $\left( 1-\alpha \right)$-quantile of the $\chi _{m}^{2}$-distribution
    \item Usually, $Cov(r(\hat{\theta}_{ML}))$ must be replaced by $\widehat{Cov}(r(\hat{\theta}_{ML}))$
    \item The Wald test is not invariant with respect to re-parametrizations
    \item The Wald test only requires the unrestricted ML estimator
    \item Ideal, if $\hat{\theta}_{ML}$ is much easier to calculate than $\hat{\theta}_{R}$
\end{itemize}
\end{frame}


\begin{frame}\frametitle{The three classical tests}\framesubtitle{Likelihood ratio test}
Likelihood ratio test
\begin{itemize}
    \item Is $\ln L(\hat{\theta}_{ML})$ significantly larger than $\ln L(\hat{\theta}_{R})$ ?
    \item LR test statistic
    \begin{eqnarray*}
        LR &=&-2\ln \left( \frac{L(\hat{\theta}_{R})}{L(\hat{\theta}_{ML})}\right) \\
        &=&-2\left( \ln L(\hat{\theta}_{R})-\ln L(\hat{\theta}_{ML})\right)
    \end{eqnarray*}
    \item Asymptotic distribution: $LR\overset{asy}{\sim }\chi _{m}^{2}$
\end{itemize}
\end{frame}


\begin{frame}\frametitle{The three classical tests}\framesubtitle{Likelihood ratio test}
Remarks:
\begin{itemize}
    \item Reject $H_{0}$ if $LR$ is larger than the $\left( 1-\alpha \right)$-quantile of the $\chi _{m}^{2}$-distribution
    \item To compute $LR,$ one requires both the unrestricted estimator $\hat{\theta}_{ML}$ and the restricted estimator $\hat{\theta}_{R}$
    \item Ideal, if both $\hat{\theta}_{ML}$ and $\hat{\theta}_{R}$ are easy to calculate
    \item The LR test is often used to compare different models to each other
    \end{itemize}
\end{frame}


\begin{frame}\frametitle{The three classical tests}\framesubtitle{Lagrange multiplier test}
Lagrange multiplier test
\begin{itemize}
    \item Is $g(\hat{\theta}_{R})$ significantly different from 0?
    \item The test is based on the restricted estimator $\hat{\theta}_{R}$
    \item Lagrange approach: $\max_{\theta }\ln L(\theta )$ s.t. $r(\theta )=0$
    \item LM test statistic
    \begin{equation*}
        LM=g(\hat{\theta}_{R})^{\prime }\cdot \left[ \mathcal{I}(\hat{\theta}_{R})\right]^{-1}\cdot g(\hat{\theta}_{R})\overset{asy}{\sim }\chi _{m}^{2}
    \end{equation*}
    with
    \begin{equation*}
        \mathcal{I}(\hat{\theta}_{R})=-E\left( \frac{\partial ^{2}\ln L(\hat{\theta}_{R})}{\partial \theta \partial \theta ^{\prime }}\right)
    \end{equation*}
\end{itemize}
\end{frame}


\begin{frame}\frametitle{The three classical tests}\framesubtitle{Lagrange multiplier test}
Remarks:
\begin{itemize}
    \item Reject $H_{0}$ if $LM$ is larger than the $\left( 1-\alpha \right) $-quantile of the $\chi _{m}^{2}$-distribution
    \item The LM test only requires the restricted estimator
    \item Ideal, if $\hat{\theta}_{R}$ is much easier to calculate than $\hat{\theta}_{ML}$
    \item The LM test is often used to test misspecifications\newline
    (heteroskedasticity, autocorrelation, omitted variables etc.)
    \item Asymptotically, the three tests are equivalent
\end{itemize}
\end{frame}


\begin{frame}\frametitle{The three classical tests}
Multivariate case
\begin{itemize}
    \item Example: Production function
    \begin{equation*}
        Y_{i}=X_{i1}^{a_{1}}\cdot X_{i2}^{a_{2}}+u_{i}
    \end{equation*}%
    where $u_{i}\sim N(0,0.05^{2})$
    \item Log-likelihood function $\ln L(a_{1},a_{2})$
    \item ML estimators $\hat{a}_{1}$ and $\hat{a}_{2}$
    \item Hypothesis test of $a_{1}+a_{2}=1$ or $a_{1}+a_{2}-1=0$
    \item See \texttt{classtest.R}
\end{itemize}
\end{frame}

\end{document}
