\documentclass{beamer}
\usepackage{amsmath,amsfonts}
\usetheme{Madrid}
\usecolortheme{seagull}
\setbeamertemplate{navigation symbols}{}
\begin{document}
\title{GMM, Indirect Inference and Bootstrap}
\subtitle{General Method of Moments (GMM)}
\author[Willi Mutschler]{Willi Mutschler}
\date{Winter 2015/2016}
\institute{TU Dortmund}
\maketitle

\begin{frame}\frametitle{TO IMPROVE}
  \begin{itemize}
    \item Example or exercise for weighting matrix, in particular what is optimal what if not.
    \item Example without gmm package and optimal weighting matrix
    \item Clarify $E_{\theta }$:
    \item Clairfy identification and asymptotic identification: DGP, der zu $\theta$ gehört. Identifikation: Zu $\theta$ können viele DGPs gehören, aber zu jedem DGP gibt es genau ein $\theta$
    \item Interpretation of asset pricing models
    \item Do you really want distinction between elementary zero functions or simply moment conditions? NOTATION!!!
    \item Slide with motivation and ideas
    \item List with assumptions needed for consistency and normality, see Pesharan's book!
  \end{itemize}
\end{frame}
\section{General Method of Moments}
\begin{frame}\frametitle{GMM}\framesubtitle{Motivation}
\begin{itemize}
    \item Introduced by Lars Peter Hansen (1982, Econometrica)
    \item Can be applied to time series, cross sectional and panel data
    \item Many applications: macroeconomics, finance, microeconomics, agricultural economics, environmental economics, labor economics
    \item Belongs to the class of limited information methods
    \item Builds upon ideas from method of moments, minimum Chi-Square methods and instrumental variables, however, overcomes (some of) their shortcomings
\end{itemize}
\end{frame}
\begin{frame}\frametitle{GMM}\framesubtitle{Model description}
Hansen, L. (1982), Large Sample Properties of Generalized Method of Moments Estimators, Econometrica 50, 1029-1054:
\begin{quote}
In this paper we study the large sample properties of a class of generalized
method of moments (GMM) estimators which subsumes many standard econometric
estimators. To motivate this class, consider an econometric model whose
parameter vector we wish to estimate. The model implies a family of \textbf{%
orthogonality conditions that embed any economic theoretical restrictions }%
that we wish to impose or test.
\end{quote}
\end{frame}
\note{\begin{itemize}
\item Many applications in macro,finance,agricultural economics, environmental economics, labor economics. 
\item TS, cross secitional and panel data
\item ML is a full-information estimator, optimal estimator, we need to specify complete probability distribution, 2 Problems:
\begin{itemize}
  \item Sensitivity of statistical properties to distributional assumption, misspecification leads to nonoptimality and worse: biased inference
  \item Computational burden: implied likelihood hard to evaluate numerically, many additional parameters not pinned down by economic model
\end{itemize}
\item GMM is limited-information estimation method, does not need a specification of likelihood, ML is a special case!
\item We need only population moments which are deduced from assumptions of econometric model
\item Essentialy GMM builds upon ideas from method of moments but overcomes shortcomings.
\item Example for shortcomings of method of moments: normal distribution with $\mu_0$ and $\sigma_0^2$. Population moments: $E(X_t)-\mu_0=$ and $E(X_t^2)-\mu_0^2-\sigma_0^2 = 0$. Estimating equations: $T^{-1}\sum X_t - \hat{\mu}=0$ and $T^{-1}\sum X_t^2 - \hat{\mu}^2-\hat{\sigma}^2$.
\item Weakness 1: higher moments are also functions of $\mu$ and $\sigma$, however, give different estimators, which one to use? Best one: ML!
\item Weakness 2: What if we want to base estimation on 3 moment conditions? Minimum Chi-Squared approach, setup criterion function, which needs to be minimized
\item Exploit information of further moment conditions: Instrumental variables!
\item Instrumental variables are only for linear models, GMM extends this to nonlinear models
\end{itemize}}


\begin{frame}\frametitle{GMM}\framesubtitle{Model description}
John Cochrane (2005), Asset Pricing, p. 196: \medskip
\begin{quote}
Most of the effort involved with GMM is simply mapping a given problem into
the very general notation.
\end{quote}
\end{frame}

\begin{frame}\frametitle{GMM}\framesubtitle{Model description}
\begin{itemize}
    \item Describe the model by elementary zero functions%
    \begin{equation*}
    E_{\theta }\left( f_{t}\left( \theta ,y_{t}\right) \right) =0
    \end{equation*}%
    where everything can be vector-valued
    \item Parameter vector $\theta $ of length $K$
    \item Observation vectors $y_{t}$
    \item Identification condition
    \begin{equation*}
    E_{\theta _{0}}\left( f_{t}\left( \theta ,y_{t}\right) \right) \neq 0\quad \text{for all }\theta \neq \theta _{0}
    \end{equation*}
\end{itemize}
\end{frame}


\begin{frame}\frametitle{GMM}\framesubtitle{Model description}
\begin{example}[Linear regression model]
    Consider the standard model
    \begin{eqnarray*}
    y &=&X\beta +u \\
    u &\sim &N(0,\sigma ^{2}I)\text{, independent of }X
    \end{eqnarray*}
    Parameter vector $\theta =?$\\ %$\beta$
    Observations $y_{t}=?$\\ %$(y_t,x_t)$
    Elementary zero functions $f_{t}(\theta ,y_{t})=?$ %$E(y_t-x_t'\beta)=0$
\end{example}
\end{frame}


\begin{frame}\frametitle{GMM}\framesubtitle{Model description}
\begin{example}[Lognormal distribution]
    Suppose there is a random sample $X_{1},\ldots ,X_{n}$ from
    \begin{equation*}
    X\sim LN(\mu ,\sigma ^{2})
    \end{equation*}
    Parameter vector $\theta =?$\\ %\mu,\sigma^2$
    Observations $y_{t}=?$\\ %x_1,\dots,x_n$
    Elementary zero functions $f_{t}(\theta ,y_{t})=?$ %$1. x_t -exp(\mu+1/2 \sigma^2)$ und 2.$ x_t^2 - exp(2\mu+2\sigma^2)$
\end{example}
\end{frame}


\begin{frame}\frametitle{GMM}\framesubtitle{Model description}\small
\begin{example}[Asset pricing]
    The basic asset pricing formula is
    \begin{equation*}
    p_{t}=E\left( m_{t+1}x_{t+1}|\Omega _{t}\right)
    \end{equation*}
    with asset price $p$, stochastic discount factor $m$, payoff $x,$ and information set $\Omega _{t}$. Assume
    \begin{equation*}
    m_{t+1} = \beta \left(\frac{c_{t+1}}{c_t}\right)^{-\gamma}
    \end{equation*}
    with $c$ consumption, $\beta$ the time preference parameter and $\gamma$ the coefficient of relative risk aversion in the utility function $u(c)=c^{1-\gamma}/(1-\gamma)$.\\~\\
    Parameter vector $\theta =?$\\
    Observations $y_{t}=?$\\
    Elementary zero functions $f_{t}(\theta ,y_{t})=?$
\end{example}
\end{frame}


\begin{frame}\frametitle{GMM}\framesubtitle{Model description}
\begin{itemize}
    \item Stack all elementary zero functions
    \begin{equation*}
    f\left( \theta ,y\right) =\left[
    \begin{array}{c}
    f_{1}\left( \theta ,y_{1}\right) \\
    \vdots \\
    f_{n}\left( \theta ,y_{n}\right)%
    \end{array}
    \right]
    \end{equation*}
    \item Covariance matrix
    \begin{equation*}
    E\left( f\left( \theta ,y\right) \,f\left( \theta ,y\right) ^{\prime}\right) =\Omega
    \end{equation*}
    \item Dimension of $\Omega $ depends on dimension of $f_{t}(\theta ,y_{t})$
    \end{itemize}
\end{frame}
\note{not necessary n, lognormal is 2n}

\begin{frame}\frametitle{GMM}\framesubtitle{Model description}
\begin{example}[Linear regression model]
    The covariance matrix $\Omega $ is
    \begin{eqnarray*}
    E\left( f(\theta ,y\right) \,f\left( \theta ,y\right) ^{\prime }) &=&E\left(u\,u^{\prime }\right) \\
     &=&\sigma ^{2}I
    \end{eqnarray*}
    If there are autocorrelation and heteroskedasticity
    \begin{equation*}
    E\left( u\,u^{\prime }\right) =\Omega
    \end{equation*}
    \end{example}
\end{frame}


\begin{frame}\frametitle{GMM}\framesubtitle{Model description}
\begin{example}[Lognormal distribution]
    The covariance matrix $\Omega $ is
    \begin{eqnarray*}
    E\left( f(\theta ,y\right) \,f\left( \theta ,y\right) ^{\prime }) &=&E\left(
    \begin{array}{ccccc}
    f_{11}^{2} & f_{11}f_{12} & \ldots & f_{11}f_{n1} & f_{11}f_{n2} \\
    f_{12}f_{11} & f_{12}^{2} & \ldots & f_{12}f_{n1} & f_{12}f_{n2} \\
    \vdots & \vdots & \ddots & \vdots & \vdots \\
    f_{n1}f_{11} & f_{n1}f_{12} & \ldots & f_{n1}^{2} & f_{n1}f_{n2} \\
    f_{n2}f_{11} & f_{n2}f_{12} & \ldots & f_{n2}f_{n1} & f_{n2}^{2}
    \end{array}
    \right) \\
    &=&?
    \end{eqnarray*}
\end{example}
\end{frame}


\begin{frame}\frametitle{GMM}\framesubtitle{Model description}
\begin{example}[Asset pricing]
    The covariance matrix $\Omega $ is
    \begin{eqnarray*}
    E\left( f(\theta ,y\right) \,f\left( \theta ,y\right) ^{\prime }) &=&E\left(
    \begin{array}{ccc}
    f_{11}^{2} & \ldots & f_{11}f_{n1} \\
    \vdots & \ddots & \vdots \\
    f_{n1}f_{11} & \ldots & f_{n1}^{2}%
    \end{array}%
    \right) \\
    &=&?
    \end{eqnarray*}
\end{example}
\end{frame}


\begin{frame}\frametitle{GMM}\framesubtitle{Estimating equations}
\begin{itemize}
    \item To estimate $\theta $, we need $K$ estimating equations
    \item In general, they are weighted averages of the $f_{t}$
    \item In most cases, the estimating equations are based on $L\geq K$ instrumental variables $W$
    \item If $L>K$, we need to form linear combinations of the instruments
    \item Let $W$ be the $n\times L$ matrix of instruments and $J$ be an $L\times K$ matrix of full rank (we will discuss how to set J optimally later)
    \item Define the $n\times K$ matrix $Z=WJ$
\end{itemize}
\end{frame}


\begin{frame}\frametitle{GMM}\framesubtitle{Estimating equations}
\begin{itemize}
    \item Theoretical \textbf{moment conditions} (orthogonality conditions)
    \begin{equation*}
    E\left( Z_{t}^{\prime }f_{t}\left( \theta ,y_{t}\right) \right) =0
    \end{equation*}
    \item The \textbf{estimating equations }are the empirical counterpart
    \begin{equation*}
    \frac{1}{n}Z^{\prime }f(\theta ,y)=0
    \end{equation*}
    \item Solving this system yields the GMM estimator $\hat{\theta}$
\end{itemize}
\end{frame}


\begin{frame}\frametitle{GMM}\framesubtitle{Estimating equations}
\begin{example}[Linear regression model]
    The $K$ moment conditions for the linear regression model are
    \begin{equation*}
    E\left( Z_{t}^{\prime }f_{t}\left( \theta ,y_{t}\right) \right) =E\left(X_{t}^{\prime }\left( y_{t}-X_{t}^{\prime }\beta \right) \right) =0
    \end{equation*}%
    and the estimating equations are
    \begin{equation*}
    \frac{1}{n}X^{\prime }\left( y-X\beta \right) =0.
    \end{equation*}
\end{example}
\end{frame}


\begin{frame}\frametitle{GMM}\framesubtitle{Estimating equations}
\begin{example}[Lognormal distribution]
    The two moment conditions for the lognormal distribution are
    \begin{eqnarray*}
    E\left( Z_{t}^{\prime }f_{t}\left( \theta ,y_{t}\right) \right) &=&E\left(
    \left[
    \begin{array}{ll}
    1 & 0 \\
    0 & 1%
    \end{array}
    \right] \left[
    \begin{array}{c}
    f_{t1}\left( \theta ,y_{t}\right) \\
    f_{t2}\left( \theta ,y_{t}\right)
    \end{array}
    \right] \right) \\
    &=&E\left( \left[
    \begin{array}{c}
    X_{t}-\exp \left( \mu +\frac{1}{2}\sigma ^{2}\right) \\
    X_{t}^{2}-\exp \left( 2\mu +2\sigma ^{2}\right)
    \end{array}
    \right] \right) \\
    &=&\left(
    \begin{array}{c}
    0 \\
    0
    \end{array}
    \right)
    \end{eqnarray*}
\end{example}
\end{frame}


\begin{frame}\frametitle{GMM}\framesubtitle{Estimating equations}
\begin{example}[contd]
    \ldots\ and the estimating equations are
    \begin{eqnarray*}
    \frac{1}{n}Z^{\prime }f(\theta ,y) &=&\frac{1}{n}\left[
    \begin{array}{lllllll}
    1 & 0 & 1 & 0 & \ldots & 1 & 0 \\
    0 & 1 & 0 & 1 & \ldots & 0 & 1
    \end{array}
    \right] \left[
    \begin{array}{c}
    f_{11} \\
    f_{12} \\
    \vdots \\
    f_{n1} \\
    f_{n2}%
    \end{array}
    \right] \\
    &=&\left[
    \begin{array}{c}
    \frac{1}{n}\sum_{t=1}^{n}\left( X_{t}-\exp \left( \mu +\frac{1}{2}\sigma
    ^{2}\right) \right) \\
    \frac{1}{n}\sum_{t=1}^{n}\left( X_{t}^{2}-\exp \left( 2\mu +2\sigma
    ^{2}\right) \right)
    \end{array}
    \right] =\left[
    \begin{array}{c}
    0 \\
    0
    \end{array}
    \right]
    \end{eqnarray*}
\end{example}
\end{frame}


\begin{frame}\frametitle{GMM}\framesubtitle{Properties of GMM estimators}
Consistency
\begin{itemize}
    \item Assume that a law of large numbers applies to $\frac{1}{n}Z^{\prime}f(\theta ,y)$
    \item Define the limiting estimation functions
    \begin{equation*}
    \alpha \left( \theta \right) =\textsl{plim}\frac{1}{n}Z^{\prime }f\left(\theta ,y\right)
    \end{equation*}
    and the limiting estimation equations $\alpha \left( \theta \right) =0$
    \item The GMM estimator $\hat{\theta}$ is consistent if the asymptotic identification condition holds, $\alpha \left( \theta \right) \neq \alpha \left( \theta _{0}\right) $ for all $\theta \neq \theta _{0}$\hfill [P]
\end{itemize}
\end{frame}


\begin{frame}\frametitle{GMM}\framesubtitle{Properties of GMM estimators}
Asymptotic normality
\begin{itemize}
    \item Simplified notation: $f_{t}(\theta )=f_{t}(\theta ,y_{t})$, $f\left(\theta \right) =f\left( \theta ,y\right) $
    \item Additional assumption: $f_{t}\left( \theta \right) $ is continuously differentiable at $\theta _{0}$
    \item First order Taylor series expansion of
    \begin{equation*}
    \frac{1}{n}Z^{\prime }f\left( \theta \right) =0
    \end{equation*}
    in $\hat{\theta}$ around $\theta _{0}$\hfill [P]
\end{itemize}
\end{frame}


\begin{frame}\frametitle{GMM}\framesubtitle{Asymptotic efficiency}
\begin{itemize}
    \item The asymptotic distribution of $\sqrt{n}\left( \hat{\theta}-\theta_{0}\right) $ is normal with \newline
    mean 0 and covariance matrix
    \begin{equation*}
    \left( \textsl{plim}\frac{1}{n}Z^{\prime }F(\theta _{0})\right) ^{-1}\left(\textsl{plim}\frac{1}{n}Z^{\prime }\Omega Z\right) \left( \textsl{plim} \frac{1}{n}F(\theta _{0})^{\prime }Z\right)^{-1}
    \end{equation*}
    \item What is the optimal choice of $Z$ in the estimating equations?
    \item The optimal choice depends on assumptions about the matrices $F(\theta)$ and $\Omega $
\end{itemize}
\end{frame}


\begin{frame}\frametitle{GMM}\framesubtitle{Asymptotic efficiency}
Case I
\begin{itemize}
    \item If $\Omega =\sigma ^{2}I$ and $E(F_{t}(\theta _{0})f_{t}(\theta_{0}))=0$ the optimal choice is
    \begin{equation*}
    Z=F(\theta _{0})
    \end{equation*}
    \item Problem: $Z$ depends on the unknown $\theta _{0}$
    \item Solution: Solve the estimating equations
    \begin{equation*}
    \frac{1}{n}F^{\prime }(\theta )f(\theta )=0
    \end{equation*}
\end{itemize}
\end{frame}


\begin{frame}\frametitle{GMM}\framesubtitle{Asymptotic efficiency}
Case II
\begin{itemize}
    \item If $\Omega =\sigma ^{2}I$ and $E(F_{t}(\theta _{0})f_{t}(\theta_{0}))\neq 0$ but $W_{t}\in \Omega _{t},$\newline
    the optimal choice is
    \begin{equation*}
    Z=P_{W}F(\theta _{0})
    \end{equation*}
    \item Problem: $Z$ depends on the unknown $\theta _{0}$
    \item Solution: Solve the estimating equations%
    \begin{equation*}
    \frac{1}{n}F^{\prime }(\theta )P_{W}f(\theta )=0
    \end{equation*}
\end{itemize}
\end{frame}


\begin{frame}\frametitle{GMM}\framesubtitle{Asymptotic efficiency}
Case III
\begin{itemize}
    \item Suppose, the covariance matrix $\Omega $ is unknown
    \item Since $Z=WJ$, the covariance matrix of $\sqrt{n}(\hat{\theta}-\theta_{0})$ is
    \begin{equation*}
    \left( \textsl{plim}\frac{1}{n}J^{\prime }W^{\prime }F_{0}\right)^{-1}\left( \textsl{plim}\frac{1}{n}J^{\prime }W^{\prime }\Omega WJ\right)
    \left( \textsl{plim}\frac{1}{n}F_{0}{}^{\prime }WJ\right) ^{-1}
    \end{equation*}
    \item For the optimal $J=\left( W^{\prime }\Omega W\right) ^{-1}W^{\prime}F_{0}$ this becomes
    \begin{equation*}
    \left( \textsl{plim}\frac{1}{n}F_{0}^{\prime }W\left( W^{\prime }\Omega W\right) ^{-1}W^{\prime }F_{0}\right) ^{-1}
    \end{equation*}
\end{itemize}
\end{frame}


\begin{frame}\frametitle{GMM}\framesubtitle{Asymptotic efficiency}
\begin{itemize}
    \item Although $\Omega $ \textbf{cannot} be estimated consistently, the term $\frac{1}{n}W^{\prime }\Omega W$ \textbf{can} be estimated consistently (we will do that later)
    \item If $\hat{\Sigma}$ is an estimator of $\frac{1}{n}W^{\prime }\Omega W$,the optimal estimating equations are
    \begin{equation*}
    \frac{1}{n}J^{\prime }W^{\prime }f(\theta )=\frac{1}{n}F(\theta )^{\prime }W \hat{\Sigma}^{-1}W^{\prime }f(\theta )=0
    \end{equation*}
    and the estimated covariance matrix of $\hat{\theta}$ is%
    \begin{equation*}
    \widehat{Cov}(\hat{\theta})=n\left( \hat{F}^{\prime }W\hat{\Sigma}^{-1}W^{\prime }\hat{F}\right) ^{-1}
    \end{equation*}
\end{itemize}
\end{frame}


\begin{frame}\frametitle{GMM}\framesubtitle{Alternative notation}
\textbf{Attention}
\begin{itemize}
    \item Many textbooks use a different notation \newline
        (and so does the \texttt{gmm} package in R)
    \item The two approaches are equivalent
    \item The moment conditions are notated as
    \begin{equation*}
    E\left( g\left( \theta ,y_{t}\right) \right) =E\left( W_{t}^{\prime}f_{t}\left( \theta ,y\right) \right) =0
    \end{equation*}
    \item The number of moment conditions $L$ can be larger than the number of parameters $K$
\end{itemize}
\end{frame}


\begin{frame}\frametitle{GMM}\framesubtitle{Alternative notation}
\begin{itemize}
    \item The $L$ estimating equations cannot be solved exactly
    \begin{equation*}
    \bar{g}_{n}(\theta ,y)=\frac{1}{n}\sum_{t=1}^{n}g(\theta ,y_{t})=0
    \end{equation*}
    \item The GMM estimator is defined by
    \begin{equation*}
    \hat{\theta}=\arg \min \bar{g}_{n}(\theta ,y)^{\prime }\,A_{n}\,\bar{g}_{n}(\theta ,y)
    \end{equation*}
    where $A_{n}$ is a sequence of $L\times L$ weighting matrices \newline
    (which can be chosen by the user) with limit $A$
\end{itemize}
\end{frame}


\begin{frame}\frametitle{GMM}\framesubtitle{Alternative notation}
\begin{itemize}
    \item The GMM estimator based on $\bar{g}_{n}$ is consistent, $\hat{\theta}\overset{p}{\rightarrow }\theta $
    \item Asymptotic normality: Define the $L\times K$ matrix
    \begin{equation*}
    G(\theta )=\frac{\partial \bar{g}_{n}\left( \theta ,y_{t}\right) }{\partial\theta ^{\prime }}=\frac{1}{n}\sum_{t=1}^{n}\frac{\partial g(x_{t},\theta )}{\partial \theta ^{\prime }}
    \end{equation*}
    \item Assume that $\sqrt{n}\bar{g}_{n}(\theta ,y)\overset{d}{\rightarrow } N\left( 0,V\right) $, then\hfill \lbrack P]
    \begin{equation*}
    \sqrt{n}\left( \hat{\theta}-\theta _{0}\right) \overset{d}{\rightarrow } N\left( 0,\left( G^{\prime }AG\right) ^{-1}G^{\prime }AVAG\left( G^{\prime}A^{\prime }G\right) ^{-1}\right)
    \end{equation*}
    \item Asymptotically optimal weighting matrix $A$\hfill [P]
\end{itemize}
\end{frame}


\begin{frame}\frametitle{GMM}\framesubtitle{Equivalence}
\begin{itemize}
    \item The two GMM approaches (based on $f_{t}$ and $g$) are equivalent
    \item The first order condition of $\bar{g}(\theta )^{\prime }A\bar{g}(\theta )$ is
    \begin{equation*}
    \underset{K\times L\quad }{G^{\prime }}\underset{L\times L\quad }{A}\underset{L\times 1}{g}=\underset{K\times 1}{0}
    \end{equation*}
    which is the same as
    \begin{equation*}
    \underset{K\times L\quad }{J^{\prime }}\underset{L\times n\quad }{W^{\prime }}\underset{n\times 1}{f}=\underset{K\times 1}{0}
    \end{equation*}
    \item List of equivalences\hfill [P]
\end{itemize}
\end{frame}


\begin{frame}\frametitle{GMM}\framesubtitle{Covariance matrix estimation}
\begin{itemize}
    \item The covariance matrix of the elementary zero functions
    \begin{equation*}
    E\left( f\left( \theta ,y\right) f\left( \theta ,y\right) ^{\prime }\right) =\Omega
    \end{equation*}
    is often unknown
    \item There may be heteroskedasticity and autocorrelation in $\Omega $
    \item Although $\Omega $ cannot be estimated consistently, the term $\frac{1}{n}W^{\prime }\Omega W$ can be estimated consistently
\end{itemize}
\end{frame}


\begin{frame}\frametitle{GMM}\framesubtitle{Covariance matrix estimation}
\begin{itemize}
    \item Write
    \begin{equation*}
    \Sigma =\textsl{plim}_{n\rightarrow \infty }\frac{1}{n}W^{\prime }\Omega W
    \end{equation*}
    \item Assume that a suitable law of large numbers holds,
    \begin{equation*}
    \Sigma =\lim_{n\rightarrow \infty }\frac{1}{n}\sum_{t=1}^{n}\sum_{s=1}^{n}E\left( f_{t}f_{s}W_{t}^{\prime }W_{s}\right)
    \end{equation*}
    where $f_{t}=f_{t}\left( \theta ,y_{t}\right) $
\end{itemize}
\end{frame}


\begin{frame}\frametitle{GMM}\framesubtitle{Covariance matrix estimation}
\begin{itemize}
    \item Define the autocovariance matrices
    \begin{equation*}
    \Gamma (j)=\left\{
    \begin{array}{ll}
    \frac{1}{n}\sum_{t=j+1}^{n}E\left( f_{t}f_{t-j}W_{t}^{\prime }W_{t-j}\right)
    & \quad \text{for }j\geq 0 \\
    \frac{1}{n}\sum_{t=-j+1}^{n}E\left( f_{t+j}f_{t}W_{t+j}^{\prime
    }W_{t}\right)  & \quad \text{for }j<0
    \end{array}
    \right.
    \end{equation*}
    \item Then
    \begin{equation*}
    \Sigma =\lim_{n\rightarrow \infty }\sum_{j=-n+1}^{n-1}\Gamma(j)=\lim_{n\rightarrow \infty }\left( \Gamma (0)+\sum_{j=1}^{n-1}\left(\Gamma (j)+\Gamma ^{\prime }(j)\right) \right)
    \end{equation*}
\end{itemize}
\end{frame}


\begin{frame}\frametitle{GMM}\framesubtitle{Covariance matrix estimation}
\begin{itemize}
    \item The autocovariance matrix $\Gamma (j)$, $j\geq 0$, can be estimated by
    \begin{equation*}
    \hat{\Gamma}(j)=\frac{1}{n}\sum_{t=j+1}^{n}\hat{f}_{t}\hat{f}_{t-j}W_{t}^{\prime }W_{t-j}
    \end{equation*}
    \item Newey-West estimator of $\Sigma $
    \begin{equation*}
    \hat{\Sigma}=\hat{\Gamma}(0)+\sum_{j=1}^{p}\underbrace{\left( 1-\frac{j}{p+1}\right)}_{\text{Adjustment term}} \left( \hat{\Gamma}(j)+\hat{\Gamma}^{\prime }(j)\right)
    \end{equation*}
    \item Adjustment term ensures positive (semi)definiteness in finite samples
    \item Estimator tends to underestimate the autocovariance matrices, unless $p$ increase at an appropriate rate of $n^{1/3}$, see Newey and West (1994) for a procedure to select p automatically

\end{itemize}
\end{frame}


\begin{frame}\frametitle{GMM}\framesubtitle{Test of overidentifying restrictions}
    \begin{itemize}
    \item The GMM estimators minimize the criterion function
    \begin{equation*}
    \frac{1}{n}f^{\prime }(\theta )W\hat{\Sigma}^{-1}W^{\prime }f(\theta )
    \end{equation*}
    \item Asymptotically, the minimized value (Hansen's $J$ statistics, \newline
    Hansen's overidentification statistic, Hansen-Sargan statistic)\newline
    is distributed as $\chi _{L-K}^{2}$ if the overidentifying restrictions hold
    \item If the null hypothesis is rejected, then something went wrong,\newline
    e.g. the model is misspecified
\end{itemize}
\end{frame}

\end{document}
