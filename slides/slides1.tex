\documentclass[notes=show]{beamer}
\usepackage{amsmath,amsfonts}
\usetheme{Madrid}
\usecolortheme{seagull}
\setbeamertemplate{navigation symbols}{}

\begin{document}

\title{GMM, Indirect Inference and Bootstrap}
\subtitle{Prerequisites: Probability theory, statistical inference, multiple linear regression}
\author[Willi Mutschler]{Willi Mutschler}
\date{Winter 2015/2016}
\institute{TU Dortmund}
\maketitle

\section{Prerequisites: Probability theory, statistical inference, multiple linear regression}
\begin{frame}\frametitle{TO IMPROVE}
  DO YOU REALLY NEED THIS??? 
\end{frame}

\begin{frame}\frametitle{Prerequisites}\framesubtitle{Probability theory and statistical inference}
Random experiments and probability
\begin{itemize}
    \item Random experiments\newline
        (sample space, outcomes, events and their combinations)
    \item Probabilities\\*[0pt]
        (definition of probability)
    \item Conditional probability and independence\\*[0pt]
        (conditional probability, total probability, Bayes' formula, independence)
\end{itemize}
\end{frame}


\begin{frame}\frametitle{Prerequisites}\framesubtitle{Probability theory and statistical inference}
Random variables and distributions
\begin{itemize}
    \item Basics\\*[0pt]
        (cumulative distribution function, quantile function, discrete random variables, continuous random variables, linear transformations)
    \item Parameters of distributions\\*[0pt]
        (expectation, variance, moments)
    \item Important continuous distributions\\*[0pt]
        (normal or Gaussian, exponential, log-normal, $\chi ^{2}$, $t$, $F$)
\end{itemize}
\end{frame}


\begin{frame}\frametitle{Prerequisites}\framesubtitle{Probability theory and statistical inference}
Joint distributions and limit theorems
\begin{itemize}
    \item Joint distribution of two (or $n$) random variables
    \item Limit theorems\\*[0pt]
        (law of large numbers, central limit theorem)
\end{itemize}

Random samples and statistics
\begin{itemize}
    \item Random samples
    \item Statistics
    \item Statistics of normally distributed samples
\end{itemize}
\end{frame}


\begin{frame}\frametitle{Prerequisites}\framesubtitle{Probability theory and statistical inference}
Estimation
\begin{itemize}
    \item Point estimation\\*[0pt]
        (unbiasedness, consistency, estimation of expectations, estimation of probabilities and proportions, estimation of variances, of quantiles, and of correlation coefficients)
    \item Interval estimation
\end{itemize}
\end{frame}


\begin{frame}\frametitle{Prerequisites}\framesubtitle{Probability theory and statistical inference}
Hypothesis testing
\begin{itemize}
    \item Basics (null and alternative hypothesis, test statistic, critical area, error of the first and second kind, significance level, power, $p$-value)
    \item Tests for expectations\\*[0pt]
        (test for a single expectation, comparison of two expectations)
    \item Tests for probabilities and proportions
\end{itemize}
\end{frame}


\section{Prerequisites: Multiple linear regression}

\begin{frame}\frametitle{Prerequisites}\framesubtitle{Multiple linear regression}
\begin{itemize}
    \item Model specification
        \begin{equation*}
            y_{t}=\alpha +\beta _{1}x_{1t}+\ldots +\beta _{K}x_{Kt}+u_{t}\quad \text{for }t=1,\ldots ,T
        \end{equation*}
    \item In matrix notation
      \begin{equation*}
        \left[
        \begin{array}{c}
        y_{1} \\
        \vdots \\
        y_{T}%
        \end{array}
        \right] =\left[
        \begin{array}{llll}
        1 & x_{11} & \ldots & x_{K1} \\
        \vdots & \vdots &  & \vdots \\
        1 & x_{1T} & \ldots & x_{KT}
        \end{array}
        \right] \left[
        \begin{array}{c}
        \alpha \\
        \beta _{1} \\
        \vdots \\
        \beta _{K}
        \end{array}
        \right] +\left[
        \begin{array}{c}
        u_{1} \\
        \vdots \\
        u_{T}
        \end{array}
        \right]
    \end{equation*}
    or
    \begin{equation*}
        y=X\beta +u
    \end{equation*}
\end{itemize}
\end{frame}


\begin{frame}\frametitle{Prerequisites}\framesubtitle{Multiple linear regression}
Model assumptions
\begin{description}
    \item[A1] There are no omitted and no redundant exogenous variables.
    \item[A2] The relation between $y$ and $X$ is linear.
    \item[A3] The parameters $\beta $ are constant over all $T$ observations.
    \item[B1-4] Error terms
    \begin{equation*}
        u\sim N\left( 0,\sigma ^{2}I_{T}\right) .
    \end{equation*}
    \item[C1] The matrix $X$ is not stochastic.
    \item[C2] No multicollinearity, $rank(X)=K+1.$
\end{description}
\end{frame}


\begin{frame}\frametitle{Prerequisites}\framesubtitle{Multiple linear regression}
\begin{itemize}
    \item OLS estimators
    \begin{equation*}
    \hat{\beta}=\left( X^{\prime }X\right) ^{-1}X^{\prime }y
    \end{equation*}
\item Estimated model
    \begin{equation*}
        \hat{y}=X\hat{\beta}
    \end{equation*}
\item Residuals
    \begin{equation*}
        \hat{u}=y-\hat{y}
    \end{equation*}
\item Measure of fit
    \begin{equation*}
        R^{2}=\frac{\sum_{t}\left( \hat{y}_{t}-\bar{y}\right) ^{2}}{\sum_{t}\left(    y_{t}-\bar{y}\right) ^{2}}=1-\frac{\sum_{t}\hat{u}^{2}}{\sum_{t}\left( y_{t}-    \bar{y}\right) ^{2}}
    \end{equation*}
\end{itemize}
\end{frame}


\begin{frame}\frametitle{Prerequisites}\framesubtitle{Multiple linear regression}
\begin{itemize}
    \item Distribution of the OLS estimator
    \begin{equation*}
        \hat{\beta}\sim N\left( \beta ,\sigma ^{2}\left( X^{\prime }X\right)^{-1}\right)
    \end{equation*}
    \item Distribution of linear combination
    \begin{equation*}
    r^{\prime }\hat{\beta}\sim N\left( r^{\prime }\beta ,\sigma ^{2}r^{\prime}\left( X^{\prime }X\right) ^{-1}r\right)
    \end{equation*}
    where $r$ is a column vector
\end{itemize}
\end{frame}



\begin{frame}\frametitle{Prerequisites}\framesubtitle{Multiple linear regression}
\begin{itemize}
    \item The $t$ test is used to test the hypotheses%
        \begin{eqnarray*}
            H_{0} &:&r^{\prime }\beta =q \\
            H_{1} &:&r^{\prime }\beta \neq q
        \end{eqnarray*}
    \item The test statistic is
        \begin{equation*}
            t=\frac{r^{\prime }\hat{\beta}-q}{\sqrt{\sigma ^{2}r^{\prime }\left(X^{\prime }X\right) ^{-1}r}}
        \end{equation*}
    \item Reject $H_{0}$ if $|t|>c$ where $c$ is the $\left( 1-\alpha /2\right) $-quantile of the $t$-distribution with $n-K-1$ degrees of freedom
\end{itemize}
\end{frame}


\begin{frame}\frametitle{Prerequisites}\framesubtitle{Multiple linear regression}
\begin{itemize}
    \item The $F$ test is used to test the hypotheses%
        \begin{eqnarray*}
            H_{0} &:&R\beta =q \\
            H_{1} &:&R\beta \neq q
        \end{eqnarray*}
    \item The test statistic is
        \begin{equation*}
            F=\frac{\left( R\hat{\beta}-q\right) ^{\prime }\left[ R\left( X^{\prime}X\right) ^{-1}R^{\prime }\right] ^{-1}\left( R\hat{\beta}-q\right) /L}{\hat{u}^{\prime }\hat{u}/(T-K-1)}
        \end{equation*}
    \item Reject $H_{0}$ if $F>c$ where $c$ is the $\left( 1-\alpha \right) $-quantile of the $F$-distribution with $L$ and $T-K-1$ degrees of freedom
\end{itemize}
\end{frame}


\begin{frame}\frametitle{Prerequisites}\framesubtitle{Multiple linear regression}
\begin{itemize}
    \item Generalized Least Squares assumption about the error term vector%
    \begin{equation*}
        u\sim N\left( 0,\Omega \right)
    \end{equation*}
\item The GLS estimator
\begin{equation*}
    \hat{\beta}^{GLS}=\left( X^{\prime }\Omega ^{-1}X\right) ^{-1}X^{\prime}\Omega ^{-1}y
\end{equation*}
is efficient
\item GLS is a unifying framework for autocorrelation and heteroscedasticity
\end{itemize}
\end{frame}


\begin{frame}\frametitle{Prerequisites}\framesubtitle{Multiple linear regression}
\begin{itemize}
    \item Stochastic exogenous variables
    \item Case 1: The error term $u$ is independent of $X$
    \item Case 2: The error term and the exogenous variables are contemporaneously uncorrelated, i.e. for $t=1,\ldots ,T$
        \begin{equation*}
            Cov(X_{kt},u_{t})=0
        \end{equation*}
    \item Case 3: The error term and the exogenous variables are contemporaneously correlated
\end{itemize}
\end{frame}

\end{document} 