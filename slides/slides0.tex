\documentclass[notes=show]{beamer}
%%%%%%%%%%%%%%%%%%%%%%%%%%%%%%%%%%%%%%%%%%%%%%%%%%%%%%%%%%%%%%%%%%%%%%%%%%%%%%%%%%%%%%%%%%%%%%%%%%%%%%%%%%%%%%%%%%%%%%%%%%%%%%%%%%%%%%%%%%%%%%%%%%%%%%%%%%%%%%%%%%%%%%%%%%%%%%%%%%%%%%%%%%%%%%%%%%%%%%%%%%%%%%%%%%%%%%%%%%%%%%%%%%%%%%%%%%%%%%%%%%%%%%%%%%%%
\usepackage{amsmath,amsfonts}
\usetheme{Madrid}
\usecolortheme{seagull}
\setbeamertemplate{navigation symbols}{}

\begin{document}

\title{GMM, Indirect Inference and Bootstrap}
\subtitle{General information}
\author[Willi Mutschler]{Willi Mutschler}
\date{Winter 2015/2016}
\institute{TU Dortmund}
\maketitle

\section{General information}

\begin{frame}\frametitle{General information}\framesubtitle{Aims and prerequisites}
    \begin{itemize}
        \item \textbf{Objective}: learn to understand and \emph{use} advanced econometric estimation techniques
        \item Applications in Finance, Macro- and Microeconometrics
        \item \textbf{Prerequisites}:
        \begin{enumerate}
            \item Basics in probability theory \& statistical inference (Bachelor level)
            \item Basics in econometrics (multiple linear regression)
            \item Basic knowledge of R is helpful but not required
        \end{enumerate}
        \end{itemize}
\end{frame}

\begin{frame}\frametitle{General information}\framesubtitle{Literature}
    \begin{itemize}
        \item R. Davidson and J.G. MacKinnon, \emph{Econometric Theory and Methods}, Oxford University Press, 2004.
        \item A. Spanos, \emph{Statistical Foundations of Econometric Modelling}, Cambridge University Press, 1989.
        \item A.C. Davison and D.V. Hinkley, \emph{Bootstrap Methods and their Application}, Cambridge University Press, 1997.
        \item M.J. Crawley, The R Book, Wiley, 2007.
        \item Further relevant literature will be provided.
    \end{itemize}
\end{frame}

\begin{frame}\frametitle{General information}\framesubtitle{Schedule: Part I}
Part I: Repetition
    \begin{itemize}
        \item Prerequisites: Probability theory, statistical inference, multiple linear regression
        \item Multidimensional random variables
        \item Stochastic convergence and limit theorems
        \item Estimators and their properties
    \end{itemize}
\end{frame}


\begin{frame}\frametitle{General information}\framesubtitle{Schedule: Part II}
Part II: Estimation techniques
    \begin{itemize}
        \item Least squares estimation and method of moments
        \item Maximum-Likelihood estimation
        \item Instrument variables estimation
        \item GMM
        \item Indirect Inference
        \item Bootstrap
    \end{itemize}
\end{frame}


\begin{frame}\frametitle{General information}\framesubtitle{Exercises and additional material}
    \begin{itemize}
    \item All material (slides, exercises, data, further readings) can be found on the internet site of the course:\\
        \texttt{http://www.statistik.tu-dortmund.de/gmm.html}
    \item \textbf{Time and Room}
    \begin{itemize}
        \item Thursday, 08.30 - 10.00, CDI 120
        \item Thursday, 12.15 - 13.45, CDI 120
        \item The lecture and exercises are intertwined.
        \item Please bring your computer to both.
    \end{itemize}
    \item Class teacher is Rafael Kawka
    \end{itemize}
\end{frame}



\begin{frame}\frametitle{General information}\framesubtitle{Credits and Examination}
    \begin{itemize}
      \item \textbf{Credits:} This course is designed for Master students (or advanced Bachelor students). Please check with the Pr\"ufungsamt if you are eligible to get credits.
      \item \textbf{Examination:} At the end of the semester each student is required to complete an exercise sheet \emph{within a week}. 
      \begin{itemize}
      \item See exercise \emph{Estimation of the Cox-Ingersoll-Ross model} as an example
      \end{itemize}
    \end{itemize}
\end{frame}
\end{document} 