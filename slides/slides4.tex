\documentclass[notes=show]{beamer}
\usepackage{amsmath,amsfonts}
\usetheme{Madrid}
\usecolortheme{seagull}
\setbeamertemplate{navigation symbols}{}

\begin{document}

\title{GMM, Indirect Inference and Bootstrap}
\subtitle{Estimators and their properties}
\author[Willi Mutschler]{Willi Mutschler}
\date{Winter 2015/2016}
\institute{TU Dortmund}
\maketitle

\section{Estimators and their properties}
\begin{frame}\frametitle{TO IMPROVE}
  DO YOU REALLY NEED THIS???
\end{frame}
\begin{frame}\frametitle{Estimators and their properties}\framesubtitle{Statistical estimation theory}
\begin{itemize}
    \item Let $X$ be a random variable (or random vector) representing a random experiment we are interested in
    \item We would like to say something about the distribution of $X$
    \item Usually, the distribution of $X$ is unknown
    \item We have to collect information about the distribution by observing the random outcome $n$ times
    \item Before the outcomes are actually observed, we may regard the \newline $n$ observations as random variables $X_{1},\ldots ,X_{n}$
\end{itemize}
\end{frame}


\begin{frame}\frametitle{Estimators and their properties}\framesubtitle{Random samples}
\begin{itemize}
    \item The random variables $X_{1},\ldots ,X_{n}$ are called a (simple)\newline \textbf{random sample} from $X$, if
    \begin{enumerate}
        \item each $X_{i},i=1,\ldots ,n$, is distributed in the same way as $X$,
        \item $X_{1},\ldots ,X_{n}$ are stochastically independent.
    \end{enumerate}
    \item The sample elements are i.i.d.
    \item $n$ is the sample size
\end{itemize}
\end{frame}


\begin{frame}\frametitle{Estimators and their properties}\framesubtitle{Sample statistics}
\begin{itemize}
    \item The joint density of the sample elements $X_{1},\ldots ,X_{n}$ is
        \begin{equation*}
            f_{X_{1},\ldots ,X_{n}}(x_{1},\ldots,x_{n})=\prod_{i=1}^{n}f_{X_{i}}(x_{i})=\prod_{i=1}^{n}f_{X}(x_{i})
        \end{equation*}
    \item Let $g:\mathbb{R}^{n}\longrightarrow \mathbb{R}$ be a real-valued function with $n$ arguments, not containing any unknown parameters, then
        \begin{equation*}
            T=g(X_{1},\ldots ,X_{n})
        \end{equation*}
        is called a \textbf{statistic} (or sample function)
\end{itemize}
\end{frame}


\begin{frame}\frametitle{Estimators and their properties}\framesubtitle{Sample statistics}
Examples
\begin{itemize}
    \item Sample mean:
        \begin{equation*}
            \bar{X}=g(X_{1},\ldots ,X_{n})=\frac{1}{n}\cdot \sum_{i=1}^{n}X_{i}
        \end{equation*}
    \item Sample variance:
        \begin{eqnarray*}
            S^{2} &=&g(X_{1},\ldots ,X_{n})=\frac{1}{n}\cdot \sum_{i=1}^{n}\left( X_{i}-\bar{X}\right) ^{2} \\
            S^{\ast 2} &=&g(X_{1},\ldots ,X_{n})=\frac{1}{n-1}\cdot \sum_{i=1}^{n}\left(X_{i}-\bar{X}\right) ^{2}
        \end{eqnarray*}
\end{itemize}
\end{frame}

\begin{frame}\frametitle{Estimators and their properties}\framesubtitle{Sample statistics}
Examples
\begin{itemize}
    \item Empirical distribution function
        \begin{equation*}
            \hat{F}(x) = \frac{1}{n}\sum_{i=1}^n 1(X_i < x)
        \end{equation*}
        where $1(A)=1$ if A is true and $1(A)=0$ else
    \item Empirical p-quantile
        \begin{eqnarray*}
            \hat{x}_p = inf\{x \in \mathbb{R}:\hat{F}(x)\geq p\}
        \end{eqnarray*}
\end{itemize}
\end{frame}



\begin{frame}\frametitle{Estimators and their properties}\framesubtitle{Sample statistics}
Remarks:
\begin{itemize}
    \item All concepts are easily generalized to the multivariate case
    \item The statistic $T=g(X_{1},\ldots ,X_{n})$ is a function of random variables and hence also a random variable
    \item A statistic has a distribution (and thus an expectation and variance)
    \item Statistics are basic tools for estimation of parameters and hypothesis tests about parameters
\end{itemize}
\end{frame}


\begin{frame}\frametitle{Estimators and their properties}\framesubtitle{Estimators and estimates}
\begin{itemize}
    \item Let $\theta $ be a vector of unknown parameters we are interested in
    \item A statistic $\hat{\theta}(X_{1},\ldots ,X_{n})$ is called \textbf{estimator (Sch\"{a}tzer)} of $\theta $
    \item The realization $\hat{\theta}(x_{1},\ldots ,x_{n})$ is called \textbf{estimate (Sch\"{a}tzwert)}
    \item The estimator $\hat{\theta}(X_{1},\ldots ,X_{n})$ is a random vector
    \item The estimate $\hat{\theta}(x_{1},\ldots ,x_{n})$ is a vector of real numbers
    \item Notation: Usually we simply write $\hat{\theta}$ for both, but $\hat{\theta}$ and $\hat{\theta}$ are not the same thing!
\end{itemize}
\end{frame}


\begin{frame}\frametitle{Estimators and their properties}\framesubtitle{Estimators and estimates}
Example:
\begin{itemize}
    \item Let $X\sim N(\mu ,\sigma ^{2})$ with unknown parameters $\mu $ and $\sigma ^{2}$
    \item We would like to estimate the parameter vector
        \begin{equation*}
            \theta =\left[
            \begin{array}{c}
            \mu \\
            \sigma ^{2}
            \end{array}
            \right] =\left[
            \begin{array}{c}
            E(X) \\
            Var(X)
            \end{array}
            \right]
        \end{equation*}
    \item A possible estimator of $\mu $ is
        \begin{equation*}
            \hat{\mu}=\frac{1}{n}\sum_{i=1}^{n}X_{i}
        \end{equation*}
\end{itemize}
\end{frame}


\begin{frame}\frametitle{Estimators and their properties}\framesubtitle{Estimators and estimates}
\begin{itemize}
    \item A possible estimator of $\sigma ^{2}$ is
        \begin{equation*}
            \hat{\sigma}^{2}=\frac{1}{n-1}\sum_{i=1}^{n}\left( X_{i}-\hat{\mu}\right)^{2}
        \end{equation*}
    \item The estimator of $\theta $ and the estimate are
        \begin{eqnarray*}
            \hat{\theta} &=&\left[
            \begin{array}{c}
            \hat{\mu} \\
            \hat{\sigma ^{2}}%
            \end{array}%
            \right] =\left[
            \begin{array}{c}
            \frac{1}{n}\sum_{i=1}^{n}X_{i} \\
            \frac{1}{n-1}\sum_{i=1}^{n}\left( X_{i}-\hat{\mu}\right) ^{2}%
            \end{array}%
            \right] \\
            \hat{\theta} &=&\left[
            \begin{array}{c}
            \hat{\mu} \\
            \hat{\sigma ^{2}}%
            \end{array}%
            \right] =\left[
            \begin{array}{c}
            \frac{1}{n}\sum_{i=1}^{n}x_{i} \\
            \frac{1}{n-1}\sum_{i=1}^{n}\left( x_{i}-\hat{\mu}\right) ^{2}%
            \end{array}%
            \right]
        \end{eqnarray*}
\end{itemize}
\end{frame}


\begin{frame}\frametitle{Estimators and their properties}\framesubtitle{Estimators and estimates}
\begin{itemize}
    \item Why do we need the complex theoretical concept of estimators as random variables?
    \item Note that \textbf{the} estimator of $\theta $ does not exist, there are always many possible estimators
    \item Example: Let $\theta =Var(X)$; two possible estimators of $\theta $ are
        \begin{eqnarray*}
        \hat{\theta}_{1}(X_{1},\ldots ,X_{n}) &=&\frac{1}{n}\sum_{i=1}^{n}\left(X_{i}-\bar{X}\right) ^{2} \\
        \hat{\theta}_{2}(X_{1},\ldots ,X_{n}) &=&\frac{1}{n-1}\sum_{i=1}^{n}\left(X_{i}-\bar{X}\right) ^{2}
        \end{eqnarray*}
\end{itemize}
\end{frame}



\begin{frame}\frametitle{Estimators and their properties}\framesubtitle{Estimators and estimates}
Important questions:
\begin{itemize}
    \item How can we compare different estimators?
    \item What is a good estimator?
    \item Which criteria should a good estimator satisfy?
    \item Is there an optimal estimator?
    \item How can we find good estimators?
\end{itemize}
\end{frame}


\begin{frame}\frametitle{Estimators and their properties}\framesubtitle{Properties of estimators}
\begin{itemize}
    \item We distinguish two groups of properties:
    \begin{itemize}
        \item small (finite) sample properties
        \item asymptotic properties
    \end{itemize}
    \item We consider finite sample properties first
    \item For simplicity, we only consider univariate estimators
    \item Thought experiment: repeated samples
\end{itemize}
\end{frame}


\begin{frame}\frametitle{Estimators and their properties}\framesubtitle{Unbiasedness}
\begin{itemize}
    \item An estimator $\hat{\theta}(X_{1},\ldots ,X_{n})$ is called \textbf{unbiased} for $\theta $ if
    \begin{equation*}
        E\left( \hat{\theta}\right) =\theta
    \end{equation*}
    \item The bias is defined as
    \begin{equation*}
        bias(\hat{\theta})=E(\hat{\theta})-\theta
    \end{equation*}
    \item Generalization to multivariate case is obvious
\end{itemize}
\end{frame}


\begin{frame}\frametitle{Estimators and their properties}\framesubtitle{Relative efficiency}
\begin{itemize}
    \item How can two unbiased estimators of the unknown parameter $\theta $ be compared to each other?
    \item Let $\hat{\theta}_{1}$ and $\hat{\theta}_{2}$ be two unbiased estimators for $\theta $. The estimator $\hat{\theta}_{1}$ is \textbf{relatively more efficient }than $\hat{\theta}_{2}$, if
        \begin{equation*}
            Var(\hat{\theta}_{1})\leq Var(\hat{\theta}_{2})
        \end{equation*}
        for all possible $\theta $ and $Var(\hat{\theta}_{1})<Var(\hat{\theta}_{2})$ for at least one possible $\theta $
\end{itemize}
\end{frame}


\begin{frame}\frametitle{Estimators and their properties}\framesubtitle{Mean squared error}
\begin{itemize}
    \item How can two biased estimators be compared?
    \item Let $\hat{\theta}$ be an arbitrary estimator for $\theta $. Then
        \begin{eqnarray*}
            MSE(\hat{\theta}) &=&E\left[ \left( \hat{\theta}-\theta \right) ^{2}\right]\\
            &=&Var\left( \hat{\theta}\right) +\left[ bias(\hat{\theta})\right] ^{2}
        \end{eqnarray*}
        is called the \textbf{mean-squared error} of the estimator
    \item If the estimator is unbiased, its MSE is equal to its variance
    \item If $MSE(\hat{\theta}_{1})<MSE(\hat{\theta}_{2})$, then $\hat{\theta}_{1}$ is more MSE-efficient
\end{itemize}
\end{frame}


\begin{frame}\frametitle{Estimators and their properties}\framesubtitle{Asymptotic properties}
\begin{itemize}
    \item What happens if the sample size goes to infinity?
    \item Practical relevance: How do estimators behave in large samples?
    \item We consider a sequence of estimators $\hat{\theta}_{n}(X_{1},\ldots,X_{n})$ for $n=1,2,\ldots $
    \item Consistency
    \item Asymptotic normality
    \item Asymptotic efficiency
\end{itemize}
\end{frame}


\begin{frame}\frametitle{Estimators and their properties}\framesubtitle{Consistency}
\begin{itemize}
    \item An estimator $\hat{\theta}_{n}(X_{1},\ldots ,X_{n})$ is called \textbf{consistent} for $\theta $, if
        \begin{equation*}
            \textsl{plim}~\hat{\theta}_{n}(X_{1},\ldots ,X_{n})=\theta
        \end{equation*}
    \item Sufficient (but not necessary) condition for consistency:
    \begin{eqnarray*}
        \lim_{n\rightarrow \infty }E(\hat{\theta}_{n}) &=&\theta \\
        \lim_{n\rightarrow \infty }Var(\hat{\theta}_{n}) &=&0
    \end{eqnarray*}
    \item Consistency is a basic and very important property of estimators
\end{itemize}
\end{frame}


\begin{frame}\frametitle{Estimators and their properties}\framesubtitle{Consistency}
\begin{itemize}
    \item \textbf{Attention:\\} Consistency and (asymptotic) unbiasedness are not the same thing
    \item An estimator can be
    \begin{itemize}
        \item consistent and unbiased
        \item inconsistent and unbiased
        \item consistent and biased
        \item inconsistent and biased
        \item consistent and asymptotically unbiased
        \item inconsistent and asymptotically unbiased
        \item consistent and asymptotically biased
        \item inconsistent and asymptotically biased
    \end{itemize}
\end{itemize}
\end{frame}


\begin{frame}\frametitle{Estimators and their properties}\framesubtitle{Asymptotic normality}
\begin{itemize}
    \item An estimator $\hat{\theta}_{n}(X_{1},\ldots ,X_{n})$ for $\theta $ is called \textbf{asymptotically normal}, if there is a sequence of real numbers $\theta _{1},\theta _{2},\ldots $ and a function $V(\theta )$ such that
        \begin{equation*}
            \sqrt{n}\cdot \left( \hat{\theta}_{n}-\theta _{n}\right) \overset{d}{\rightarrow }U\sim N(0,V(\theta ))
        \end{equation*}
    \item Alternative notation:
    \begin{equation*}
        \hat{\theta}_{n}\overset{appr}{\sim }N(\theta _{n},V(\theta )/n)
    \end{equation*}
    \item Generalization to the multivariate case
\end{itemize}
\end{frame}


\begin{frame}\frametitle{Estimators and their properties}\framesubtitle{Laws of large number and central limit theorems}
\begin{itemize}
    \item The estimator
        \begin{equation*}
            \bar{X}=\frac{1}{n}\sum_{i=1}^{n}X_{i}
        \end{equation*}
        for the expectation $E(X)$ is consistent and asymptotically normal under some mild regularity conditions\medskip
    \item Consistency $\longrightarrow $ laws of large number
    \item Asymptotic normality $\longrightarrow $ central limit theorems
\end{itemize}
\end{frame}


\begin{frame}\frametitle{Estimators and their properties}\framesubtitle{Laws of large number and central limit theorems}
\begin{itemize}
    \item \textbf{Weak law of large numbers}: Let $X_{1},X_{2},\ldots $ be a sequence of i.i.d. random variables with $E(X_{i})=\mu $ and $Var(X_{i})=\sigma ^{2}<\infty $
    \item Consider the sequence of random variables
    \begin{equation*}
        \bar{X}_{n}=\frac{1}{n}\sum_{i=1}^{n}X_{i}
    \end{equation*}
    \item Then $\textsl{plim}~\bar{X}_{n}=\mu $
\end{itemize}
\end{frame}

\begin{frame}\frametitle{Estimators and their properties}\framesubtitle{Laws of large number and central limit theorems}
Remarks:
\begin{itemize}
    \item The law of large number states that $\bar{X}_{n}$ is consistent for $E(X)=\mu $
    \item For every (arbitrarily small) $\epsilon >0$, the probability that the sample mean $\bar{X}_{n}$ deviates around $\mu $ by less than $\pm \epsilon $ converges to zero as the sample size goes to infinity
    \item Generalization to multivariate case is obvious
    \item Both the assumption of independence and the assumption of identical distributions may be weakened
\end{itemize}
\end{frame}


\begin{frame}\frametitle{Estimators and their properties}\framesubtitle{Laws of large number and central limit theorems}
\begin{itemize}
    \item \textbf{Central limit theorem}: Let $X$ be a random variable with $E(X)=\mu $ and $Var(X)=\sigma ^{2}<\infty $, and let $X_{1},\ldots ,X_{n}$ be a random sample of $X$
    \item Consider the sequence of random variables
    \begin{equation*}
        Z_{n}=\sqrt{n}\frac{\bar{X}_{n}-\mu }{\sigma }
    \end{equation*}
    \item Then
    \begin{equation*}
        Z_{n}\overset{d}{\rightarrow }U\sim N(0,1)
    \end{equation*}
\end{itemize}
\end{frame}

\begin{frame}\frametitle{Estimators and their properties}\framesubtitle{Laws of large number and central limit theorems}
\begin{itemize}
    \item Common notations:
    \begin{eqnarray*}
        \sqrt{n}\frac{\bar{X}_{n}-\mu }{\sigma }\overset{d}{\rightarrow }U &\sim &N(0,1) \\
        \sqrt{n}\left( \bar{X}_{n}-\mu \right) \overset{d}{\rightarrow }U &\sim &N(0,\sigma ^{2})
    \end{eqnarray*}
    \begin{equation*}
        \bar{X}_{n}\overset{appr}{\sim }N\left( \mu ,\frac{\sigma ^{2}}{n}\right)
    \end{equation*}
    \item Convenient (but wrong) notation: $\sqrt{n}\left( \bar{X}_{n}-\mu \right) \overset{d}{\rightarrow }N(0,\sigma ^{2})$
\end{itemize}
\end{frame}


\begin{frame}\frametitle{Estimators and their properties}\framesubtitle{Laws of large number and central limit theorems}
\begin{itemize}
    \item \textbf{Multivariate central limit theorem}: Let $X=(X_{1},\ldots,X_{m})^{\prime }$ be a random vector with $E(X)=\mu $ and $Cov(X)=\Sigma $
    \item Let $X_{1},\ldots ,X_{n}$ be a (multivariate) random sample of $X$ and
    \begin{equation*}
        \bar{X}_{n}=\frac{1}{n}\sum_{i=1}^{n}X_{i}
    \end{equation*}
    \item Then
    \begin{equation*}
        \sqrt{n}\left( \bar{X}_{n}-\mu \right) \overset{d}{\rightarrow }U\sim N(0,\Sigma )
    \end{equation*}
\end{itemize}
\end{frame}


\begin{frame}\frametitle{Estimators and their properties}\framesubtitle{Laws of large number and central limit theorems}
Estimators of moments
\begin{itemize}
    \item Let $X_{1},\ldots ,X_{n}$ be a random sample of $X$, then
    \begin{equation*}
        \hat{\mu}_{p}=\frac{1}{n}\sum_{i=1}^{n}X_{i}^{p}
    \end{equation*}
    is an estimator for the $p$-th raw moment $\mu _{p}$ of $X$ and
    \begin{equation*}
        \hat{\mu}_{p}^{\prime }=\frac{1}{n}\sum_{i=1}^{n}\left( X_{i}-\hat{\mu}_{1}\right) ^{p}
    \end{equation*}
    is an estimator for the $p$-th central moment $\mu _{p}^{\prime }$ of $X$
\end{itemize}
\end{frame}


\begin{frame}\frametitle{Estimators and their properties}\framesubtitle{Laws of large number and central limit theorems}
Weak law of large numbers for moments
\begin{itemize}
    \item Let $X_{1},X_{2},\ldots $ be a sequence of iid random variables with
    \begin{eqnarray*}
        E(X_{i}^{p}) &=&\mu _{p} \\
        E(X_{i}^{2p}) &=&\mu _{2p}<\infty
    \end{eqnarray*}
    \item Then $\textsl{plim}~\hat{\mu}_{p}=\mu _{p}$
    \item Attention: The assumption $\mu _{2p}<\infty $ is \emph{not} innocuous!
\end{itemize}
\end{frame}


\begin{frame}\frametitle{Estimators and their properties}\framesubtitle{Laws of large number and central limit theorems}
Central limit theorem for moments
\begin{itemize}
    \item Let $X_{1},X_{2},\ldots $ be a sequence of iid random variables with
    \begin{eqnarray*}
        E(X_{i}^{p}) &=&\mu_{p} \\
        E(X_{i}^{2p}) &=&\mu_{2p}<\infty
    \end{eqnarray*}
\item Then
    \begin{equation*}
            \sqrt{n}\left( \hat{\mu}_{p}-\mu _{p}\right) \overset{d}{\rightarrow }U\sim N(0,Var\left( \hat{\mu}_{p}\right) )
    \end{equation*}
    where
    \begin{equation*}
        Var\left( \hat{\mu}_{p}\right) =\frac{\mu _{2p}-\mu _{p}^{2}}{n}
    \end{equation*}
\end{itemize}
\end{frame}


\begin{frame}\frametitle{Estimators and their properties}\framesubtitle{Glivenko-Cantelli theorem}
Fundamental theorem of mathematical statistics
\begin{itemize}
  \item Define
    \begin{align*}
      \Delta_n = sup_{x\in\mathbb{R}} \left|\hat{F}(x) - F(x) \right|.
    \end{align*}
  \item Let $X_{1},X_{2},\ldots $ be a sequence of iid random variables with distribution function $F(x)$. Then
      \begin{align*}
        P\left(lim_{n\rightarrow \infty} \Delta_n = 0\right) =1.
      \end{align*}
  \item The empirical distribution function $\hat{F}$ converges uniformly to the cumulative distribution function $F$.
\end{itemize}
\end{frame}
\end{document} 