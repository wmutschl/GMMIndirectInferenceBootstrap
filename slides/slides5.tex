\documentclass[notes=show]{beamer}
\usepackage{amsmath,amsfonts}
\usetheme{Madrid}
\usecolortheme{seagull}
\setbeamertemplate{navigation symbols}{}

\begin{document}

\title{GMM, Indirect Inference and Bootstrap}
\subtitle{Method of Moments}
\author[Willi Mutschler]{Willi Mutschler}
\date{Winter 2015/2016}
\institute{TU Dortmund}
\maketitle

\section{Least squares}

\begin{frame}\frametitle{Least squares}\framesubtitle{Linear regression}
\begin{itemize}
    \item Multiple linear regression model
    \begin{eqnarray*}
        y &=&X\beta +u \\
        u &\sim &N\left( 0,\sigma ^{2}I\right)
    \end{eqnarray*}
    \item OLS estimator
    \begin{equation*}
        \hat{\beta}=\left( X^{\prime }X\right) ^{-1}X^{\prime }y
    \end{equation*}
    \item Covariance matrix
    \begin{equation*}
        Cov\left( \hat{\beta}\right) =\sigma ^{2}\left( X^{\prime }X\right) ^{-1}
    \end{equation*}
    \item Gauss-Markov theorem
\end{itemize}
\end{frame}


\begin{frame}\frametitle{Least squares}\framesubtitle{Nonlinear regression}
\begin{itemize}
    \item Notation of Davidson and MacKinnon (2004),
    \begin{eqnarray*}
        y_{t} &=&x_{t}\left( \beta \right) +u_{t} \\
        u_{t} &\sim &IID(0,\sigma ^{2})
    \end{eqnarray*}
    \item $x_{t}(\beta )$ is a nonlinear function of the parameter vector $\beta$
    \item Example:
    \begin{equation*}
        y_{t}=\beta _{1}+\beta _{2}x_{t1}+\frac{1}{\beta _{2}}x_{t2}+u_{t}
    \end{equation*}
\end{itemize}
\end{frame}

\begin{frame}\frametitle{Least squares}\framesubtitle{Nonlinear regression}
\begin{itemize}
    \item Minimize the sum of squared residuals%
    \begin{equation*}
        \sum_{t=1}^{T}\left( y_{t}-x_{t}\left( \beta \right) \right) ^{2}
    \end{equation*}%
    with respect to $\beta $
    \item Usually, the minimization must be done numerically
\end{itemize}
\end{frame}


\section{Method of moments}

\begin{frame}\frametitle{Method of moments}\framesubtitle{Definition of moments}
\begin{itemize}
    \item Raw moment of order $p$
    \begin{equation*}
        \mu _{p}=E(X^{p})
    \end{equation*}
    \item Empirical raw moment of order $p$
    \begin{equation*}
        \hat{\mu}_{p}=\frac{1}{n}\sum_{i=1}^{n}X_{i}^{p}
    \end{equation*}%
    for a simple random sample $X_{1},\ldots ,X_{n}$
\end{itemize}
\end{frame}


\begin{frame}\frametitle{Method of moments}\framesubtitle{Basic idea: Step 1}
\begin{itemize}
    \item Write $r$ theoretical moments as functions of $r$ unknown parameters%
    \begin{eqnarray*}
        \mu _{1} &=&g_{1}\left( \theta _{1},\ldots ,\theta _{r}\right) \\
        &&\vdots \\
        \mu _{r} &=&g_{r}\left( \theta _{1},\ldots ,\theta _{r}\right)
    \end{eqnarray*}
    \item Of course, central moments may be used as well
\end{itemize}
\end{frame}

\begin{frame}\frametitle{Method of moments}\framesubtitle{Basic idea: Step 2}
\begin{itemize}
    \item Invert the system of equations: \newline
    Write the $r$ unknown parameters \newline
    as functions of the $r$ theoretical moments%
    \begin{eqnarray*}
    \theta _{1} &=&h_{1}(\mu _{1},\ldots ,\mu _{r}) \\
    &&\vdots \\
    \theta _{r} &=&h_{r}(\mu _{1},\ldots ,\mu _{r})
    \end{eqnarray*}
\end{itemize}
\end{frame}

\begin{frame}\frametitle{Method of moments}\framesubtitle{Basic idea: Step 3}
\begin{itemize}
    \item Replace all theoretical moments by empirical moments
    \begin{eqnarray*}
        \hat{\theta}_{1} &=&h_{1}(\hat{\mu}_{1},\ldots ,\hat{\mu}_{r}) \\
        &&\vdots \\
        \hat{\theta}_{r} &=&h_{r}(\hat{\mu}_{1},\ldots ,\hat{\mu}_{r})
    \end{eqnarray*}
    \item The estimators $\hat{\theta}_{1},\ldots ,\hat{\theta}_{r}$ are \textbf{moment estimators}
\end{itemize}
\end{frame}

\begin{frame}\frametitle{Method of moments}\framesubtitle{Properties of moment estimators}
\begin{itemize}
    \item Moment estimators are consistent since%
    \begin{eqnarray*}
        \textsl{plim}\hat{\theta}_{1} &=&\textsl{plim}\left( h_{1}(\hat{\mu}_{1},%
        \hat{\mu}_{2},\ldots )\right) \\
        &=&h_{1}(\textsl{plim}\hat{\mu}_{1},\textsl{plim}\hat{\mu}_{2},\ldots ) \\
        &=&h_{1}(\mu _{1},\mu _{2},\ldots ) \\
        &=&\theta _{1}
    \end{eqnarray*}
    \item In general, moment estimators are not unbiased and not efficient
    \item Since the empirical moments are asymptotically normal (why?), moment
    estimators are also asymptotically normal\newline
    $\longrightarrow $ delta method\hfill \lbrack P]
\end{itemize}
\end{frame}

\begin{frame}\frametitle{Method of moments}\framesubtitle{Example}
\begin{itemize}
    \item Let $X\sim Exp\left( \lambda \right) $ with unknown parameter $\lambda$ and let $X_{1},\ldots ,X_{n}$ be a random sample
    \item Step 1: We know that $E(X)=\mu _{1}=1/\lambda $
    \item Step 2 (inversion): $\lambda =1/\mu _{1}$
    \item Step 3: The estimator is
    \begin{equation*}
        \hat{\lambda}=\frac{1}{\hat{\mu}_{1}}=\frac{1}{\frac{1}{n}\sum_{i}X_{i}}=\frac{1}{\bar{X}_{n}}
    \end{equation*}
    \item Is $\hat{\lambda}$ unbiased?
    \item Alternative: $Var(X)=1/\lambda ^{2}$, then $\hat{\lambda}=1/\sqrt{S^{2}}$
\end{itemize}
\end{frame}

\end{document}