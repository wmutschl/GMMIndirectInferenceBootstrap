
\documentclass{article}
%%%%%%%%%%%%%%%%%%%%%%%%%%%%%%%%%%%%%%%%%%%%%%%%%%%%%%%%%%%%%%%%%%%%%%%%%%%%%%%%%%%%%%%%%%%%%%%%%%%%%%%%%%%%%%%%%%%%%%%%%%%%%%%%%%%%%%%%%%%%%%%%%%%%%%%%%%%%%%%%%%%%%%%%%%%%%%%%%%%%%%%%%%%%%%%%%%%%%%%%%%%%%%%%%%%%%%%%%%%%%%%%%%%%%%%%%%%%%%%%%%%%%%%%%%%%
\usepackage{amsmath}
\usepackage[a4paper]{geometry}

\parskip2.5ex
\parindent0mm
\raggedright
\begin{document}


\section{Delta method}

See also exercise ''Delta Method''.

Let $Y_{1},Y_{2},\ldots $ be a sequence of random variables such that $E(Y_i)=\mu$, $Var(Y_i) = \sigma^2 < \infty $ and
\begin{equation*}
\sqrt{n}\left( Y_{i}-\mu \right) \rightarrow U\sim N(0,\sigma ^{2}).
\end{equation*}
Of course, in this context $Y_{i}$ is usually just the mean $\bar{X}_{i}$.

Define a new sequence of random variables
\begin{equation*}
Z_{i}=f(Y_{i})
\end{equation*}%
where $f$ is differentiable and measurable (at least at $\mu$). What does the asymptotic distribution of $Z_{i}$ look like?

The first order Taylor expansion of $f$ around $\mu $ is
\begin{equation*}
f\left( y\right) =f\left( \mu \right) +f^{\prime }\left( \mu \right) \left(y-\mu \right) +\text{rest}
\end{equation*}
where the rest is negligible in the limit (and dropped from the notation). Thus
\begin{equation*}
Z_{i}=f\left( \mu \right) +f^{\prime }\left( \mu \right) \left( Y_{i}-\mu
\right) .
\end{equation*}%
We see that $Z_{i}$ is just a linear transformation of $Y_{i}$. Since $Y_{i}$ is asymptotically normal, so is $Z_{i}$. The asymptotic mean of $Z_{i}$ is
\begin{equation*}
E\left( Z_{i}\right) =E\left( f\left( \mu \right) +f^{\prime }\left( \mu \right) \left( Y_{i}-\mu \right) \right) = f\left( \mu \right)
\end{equation*}%
and the asymptotic variance is
\begin{eqnarray*}
Var\left( Z_{i}\right) &=&Var\left( f\left( \mu \right) +f^{\prime }\left(\mu \right) \left( Y_{i}-\mu \right) \right) \\
&=&\left[ f^{\prime }\left( \mu \right) \right] ^{2}Var\left( Y_{i}\right) \\
&= &\left[ f^{\prime }\left( \mu \right) \right] ^{2}\sigma ^{2}.
\end{eqnarray*}

\section{Expected gradient vector}

First we show that for all elements of $G$%
\begin{equation*}
E_{\theta }\left( G_{ij}\left( \theta ,X_{i}\right) \right) =0.
\end{equation*}%
Consider the identity%
\begin{equation*}
\int e^{\ln f_{X}\left( x_{i};\theta \right) }dx_{i}=1
\end{equation*}%
and differentiate both sides with respect to the $j$-th component of $\theta
$,%
\begin{eqnarray*}
\int e^{\ln f_{X}\left( x_{i};\theta \right) }\frac{\partial \ln f_{X}\left(
x_{i};\theta \right) }{\partial \theta _{j}}dx_{i} &=&0 \\
E_{\theta }\left( \frac{\partial \ln f_{X}\left( x_{i};\theta \right) }{%
\partial \theta _{j}}\right) &=&0 \\
E\left( G_{ij}\left( \theta ,X_{i}\right) \right) &=&0.
\end{eqnarray*}%
Summing over $i=1,\ldots ,n$ yields%
\begin{equation*}
E\left( g_{j}\left( \theta \right) \right) =0
\end{equation*}%
and hence%
\begin{equation*}
E\left( g\left( \theta \right) \right) =0.
\end{equation*}

\section{Information matrix equality}

Consider the expected gradient component equation $E\left( G_{ij}\left(
\theta ,X_{i}\right) \right) =0$ and differentiate both sides once more with
respect to $\theta $. The left hand side is%
\begin{equation*}
E\left( G_{ij}\left( \theta ,X_{i}\right) \right) =\int e^{\ln f_{X}\left(
x_{i};\theta \right) }\frac{\partial \ln f_{X}\left( x_{i};\theta \right) }{%
\partial \theta _{j}}dx_{i}
\end{equation*}%
and the derivative with respect to $\theta _{k}$ is%
\begin{eqnarray*}
\frac{\partial }{\partial \theta _{k}}\int e^{\ln f_{X}\left( x_{i};\theta
\right) }\frac{\partial \ln f_{X}\left( x_{i};\theta \right) }{\partial
\theta _{j}}dx_{i} &=&\int \frac{\partial }{\partial \theta _{k}}e^{\ln
f_{X}\left( x_{i};\theta \right) }\frac{\partial \ln f_{X}\left(
x_{i};\theta \right) }{\partial \theta _{j}}dx_{i} \\
&=&\int e^{\ln f_{X}\left( x_{i};\theta \right) }\frac{\partial \ln
f_{X}\left( x_{i};\theta \right) }{\partial \theta _{k}}\frac{\partial \ln
f_{X}\left( x_{i};\theta \right) }{\partial \theta _{j}}dx_{i} \\
&&+\int e^{\ln f_{X}\left( x_{i};\theta \right) }\frac{\partial \ln
f_{X}\left( x_{i};\theta \right) }{\partial \theta _{j}\partial \theta _{k}}%
dx_{i}.
\end{eqnarray*}%
Thus%
\begin{equation*}
\int e^{\ln f_{X}\left( x_{i};\theta \right) }\frac{\partial \ln f_{X}\left(
x_{i};\theta \right) }{\partial \theta _{k}}\frac{\partial \ln f_{X}\left(
x_{i};\theta \right) }{\partial \theta _{j}}dx_{i}=-\int e^{\ln f_{X}\left(
x_{i};\theta \right) }\frac{\partial \ln f_{X}\left( x_{i};\theta \right) }{%
\partial \theta _{j}\partial \theta _{k}}dx_{i}
\end{equation*}%
or%
\begin{equation*}
E_{\theta }\left( \frac{\partial \ln f_{X}\left( x_{i};\theta \right) }{%
\partial \theta _{k}}\frac{\partial \ln f_{X}\left( x_{i};\theta \right) }{%
\partial \theta _{j}}\right) =-E_{\theta }\left( \frac{\partial \ln
f_{X}\left( x_{i};\theta \right) }{\partial \theta _{j}\partial \theta _{k}}%
\right) .
\end{equation*}%
Summing over all $i=1,\ldots ,n$ and arranging the terms as a matrix yields%
\begin{equation*}
Cov\left( g\left( \theta \right) \right) =-E\left( H\left( \theta \right)
\right) .
\end{equation*}

\section{Consistency of ML estimator}

See Davidson and MacKinnon (2004, section 10.3).

Finite sample identification condition: The log-likelihood must be different
for different parameter values,%
\begin{equation*}
\ln L\left( \theta ,x\right) \neq \ln L\left( \theta ^{\prime },x\right)
\end{equation*}%
for $\theta \neq \theta ^{\prime }$ and all $x$. Let $\theta _{0}$ denote
the true parameter values. Then by Jensen's inequality%
\begin{equation}
E_{0}\ln \left( \frac{L\left( \theta ^{\prime }\right) }{L\left( \theta
_{0}\right) }\right) <\ln E_{0}\left( \frac{L\left( \theta ^{\prime }\right)
}{L\left( \theta _{0}\right) }\right)  \label{jensen1}
\end{equation}%
since the logarithm is a concave transformation. The expectation on the
right hand side is%
\begin{eqnarray*}
E_{0}\left( \frac{L\left( \theta ^{\prime }\right) }{L\left( \theta
_{0}\right) }\right) &=&\int \frac{L\left( \theta ^{\prime }\right) }{%
L\left( \theta _{0}\right) }L\left( \theta _{0}\right) dx \\
&=&\int L\left( \theta ^{\prime }\right) dx \\
&=&1
\end{eqnarray*}%
since $L\left( \theta ^{\prime }\right) $ is a density function (the joint
density of $x$ at $\theta ^{\prime }$). Hence, the right hand side of (\ref%
{jensen1}) vanishes,%
\begin{equation*}
E_{0}\ln \left( \frac{L\left( \theta ^{\prime }\right) }{L\left( \theta
_{0}\right) }\right) <0
\end{equation*}%
or%
\begin{equation}
E_{0}\ln L\left( \theta ^{\prime }\right) <E_{0}\ln L\left( \theta
_{0}\right) .  \label{ineq1}
\end{equation}%
By a law of large numbers
\begin{equation*}
\textsl{plim}\left( \frac{1}{n}\ln L\left( \theta \right) \right) =\lim
\left( \frac{1}{n}E_{0}\ln L\left( \theta \right) \right)
\end{equation*}%
for all $\theta $. Therefore, from (\ref{ineq1}) we conclude%
\begin{equation}
\textsl{plim}\left( \frac{1}{n}\ln L\left( \theta ^{\prime }\right) \right)
\leq \textsl{plim}\left( \frac{1}{n}\ln L\left( \theta _{0}\right) \right)
\label{ineq2}
\end{equation}%
for all $\theta ^{\prime }$ where the inequality is not strict because of
the limits.

Since the ML estimator $\hat{\theta}$ maximizes $\ln L\left( \theta \right) $
it must be true that%
\begin{equation}
\textsl{plim}\left( \frac{1}{n}\ln L\left( \hat{\theta}\right) \right) \geq
\textsl{plim}\left( \frac{1}{n}\ln L\left( \theta _{0}\right) \right) .
\label{ineq3}
\end{equation}%
The only way that (\ref{ineq2}) and (\ref{ineq3}) can both be true is if%
\begin{equation*}
\textsl{plim}\left( \frac{1}{n}\ln L\left( \hat{\theta}\right) \right) =%
\textsl{plim}\left( \frac{1}{n}\ln L\left( \theta _{0}\right) \right) .
\end{equation*}%
Since the asymptotic identification condition requires $\textsl{plim}\left(
\frac{1}{n}\ln L\left( \theta ^{\prime }\right) \right) \neq \textsl{plim}%
\left( \frac{1}{n}\ln L\left( \theta _{0}\right) \right) $ for all $\theta
^{\prime }\neq \theta _{0}$, consistency follows.

\section{Asymptotic normality of ML estimator}

Ignoring the residual term, the Taylor expansion is%
\begin{equation*}
g(\hat{\theta})=g\left( \theta _{0}\right) +H\left( \theta _{0}\right) (\hat{%
\theta}-\theta _{0})=0.
\end{equation*}%
Hence,%
\begin{eqnarray*}
(\hat{\theta}-\theta _{0}) &=&-\left[ H\left( \theta _{0}\right) \right]
^{-1}g\left( \theta _{0}\right) \\
\sqrt{n}(\hat{\theta}-\theta _{0}) &=&-\left[ H\left( \theta _{0}\right) %
\right] ^{-1}\sqrt{n}g\left( \theta _{0}\right) \\
\sqrt{n}(\hat{\theta}-\theta _{0}) &=&-\left[ \frac{1}{n}H\left( \theta
_{0}\right) \right] ^{-1}\sqrt{n}\bar{g}\left( \theta _{0}\right)
\end{eqnarray*}%
where $\bar{g}\left( \theta _{0}\right) =n^{-1}g\left( \theta _{0}\right) $.
By the central limit theorem, $\sqrt{n}\bar{g}\left( \theta _{0}\right) $ is
asymptotically normally distributed with%
\begin{eqnarray*}
E\left( \sqrt{n}\bar{g}\left( \theta _{0}\right) \right) &=&0 \\
Cov\left( \sqrt{n}\bar{g}\left( \theta _{0}\right) \right) &=&-E\left( \frac{%
1}{n}H\left( \theta _{0}\right) \right) .
\end{eqnarray*}%
Since $\sqrt{n}(\hat{\theta}-\theta _{0})$ is just a linear transformation
of $\sqrt{n}\bar{g}\left( \theta _{0}\right) $ it is asymptotically normal
as well with%
\begin{eqnarray*}
E\left( \sqrt{n}(\hat{\theta}-\theta _{0})\right) &=&0 \\
Cov\left( \sqrt{n}(\hat{\theta}-\theta _{0})\right) &=&-\left[ \frac{1}{n}%
H\left( \theta _{0}\right) \right] ^{-1}E\left( \frac{1}{n}H\left( \theta
_{0}\right) \right) \left[ \frac{1}{n}H\left( \theta _{0}\right) \right]
^{-1}.
\end{eqnarray*}%
In the limit $\textsl{plim}\frac{1}{n}H\left( \theta _{0}\right) =E\left(
\frac{1}{n}H\left( \theta _{0}\right) \right) $ and thus the covariance
matrix simplifies to%
\begin{equation*}
Cov\left( \sqrt{n}(\hat{\theta}-\theta _{0})\right) =-\left[ E\left( \frac{1%
}{n}H\left( \theta _{0}\right) \right) \right] ^{-1},
\end{equation*}%
and hence approximately%
\begin{equation*}
\hat{\theta}\sim N\left( \theta _{0},-\left[ E\left( H\left( \theta
_{0}\right) \right) \right] ^{-1}\right) .
\end{equation*}

\section{Errors in variables}

The model with the unobservable exogenous variable is

\begin{equation*}
y_{t}=\alpha +\beta x_{t}^{\ast }+\varepsilon _{t},\quad \varepsilon
_{t}\sim iid(0,\sigma _{\varepsilon }^{2}).
\end{equation*}%
We can only observe
\begin{equation*}
x_{t}=x_{t}^{\ast }+v_{t}.
\end{equation*}%
Estimate the model%
\begin{eqnarray*}
y_{t} &=&\alpha +\beta \left( x_{t}-v_{t}\right) +\varepsilon _{t} \\
&=&\alpha +\beta x_{t}+\underbrace{\varepsilon _{t}-\beta v_{t}}_{=u_{t}}.
\end{eqnarray*}%
The error $u_{t}$ is correlated with the exogenous variable $x_{t}$ and%
\begin{equation*}
E\left( u_{t}|x_{t}\right) =E\left( u_{t}|v_{t}\right) =-\beta v_{t},
\end{equation*}%
hence%
\begin{eqnarray*}
Cov\left( x_{t},u_{t}\right) &=&E\left( x_{t}u_{t}\right) \\
&=&E\left( x_{t}E\left( u_{t}|x_{t}\right) \right) \\
&=&E\left( \left( x_{t}^{\ast }+v_{t}\right) \left( -\beta v_{t}\right)
\right) \\
&=&-\beta \sigma _{v}^{2}.
\end{eqnarray*}%
Generalization:
\begin{equation*}
y=X^{\ast }\beta +\varepsilon ,\quad \varepsilon \sim N(0,\sigma ^{2}I)
\end{equation*}%
where a single column of the $T\times K$ matrix $X^{\ast }$ is latent (say,
the last column). We can observe%
\begin{equation*}
X=X^{\ast }+V
\end{equation*}%
where the $T\times K$ matrix of measurement errors $V$ is zero everywhere
except in the last column. Estimate the model%
\begin{eqnarray*}
y &=&\left( X-V\right) \beta +\varepsilon \\
&=&X\beta +\underbrace{\left( \varepsilon -V\beta \right) }_{=u}.
\end{eqnarray*}%
Then $E\left( u|X\right) =E\left( u|V\right) =-V\beta $. Note that \emph{all}
parameters are estimated inconsistently -- not just the last one. See
\texttt{errorsinvars.R}.

\section{Consistency of simple IV estimator}

The probability limit of%
\begin{eqnarray*}
\hat{\beta} &=&(W'X)^{-1}W'y = \beta +\left( W^{\prime }X\right) ^{-1}W^{\prime }u \\
&=&\beta +\left( \frac{1}{n}W^{\prime }X\right) ^{-1}\frac{1}{n}W^{\prime }u
\end{eqnarray*}%
is%
\begin{eqnarray*}
\textsl{plim}\hat{\beta}_{IV} &=&\beta +\textsl{plim}\left( \left( \frac{1%
}{n}W^{\prime }X\right) ^{-1}\frac{1}{n}W^{\prime }u\right) \\
&=&\beta +\textsl{plim}\left( \frac{1}{n}W^{\prime }X\right) ^{-1}\textsl{%
plim}\frac{1}{n}W^{\prime }u \\
&=&\beta +\left( \textsl{plim}\frac{1}{n}W^{\prime }X\right) ^{-1}\textsl{%
plim}\frac{1}{n}W^{\prime }u.
\end{eqnarray*}%
$\textsl{plim}\frac{1}{n}W^{\prime }u$ is a normal average, so we can use the law of large numbers.\\
If $\textsl{plim}\frac{1}{n}W^{\prime }X=S_{WX}$ is deterministic and
nonsingular, then%
\begin{equation*}
\textsl{plim}\hat{\beta}_{IV}=\beta +S_{WX}^{-1}\cdot \textsl{plim}\frac{1%
}{n}W^{\prime }u.
\end{equation*}
In other words, asymptotically, we require that W und X are correlated, since this converges to $E(w_i X_j)$
Obviously, $\hat{\beta}_{IV}$ is consistent if $\textsl{plim}\frac{1}{n}%
W^{\prime }u=0$. Write%
\begin{equation*}
\frac{1}{n}W^{\prime }u=\frac{1}{n}\sum_{t=1}^{n}W_{t}^{\prime }u_{t}
\end{equation*}%
where $W_{t}$ is the $t$-th row of $W$ (as a row vector). By the law of
large numbers%
\begin{equation*}
\frac{1}{n}\sum_{t=1}^{n}W_{t}^{\prime }u_{t}\overset{p}{\rightarrow }%
E\left( W^{\prime }u\right) =0
\end{equation*}%
or $\textsl{plim}\frac{1}{n}W^{\prime }u=0$. This establishes the
consistency of $\hat{\beta}_{IV}$.

\section{Asymptotic normality of simple IV estimator}

Multiply%
\begin{equation*}
\hat{\beta}_{IV}-\beta =\left( \frac{1}{n}W^{\prime }X\right) ^{-1}\frac{1}{n%
}W^{\prime }u
\end{equation*}%
by $\sqrt{n}$ to get%
\begin{equation}
\sqrt{n}\left( \hat{\beta}_{IV}-\beta \right) =\left( \frac{1}{n}W^{\prime
}X\right) ^{-1}\sqrt{n}\left( \frac{1}{n}W^{\prime }u\right) .  \label{1}
\end{equation}%
$\left( \frac{1}{n}W^{\prime}X\right) ^{-1}$: Law of large numbers, converges in probability to $S_{WX}^{-1}$.\\
$\sqrt{n}\left( \frac{1}{n}W^{\prime }u\right)$: Central limit theorem, converges in distribution to a Gaussian distribution.\\
The right hand side of (\ref{1}) has two terms. We already know that the
first one, $\left( \frac{1}{n}W^{\prime }X\right) ^{-1}$ converges in
probability to $S_{WX}^{-1}$. The second term is
\begin{equation*}
\sqrt{n}\left( \frac{1}{n}W^{\prime }u\right) =\sqrt{n}\left( \frac{1}{n}%
\sum_{t=1}^{n}W_{t}^{\prime }u_{t}\right) .
\end{equation*}%
We apply the central limit theorem to the mean $\frac{1}{n}%
\sum_{t=1}^{n}W_{t}^{\prime }u_{t}$. The expectation of a single summand is $%
E\left( W_{t}^{\prime }u_{t}\right) =0$ (this is a vector of length $K$),
and the $K\times K$ covariance matrix is
\begin{eqnarray*}
Cov\left( W_{t}^{\prime }u_{t}\right) &=&E\left( W_{t}^{\prime
}u_{t}u_{t}^{\prime }W_{t}\right) \\
&=&E\left( E\left( W_{t}^{\prime }u_{t}u_{t}^{\prime }W_{t}|W_{t}\right)
\right) \\
&=&E\left( W_{t}^{\prime }E\left( u_{t}u_{t}^{\prime }|W_{t}\right)
W_{t}\right) \\
&=&E\left( W_{t}^{\prime }\sigma ^{2}IW_{t}\right) \\
&=&\sigma ^{2}E\left( W_{t}^{\prime }W_{t}\right) \\
&=&:\sigma ^{2}S_{WW}
\end{eqnarray*}%
If the instruments satisfy the asymptotic identification condition $\frac{1}{%
n}W^{\prime }W\overset{p}{\rightarrow }S_{WW},$ then the central limit
theorem implies%
\begin{equation*}
\sqrt{n}\left( \frac{1}{n}\sum_{t=1}^{n}W_{t}^{\prime }u_{t}\right)
\rightarrow N\left( 0,\sigma ^{2}S_{WW}\right) .
\end{equation*}%
Thus, the second term in (\ref{1}) converges in distribution. By Cramer's
rule, if one factor of a product converges in probability and the other
factor converges in distribution, then the product converges in
distribution. So,%
\begin{eqnarray*}
\sqrt{n}\left( \hat{\beta}_{IV}-\beta \right) &=&\left( \frac{1}{n}W^{\prime
}X\right) ^{-1}\sqrt{n}\left( \frac{1}{n}W^{\prime }u\right) \\
&&\overset{d}{\rightarrow }S_{WX}^{-1}\cdot U
\end{eqnarray*}%
where $U\sim N\left( 0,\sigma ^{2}S_{WW}\right) .$ The final step is to move
the matrix $S_{WX}^{-1}$ into the covariance matrix of the normal
distribution,%
\begin{equation*}
\sqrt{n}\left( \hat{\beta}_{IV}-\beta \right) \overset{d}{\rightarrow }%
N\left( 0,\sigma ^{2}\left( S_{WX}\right) ^{-1}S_{WW}\left( S_{WX}^{\prime
}\right) ^{-1}\right) .
\end{equation*}%
Consider the covariance matrix. By definition
\begin{eqnarray*}
\sigma ^{2}\left( S_{WX}\right) ^{-1}S_{WW}\left( S_{WX}^{\prime }\right)
^{-1} &=&\sigma ^{2}\textsl{plim}\left[ \left( \frac{1}{n}W^{\prime
}X\right) ^{-1}\left( \frac{1}{n}W^{\prime }W\right) \left( \frac{1}{n}%
X^{\prime }W\right) ^{-1}\right] \\
&=&\sigma ^{2}\textsl{plim}\left[ n\left( W^{\prime }X\right) ^{-1}\left(
W^{\prime }W\right) \left( X^{\prime }W\right) ^{-1}\right] \\
&=&\sigma ^{2}\textsl{plim}\left[ \frac{1}{n}X^{\prime }W\left( W^{\prime
}W\right) ^{-1}W^{\prime }X\right] ^{-1} \\
&=&\sigma ^{2}\textsl{plim}\left[ \frac{1}{n}X^{\prime }P_{W}X\right] ^{-1}
\end{eqnarray*}%
where $P_{W}=W\left( W^{\prime }W\right) ^{-1}W^{\prime }$ is a projection
matrix. Hence, for large samples, approximately%
\begin{equation*}
\hat{\beta}_{IV}\sim N\left( \beta ,\sigma ^{2}\left( X^{\prime
}P_{W}X\right) ^{-1}\right) .
\end{equation*}

\section{Durbin-Wu-Hausman test}

The difference between the estimators is%
\begin{eqnarray*}
&&\hat{\beta}_{IV}-\hat{\beta}_{OLS} \\
&=&\left( X^{\prime }P_{W}X\right) ^{-1}X^{\prime }P_{W}y-\left( X^{\prime
}X\right) ^{-1}X^{\prime }y \\
&=&\left( X^{\prime }P_{W}X\right) ^{-1}\left( X^{\prime }P_{W}y-\left(
X^{\prime }P_{W}X\right) \left( X^{\prime }X\right) ^{-1}X^{\prime }y\right)
\\
&=&\left( X^{\prime }P_{W}X\right) ^{-1}\left( X^{\prime }P_{W}\left(
I-X\left( X^{\prime }X\right) ^{-1}X^{\prime }\right) y\right) \\
&=&\left( X^{\prime }P_{W}X\right) ^{-1}\left( X^{\prime
}P_{W}M_{X}\,y\right)
\end{eqnarray*}%
We need to test if $X^{\prime }P_{W}M_{X}\,y$ is significantly different
from 0. This term is identically equal to zero for all variables in $X$ that
are instruments (i.e. that are also in $W$). Denote the exogenous variable
in $X$ by $Z$, then%
\begin{equation*}
Z^{\prime }P_{W}M_{X}y=0
\end{equation*}%
because $Z^{\prime }P_{W}=Z^{\prime }$ since $Z$ is a subset of $W$ and $%
P_{W}W=W\left( W^{\prime }W\right) ^{-1}W^{\prime }W=W$ and $P_{W}$ is
symmetric. Next, $Z^{\prime }M_{X}=0$ since $Z$ is a subset of $X$ and $%
M_{X}X=\left( I-X\left( X^{\prime }X\right) ^{-1}X^{\prime }\right) X=0$ and
$M_{X}$ is symmetric.

Denote by $\tilde{X}$ all possibly endogenous regressors. To test if $\tilde{%
X}^{\prime }P_{W}M_{X}\,y$ is significantly different from zero, perform a
Wald test of $\delta =0$ in the regression%
\begin{equation*}
y=X\beta +P_{W}\tilde{X}\delta +u.
\end{equation*}%
The OLS estimator of $\delta $ can be obtained by regression $M_{X}y$ on $%
M_{X}P_{W}\tilde{X}$, i.e.%
\begin{equation*}
\hat{\delta}=\left( \tilde{X}^{\prime }P_{W}M_{X}P_{W}\tilde{X}\right) ^{-1}%
\tilde{X}^{\prime }P_{W}M_{X}y.
\end{equation*}%
Since $\left( \tilde{X}^{\prime }P_{W}M_{X}P_{W}\tilde{X}\right) ^{-1}$ is
positive definite, testing if $\delta =0$ is equivalent to testing if $%
\tilde{X}^{\prime }P_{W}M_{X}y$ is significantly different from 0. This
reasoning is copied from Davidson and MacKinnon (2004, section 8.7).

\section{GMM estimation of the linear regression model}

Part I: Model description. The unknown parameter vector to be estimated is $%
\beta $ (we do not attempt to estimate $\sigma ^{2}$ at this point). The
observations are $y_{t}=(X_{t},y_{t})$ where, of course, $y_{t}$ and $y_{t}$
are different things. The elementary zero functions are%
\begin{equation*}
f_{t}(\theta ,y_{t})=f_{t}(\beta ,y_{t},X_{t})=y_{t}-X_{t}^{\prime }\beta
\end{equation*}%
because%
\begin{equation*}
E(y_{t}-X_{t}^{\prime }\beta )=0.
\end{equation*}%
Note that this equation defines the model but does not say anything about
the estimating equation.

Part II: Covariance matrix. The covariance matrix of $f\left( \theta
,y\right) $ has elements%
\begin{equation*}
E(f_{t}(\theta ,y_{t})f_{s}(\theta ,y_{s}))=\left\{
\begin{array}{ll}
\sigma ^{2} & \quad \text{if }s=t \\
0 & \quad \text{if }s\neq t%
\end{array}%
\right.
\end{equation*}%
since $f_{t}$ and $f_{s}$ are independent, and $E(f_{t}^{2})=E(u_{t}^{2})=%
\sigma ^{2}$.

\section{GMM estimation of log-normal distribution}

Part I: Model description. Suppose there is a random sample $X_{1},\ldots
,X_{n}$ from $X\sim LN(\mu ,\sigma ^{2})$. The raw moment of order $p$ is%
\begin{equation*}
E\left( X^{p}\right) =\exp \left( p\mu +\frac{1}{2}p^{2}\sigma ^{2}\right) .
\end{equation*}%
The parameters to be estimated are $\theta =\left( \mu ,\sigma ^{2}\right) $%
. The observations are $y_{t}=X_{t}$. The elementary zero functions are%
\begin{equation*}
f_{t}\left( \theta ,y_{t}\right) =\left[
\begin{array}{c}
f_{t1}\left( \theta ,y_{t}\right) \\
f_{t2}\left( \theta ,y_{t}\right)%
\end{array}%
\right] =\left[
\begin{array}{c}
X_{t}-\exp \left( \mu +\frac{1}{2}\sigma ^{2}\right) \\
X_{t}^{2}-\exp \left( 2\mu +2\sigma ^{2}\right)%
\end{array}%
\right]
\end{equation*}%
because%
\begin{eqnarray*}
E\left( f_{t}\left( \theta ,y_{t}\right) \right) &=&E\left( \left[
\begin{array}{c}
X_{t}-\exp \left( \mu +\frac{1}{2}\sigma ^{2}\right) \\
X_{t}^{2}-\exp \left( 2\mu +2\sigma ^{2}\right)%
\end{array}%
\right] \right) \\
&=&\left[
\begin{array}{c}
E\left( X_{t}-\exp \left( \mu +\frac{1}{2}\sigma ^{2}\right) \right) \\
E\left( X_{t}^{2}-\exp \left( 2\mu +2\sigma ^{2}\right) \right)%
\end{array}%
\right] \\
&=&\left[
\begin{array}{c}
E\left( X_{t}\right) -\exp \left( \mu +\frac{1}{2}\sigma ^{2}\right) \\
E\left( X_{t}^{2}\right) -\exp \left( 2\mu +2\sigma ^{2}\right)%
\end{array}%
\right] \\
&=&0
\end{eqnarray*}%
Part II: Covariance matrix. The covariance matrix of $f\left( \theta
,y\right) $ is%
\begin{equation*}
E\left( f(\theta ,y\right) \,f\left( \theta ,y\right) )=E\left(
\begin{array}{ccccc}
f_{11}^{2} & f_{11}f_{12} & \ldots & f_{11}f_{n1} & f_{11}f_{n2} \\
f_{12}f_{11} & f_{12}^{2} & \ldots & f_{12}f_{n1} & f_{12}f_{n2} \\
\vdots & \vdots & \ddots & \vdots & \vdots \\
f_{n1}f_{11} & f_{n1}f_{12} & \ldots & f_{n1}^{2} & f_{n1}f_{n2} \\
f_{n2}f_{11} & f_{n2}f_{12} & \ldots & f_{n2}f_{n1} & f_{n2}^{2}%
\end{array}%
\right) .
\end{equation*}%
Since the elements of the random sample are independent,%
\begin{equation*}
E\left( f_{ti}\left( \theta ,y_{t}\right) f_{sj}\left( \theta ,y_{t}\right)
\right) =E\left( f_{tj}\left( \theta ,y_{t}\right) \right) \cdot E\left(
f_{sj}\left( \theta ,y_{t}\right) \right) =0
\end{equation*}%
for $s\neq t$ and $i,j\in \{1,2\}$. Hence, the covariance matrix is of
block-diagonal form,%
\begin{equation*}
E\left( f(\theta ,y\right) \,f\left( \theta ,y\right) )=E\left(
\begin{array}{ccccc}
f_{11}^{2} & f_{11}f_{12} &  &  &  \\
f_{12}f_{11} & f_{12}^{2} &  &  &  \\
&  & \ddots &  &  \\
&  &  & f_{n1}^{2} & f_{n1}f_{n2} \\
&  &  & f_{n2}f_{n1} & f_{n2}^{2}%
\end{array}%
\right) .
\end{equation*}%
The blocks have elements%
\begin{eqnarray*}
E\left( f_{t1}\left( \theta ,y_{t}\right) ^{2}\right) &=&E\left( \left(
X_{t}-\exp \left( \mu +\frac{1}{2}\sigma ^{2}\right) \right) ^{2}\right) \\
&=&E\left( X_{t}^{2}-2X_{t}\exp \left( \mu +\frac{1}{2}\sigma ^{2}\right)
+\exp \left( 2\mu +\sigma ^{2}\right) \right) \\
&=&E\left( X_{t}^{2}\right) -2E\left( X_{t}\right) \exp \left( \mu +\frac{1}{%
2}\sigma ^{2}\right) +\exp \left( 2\mu +\sigma ^{2}\right) \\
&=&\exp \left( 2\mu +2\sigma ^{2}\right) -2\exp \left( \mu +\frac{1}{2}%
\sigma ^{2}\right) \exp \left( \mu +\frac{1}{2}\sigma ^{2}\right) +\exp
\left( 2\mu +\sigma ^{2}\right) \\
&=&\exp \left( 2\mu +2\sigma ^{2}\right) \allowbreak -\exp \left( 2\mu
+\sigma ^{2}\right) ,
\end{eqnarray*}%
In a similar way, one can derive the other two covariances%
\begin{eqnarray*}
E\left( f_{t1}f_{t2}\right) &=&\allowbreak \exp \left( 3\mu +\frac{9}{2}%
\sigma ^{2}\right) -\exp \left( 3\mu +\frac{5}{2}\sigma ^{2}\right) \\
E\left( f_{t2}\left( \theta ,y_{t}\right) ^{2}\right) &=&\exp \left( 4\mu
+8\sigma ^{2}\right) -\exp \left( 4\mu +4\sigma ^{2}\right)
\end{eqnarray*}

\section{GMM estimation of a general asset pricing model}

Part I: Model description. The parameters to be estimated depend on the
specific case. For instance, if the stochastic discount factor is%
\begin{equation*}
m_{t}=\beta \left( \frac{c_{t}}{c_{t-1}}\right) ^{-\gamma }
\end{equation*}%
then $\theta =\left( \beta ,\gamma \right) ^{\prime }$. The observations are
time series of prices $p_{1},\ldots ,p_{n}$ and payoffs $x_{1},\ldots ,x_{n}$%
. If required, other variables may be included, e.g. time series of
consumption $c_{t}$. The elementary zero functions are%
\begin{equation*}
f_{t}\left( \theta ,y_{t}\right) =m_{t+1}\left( \theta \right) x_{t+1}-p_{t}
\end{equation*}%
because $E\left( f_{t}\left( \theta ,y_{t}\right) \right) =E\left(
m_{t+1}\left( \theta \right) x_{t+1}\right) -E\left( p_{t}\right) $ and
asset pricing models imply the general moment condition%
\begin{equation*}
E\left( p_{t}\right) =E\left( m_{t+1}\left( \theta \right) x_{t+1}\right) .
\end{equation*}%
Part II: Covariance matrix. Since time series are dependent over time, the
observations are not independent,%
\begin{equation*}
E\left( f(\theta ,y\right) \,f\left( \theta ,y\right) )=E\left(
\begin{array}{ccc}
f_{11}^{2} & \ldots & f_{11}f_{n1} \\
\vdots & \ddots & \vdots \\
f_{n1}f_{11} & \ldots & f_{n1}^{2}%
\end{array}%
\right) .
\end{equation*}%
A typical element of this matrix is%
\begin{eqnarray*}
E\left( \left( f_{t}(\theta ,y_{t}\right) \,f_{s}\left( \theta ,y_{s}\right)
\right) &=&E\left( \left( m_{t+1}\left( \theta \right) x_{t+1}-p_{t}\right)
\left( m_{s+1}\left( \theta \right) x_{s+1}-p_{s}\right) \right) \\
&=&E\left( m_{t+1}\left( \theta \right) m_{s+1}\left( \theta \right)
x_{t+1}x_{s+1}\right) -E\left( m_{t+1}\left( \theta \right)
x_{t+1}p_{s}\right) \\
&&-E\left( m_{s+1}\left( \theta \right) x_{s+1}p_{t}\right) +E\left(
p_{t}p_{s}\right) .
\end{eqnarray*}%
The exact form depends on the time series properties of the series.

\section{Consistency of GMM estimator}

Limiting estimation functions%
\begin{equation*}
\alpha \left( \theta \right) =\textsl{plim}\frac{1}{n}Z^{\prime }f\left(
\theta ,y\right)
\end{equation*}%
and limiting estimation equations%
\begin{equation*}
\alpha \left( \theta \right) =0.
\end{equation*}%
Here we assume that $\frac{1}{n}Z^{\prime }f\left( \theta ,y\right) $
actually converges in probability, i.e. that a law of large numbers holds.
In addition, we assume the asymptotic identification condition, $\alpha
\left( \theta \right) \neq \alpha \left( \theta _{0}\right) $ for all $%
\theta \neq \theta _{0}$.

We do not prove consistency rigorously, but only heuristically (following
Davidson and MacKinnon, 2004, p. 218f). To do so, assume that $\hat{\theta}$
has a deterministic probability limit, $\hat{\theta}\rightarrow \theta
_{\infty }$. We now show that $\theta _{\infty }$ must equal the true value $%
\theta _{0}$ if the asymptotic identification condition is satisfied. We
start by assuming the opposite, i.e. that $\theta _{\infty }\neq \theta _{0}$%
. By definition, the GMM estimator satisfies%
\begin{equation*}
\frac{1}{n}Z^{\prime }f\left( \hat{\theta},y\right) =0.
\end{equation*}%
Applying the $\textsl{plim}$-operator on both sides gives%
\begin{eqnarray*}
\textsl{plim}\frac{1}{n}Z^{\prime }f\left( \hat{\theta},y\right) &=&%
\textsl{plim}0 \\
\frac{1}{n}Z^{\prime }f\left( \textsl{plim}\hat{\theta},y\right) &=&0 \\
\frac{1}{n}Z^{\prime }f\left( \hat{\theta}_{\infty },y\right) &=&0 \\
\alpha \left( \theta _{\infty }\right) &=&0
\end{eqnarray*}%
However, the asymptotic identification condition requires that $\alpha
\left( \theta \right) \neq 0$ for all $\theta \neq \theta _{0}.$ It follows
that $\theta _{\infty }=\theta _{0}$, i.e. the probability limit is the true
value. The GMM estimator is consistent.

\section{Asymptotic normality of GMM estimator}
$\sqrt{n}(\hat{\theta}-\theta) \sim N(\cdot,\cdot)$ based on Taylor Series expansion or Mean Value Theorem with LLN and CLT.
A first order Taylor series expansion of
\begin{equation*}
\frac{1}{n}Z^{\prime }f\left( \theta \right) =0
\end{equation*}%
in $\hat{\theta}$ around $\theta _{0}$ gives%
\begin{equation*}
\frac{1}{n}Z^{\prime }f\left( \hat{\theta}\right) =\frac{1}{n}Z^{\prime
}f\left( \theta _{0}\right) +\frac{1}{n}Z^{\prime }F\left( \bar{\theta}\right) \left( \hat{\theta}-\theta _{0}\right) =0
\end{equation*}%
with $\bar{\theta}$ lies element by element between $\theta_0$ and $\hat{\theta}_T$.
The Jacobian matrix is
\begin{equation*}
F(\theta )=\frac{\partial f(\theta )}{\partial \theta }.
\end{equation*}%
A typical element of the $n\times K$ matrix $F(\theta )$ is%
\begin{equation*}
F_{ti}(\theta )=\frac{\partial f_{t}(\theta )}{\partial \theta _{i}}.
\end{equation*}%
[What is $F(\theta )$ in the linear regression model? $f=y-X\beta, \frac{\partial f}{\partial \beta}=-X$]

Note that $F(\theta )$ also depends on $y$ and, thus, it is stochastic.
Multiplication by $\sqrt{n}$ gives%
\begin{equation*}
\sqrt{n}\frac{1}{n}Z^{\prime }f\left( \theta _{0}\right) +\frac{1}{n}Z^{\prime }F\left( \bar{\theta} \right) \sqrt{n}\left( \hat{\theta}-\theta
_{0}\right) =0
\end{equation*}%
or%
\begin{equation*}
\sqrt{n}\left( \hat{\theta}-\theta _{0}\right) =-\left( \frac{1}{n}Z^{\prime}F\left( \bar{\theta} \right) \right) ^{-1}\sqrt{n}\frac{1}{n}Z^{\prime
}f\left( \theta _{0}\right) .
\end{equation*}
$\left( \frac{1}{n}Z^{\prime}F\left( \bar{\theta} \right) \right) ^{-1}$ is an average and converges in probability to $\textsl{plim} \left( \frac{1}{n}Z^{\prime}F(\bar{\theta}) \right) ^{-1}$. Consistency of $\hat{\theta}$ implies $\bar{\theta}$ tends to $\theta_0$, i.e. $\textsl{plim} \left( \frac{1}{n}Z^{\prime}F(\bar{\theta}) \right) ^{-1}=\textsl{plim} \left( \frac{1}{n}Z^{\prime}F(\theta_0) \right) ^{-1}$.\\
For $\sqrt{n}\frac{1}{n}Z^{\prime}f\left( \theta _{0}\right)$ we have a CLT.\\

Now, let $n\rightarrow \infty $. By the law of large numbers the mean%
\begin{equation*}
\frac{1}{n}Z^{\prime }F\left( \bar{\theta}\right)
\end{equation*}%
converges in probability. We assume that the limit $\textsl{plim}\frac{1}{n}Z^{\prime }F\left( \theta _{0}\right) $ is deterministic and nonsingular. As
to the second term on the right hand side, i.e.%
\begin{equation*}
\sqrt{n}\left( \frac{1}{n}Z^{\prime }f\left( \theta _{0}\right) \right)
\end{equation*}%
and we can invoke the central limit theorem. Hence, asymptotic normality of $%
n^{-1/2}Z^{\prime }f\left( \theta _{0}\right) $ is established. What are the
expectation and the covariance matrix of the asymptotic normal distribution?
The expectation vector is obviously 0, since $E(f(\theta _{0}))=0$. As to
the covariance matrix,%
\begin{equation*}
E\left[ \left( \sqrt{n}\frac{1}{n}Z^{\prime }f\left( \theta _{0}\right)
\right) \left( \sqrt{n}\frac{1}{n}Z^{\prime }f\left( \theta _{0}\right)
\right) ^{\prime }\right] =E\left[ \frac{1}{n}Z^{\prime }f\left( \theta
_{0}\right) f^{\prime }(\theta _{0})Z\right] .
\end{equation*}%
For $n\rightarrow \infty $, this converges to%
\begin{equation*}
\textsl{plim}\frac{1}{n}Z^{\prime }\Omega Z.
\end{equation*}%
Note that the convergence by the law of large numbers breaks down if the
second moment of $f\left( \theta _{0}\right) f^{\prime }(\theta _{0})$ does
not exist, i.e. if the fourth moment of $f(\theta _{0})$ does not exist.
This normally happens for return distributions!

The asymptotic distribution of $\sqrt{n}\left( \hat{\theta}-\theta
_{0}\right) $ also has expectation vector 0. Its covariance matrix is the
\textquotedblleft sandwich covariance matrix\textquotedblright
\begin{equation*}
\left( \textsl{plim}\frac{1}{n}Z^{\prime }F\left( \theta _{0}\right)
\right) ^{-1}\left( \textsl{plim}\frac{1}{n}Z^{\prime }\Omega Z\right)
\left( \textsl{plim}\frac{1}{n}F\left( \theta _{0}\right) ^{\prime
}Z\right) ^{-1}
\end{equation*}%
or%
\begin{equation*}
\left( \textsl{plim}\frac{1}{n}J^{\prime }W^{\prime }F\left( \theta
_{0}\right) \right) ^{-1}\left( \textsl{plim}\frac{1}{n}J^{\prime
}W^{\prime }\Omega WJ\right) \left( \textsl{plim}\frac{1}{n}F\left( \theta
_{0}\right) ^{\prime }WJ\right) ^{-1}.
\end{equation*}
In general this is not an asymptotic efficient estimator.
\section{Asymptotic efficiency}

The estimating equations are%
\begin{equation*}
\frac{1}{n}Z^{\prime }f(\theta )=0.
\end{equation*}%
The matrix $Z$ can be chosen by the user. The optimal choice depends on the
assumptions about $F(\theta )$ and $\Omega $. We consider three cases.

\subsection{Case 1}

We start with the simplest case, $\Omega =\sigma ^{2}I$ and $F(\theta _{0})$
is predetermined, i.e. $F_{t}$ and $f_{t}$ are contemporaneously
uncorrelated,%
\begin{equation*}
E\left( F_{t}(\theta _{0})f_{t}(\theta _{0})\right) =0.
\end{equation*}%
This assumption is similar to the assumption that $X_{t}$ and $u_{t}$ are
contemporaneously uncorrelated, in the linear regression model. To keep the
notation simple, ignore the probability limit and the factors of $n^{-1}$.
The asymptotic covariance matrix of the GMM estimator is%
\begin{equation*}
\left( Z^{\prime }F_{0}\right) ^{-1}Z^{\prime }\Omega Z\left(
F_{0}{}^{\prime }Z\right) ^{-1}
\end{equation*}%
where $F_{0}$ is an abbreviation for $F(\theta _{0})$. We now make use of
the following lemma: Let $A$ and $B$ be two positive definite matrices; then
$A-B$ is positive semidefinite if and only if $B^{-1}-A^{-1}$ is positive
semidefinite.

The inverse of the covariance matrix (often called precision matrix, as a
small variance means a large precision) is%
\begin{eqnarray}
&&\left( F_{0}{}^{\prime }Z\right) \left( Z^{\prime }\Omega Z\right)
^{-1}\left( Z^{\prime }F_{0}\right)  \notag \\
&=&\frac{1}{\sigma ^{2}}F_{0}{}^{\prime }Z\left( Z^{\prime }Z\right)
^{-1}Z^{\prime }F_{0}  \notag \\
&=&\frac{1}{\sigma ^{2}}F_{0}{}^{\prime }P_{Z}F_{0}  \label{prec1}
\end{eqnarray}%
where $P_{Z}=Z\left( Z^{\prime }Z\right) ^{-1}Z^{\prime }$ is a projection
matrix. If we let $Z=F_{0}$ then the precision matrix becomes%
\begin{equation}
\frac{1}{\sigma ^{2}}F_{0}{}^{\prime }F_{0}.  \label{prec2}
\end{equation}%
The difference between (\ref{prec2}) and (\ref{prec1}) is%
\begin{equation*}
\frac{1}{\sigma ^{2}}F_{0}{}^{\prime }F_{0}-\frac{1}{\sigma ^{2}}%
F_{0}{}^{\prime }P_{Z}F_{0}=\frac{1}{\sigma ^{2}}F_{0}^{\prime
}(I-P_{Z})F_{0}\text{.}
\end{equation*}%
Since the \textquotedblleft hat matrix\textquotedblright\ $I-P_{Z}$ is a
positive semidefinite matrix, the best possible precision matrix is $\frac{1%
}{\sigma ^{2}}F_{0}{}^{\prime }F_{0}$, and hence the optimal choice is $%
Z=F(\theta _{0})$.

Special case: Standard linear regression model with covariance matrix $Cov(%
\hat{\beta})=\sigma ^{2}(X^{\prime }X)^{-1}$.

\subsection{Case 2}

Suppose, $\Omega =\sigma ^{2}I$ but $F(\theta _{0})$ is not predetermined,
i.e.%
\begin{equation*}
E(F_{t}(\theta _{0})f_{t}(\theta _{0}))\neq 0\text{.}
\end{equation*}%
Define (i.e. variables that are exogenous or predetermined)
\begin{eqnarray}
\bar{F}_{t} &:&=E\left( F_{t}(\theta _{0})|\Omega _{t}\right)  \label{Fbar}
\\
V_{t} &:&=F_{t}(\theta _{0})-\bar{F}_{t}.  \notag
\end{eqnarray}%
Consider the covariance matrix of $\sqrt{n}(\hat{\theta}-\theta _{0})$ if we
set $Z_{t}=\bar{F}_{t}$. It is%
\begin{equation}
\sigma ^{2}\left( \textsl{plim}\frac{1}{n}\bar{F}^{\prime }F_{0}{}\right)
^{-1}\left( \textsl{plim}\frac{1}{n}\bar{F}^{\prime }\bar{F}\right) \left(
\textsl{plim}\frac{1}{n}F_{0}^{\prime }\bar{F}\right) ^{-1}.  \label{vcov1}
\end{equation}%
By definition of (\ref{Fbar}) we have%
\begin{equation*}
\textsl{plim}\frac{1}{n}\bar{F}^{\prime }F_{0}=\textsl{plim}\frac{1}{n}%
\bar{F}^{\prime }\left( \bar{F}+V\right) =\textsl{plim}\frac{1}{n}\bar{F}%
^{\prime }\bar{F},
\end{equation*}%
so (\ref{vcov1}) simplifies to
\begin{equation}
\sigma ^{2}\left( \textsl{plim}\frac{1}{n}\bar{F}^{\prime }\bar{F}\right)
^{-1}.  \label{vcov2}
\end{equation}%
For other instrument matrices $Z$ we find%
\begin{equation*}
\sigma ^{2}\left( \textsl{plim}\frac{1}{n}Z^{\prime }\bar{F}\right)
^{-1}\left( \textsl{plim}\frac{1}{n}Z^{\prime }Z\right) \left( \textsl{plim%
}\frac{1}{n}\bar{F}^{\prime }Z\right) ^{-1}
\end{equation*}%
where $\textsl{plim}\frac{1}{n}Z^{\prime }\bar{F}=\textsl{plim}\frac{1}{n}%
Z^{\prime }F_{0}$. Using the shorthand notation, the limit of the precision
matrix is $1/\sigma ^{2}$ times%
\begin{equation*}
\bar{F}^{\prime }Z\left( Z^{\prime }Z\right) ^{-1}Z^{\prime }\bar{F}=\bar{F}%
^{\prime }P_{Z}\bar{F}
\end{equation*}%
while the precision matrix of (\ref{vcov2}) is $1/\sigma ^{2}$ times
\begin{equation*}
\bar{F}^{\prime }\bar{F}
\end{equation*}%
so their difference is always positive semidefinite and the optimality of $Z=%
\bar{F}$ is proved. In practice, we substitute $\bar{F}$ (which is normally
unobserved) by its linear projection $P_{W}F(\theta )$. Then the estimating
equations become%
\begin{equation*}
\frac{1}{n}F(\theta )^{\prime }P_{W}f(\theta )=0
\end{equation*}%
and the asymptotic covariance matrix of $\hat{\theta}$ is estimated by%
\begin{eqnarray*}
Cov(\hat{\theta}) &=&\hat{\sigma}^{2}\left( F(\hat{\theta})^{\prime
}P_{W}P_{W}F(\hat{\theta})\right) ^{-1} \\
&=&\hat{\sigma}^{2}\left( F(\hat{\theta})^{\prime }P_{W}F(\hat{\theta}%
)\right) ^{-1}
\end{eqnarray*}%
where $\hat{\sigma}^{2}=\frac{1}{n}\sum_{i=1}^{n}f_{t}^{2}(\hat{\theta})$.
That is regress $F(\hat{\theta})$ on instruments W, i.e. use fitted values.
\subsection{Case 3}

Suppose, $\Omega $ is unknown. Since it contains $n\left( n+1\right) /2$
elements it cannot be estimated consistently. The asymptotic covariance
matrix of $\sqrt{n}(\hat{\theta}-\theta _{0})$ has the sandwich form%
\begin{equation*}
\left( \textsl{plim}\frac{1}{n}J^{\prime }W^{\prime }F_{0}\right)
^{-1}\left( \textsl{plim}\frac{1}{n}J^{\prime }W^{\prime }\Omega WJ\right)
\left( \textsl{plim}\frac{1}{n}F_{0}{}^{\prime }WJ\right) ^{-1}.
\end{equation*}%
For $J=\left( W^{\prime }\Omega W\right) ^{-1}W^{\prime }F_{0}$ we get the
optimal covariance matrix%
\begin{eqnarray*}
Cov\left( \sqrt{n}(\hat{\theta}-\theta _{0})\right) &=&\left( \textsl{plim}%
\frac{1}{n}F_{0}^{\prime }W\left( W^{\prime }\Omega W\right) ^{-1}W^{\prime
}F_{0}\right) ^{-1} \\
&&\times \left( \textsl{plim}\frac{1}{n}F_{0}^{\prime }W\left( W^{\prime
}\Omega W\right) ^{-1}W^{\prime }\Omega W\left( W^{\prime }\Omega W\right)
^{-1}W^{\prime }F_{0}\right) \\
&&\times \left( \textsl{plim}\frac{1}{n}F_{0}{}^{\prime }W\left( W^{\prime
}\Omega W\right) ^{-1}W^{\prime }F_{0}\right) ^{-1} \\
&=&\left( \textsl{plim}\frac{1}{n}F_{0}^{\prime }W\left( W^{\prime }\Omega
W\right) ^{-1}W^{\prime }F_{0}\right) ^{-1}.
\end{eqnarray*}%
We now show (a bit heuristically) that $J$ $=\left( W^{\prime }\Omega
W\right) ^{-1}W^{\prime }F_{0}$ is in fact the optimal weighting matrix, see
also exercise 9.1 in Davidson and MacKinnon (2004). The optimal precision
matrix of $\sqrt{n}\left( \hat{\theta}-\theta _{0}\right) $ is (without
plims and $n^{-1}$)%
\begin{equation*}
F_{0}^{\prime }W\left( W^{\prime }\Omega W\right) ^{-1}W^{\prime }F_{0}
\end{equation*}%
which can be written as%
\begin{equation*}
F_{0}^{\prime }\Omega ^{-1/2}P_{\Omega ^{1/2}W}\Omega ^{-1/2\prime }F_{0}
\end{equation*}%
with the projection matrix%
\begin{equation*}
P_{\Omega ^{1/2}W}=\Omega ^{1/2}W\left( W^{\prime }\Omega W\right)
^{-1}W^{\prime }\Omega ^{1/2\prime }.
\end{equation*}%
The non-optimal precision matrix for arbitrary $J$ is (without plims and $%
n^{-1}$)%
\begin{equation*}
F_{0}{}^{\prime }WJ\left( J^{\prime }W^{\prime }\Omega WJ\right)
^{-1}J^{\prime }W^{\prime }F_{0}
\end{equation*}%
which can be written as%
\begin{equation*}
F_{0}^{\prime }\Omega ^{-1/2}P_{\Omega ^{1/2}WJ}\Omega ^{-1/2\prime }F_{0}
\end{equation*}%
where
\begin{equation*}
P_{\Omega ^{1/2}WJ}=\Omega ^{1/2}WJ\left( J^{\prime }W^{\prime }\Omega
WJ\right) ^{-1}J^{\prime }W^{\prime }\Omega ^{1/2\prime }.
\end{equation*}%
The difference between the precision matrices is%
\begin{eqnarray*}
&&F_{0}^{\prime }\Omega ^{-1/2}P_{\Omega ^{1/2}W}\Omega ^{-1/2\prime
}F_{0}-F_{0}^{\prime }\Omega ^{-1/2}P_{\Omega ^{1/2}WJ}\Omega ^{-1/2\prime
}F_{0} \\
&=&F_{0}^{\prime }\Omega ^{-1/2}\left( P_{\Omega ^{1/2}W}-P_{\Omega
^{1/2}WJ}\right) \Omega ^{-1/2\prime }F_{0}.
\end{eqnarray*}%
The matrix $P_{\Omega ^{1/2}W}-P_{\Omega ^{1/2}WJ}$ is obviously symmetric;
and it is easy to show that it is idempotent, and hence it is a positive
semidefinite orthogonal projection matrix.

The optimal asymptotic covariance matrix%
\begin{equation*}
\left( \textsl{plim}\frac{1}{n}F_{0}^{\prime }W\left( W^{\prime }\Omega
W\right) ^{-1}W^{\prime }F_{0}\right) ^{-1}
\end{equation*}%
contains the unknown $\Omega $ which cannot be estimated consistently (due
to its $n(n+1)/2$ elements). However, the $L\times L$ matrix $\frac{1}{n}%
W^{\prime }\Omega W$ can be estimated consistently. Suppose that $\hat{\Sigma%
}$ is a consistent estimator of $\frac{1}{n}W^{\prime }\Omega W$. Then%
\begin{equation*}
\left( \textsl{plim}\frac{1}{n^{2}}F_{0}^{\prime }W\hat{\Sigma}%
^{-1}W^{\prime }F_{0}\right) ^{-1}
\end{equation*}%
is the asymptotic covariance matrix of $\sqrt{n}(\hat{\theta}-\theta _{0})$
and%
\begin{equation*}
\widehat{Cov}(\hat{\theta})=n\left( F(\hat{\theta})^{\prime }W\hat{\Sigma}%
^{-1}W^{\prime }F(\hat{\theta})\right) ^{-1}
\end{equation*}%
is the estimated covariance matrix of $\hat{\theta}$.

\section{Alternative notation for GMM}

As usual, we start with a Taylor series expansion of $\bar{g}$ in $\hat{%
\theta}$ around the true value $\theta _{0}$,%
\begin{equation*}
\bar{g}\left( \hat{\theta}\right) =\bar{g}\left( \theta _{0}\right) +G\left(
\theta _{0}\right) \left( \hat{\theta}-\theta _{0}\right) .
\end{equation*}%
Pre-multiply from the left by $G(\hat{\theta})^{\prime }A$,%
\begin{equation*}
G(\hat{\theta})^{\prime }A\bar{g}\left( \hat{\theta}\right) =G(\hat{\theta}%
)^{\prime }A\bar{g}\left( \theta _{0}\right) +G(\hat{\theta})^{\prime
}AG\left( \theta _{0}\right) \left( \hat{\theta}-\theta _{0}\right) .
\end{equation*}%
Note that $G(\hat{\theta})^{\prime }A\bar{g}\left( \hat{\theta}\right) =0$
are the first order conditions for the minimization problem $\bar{g}(\hat{%
\theta})^{\prime }Ag(\hat{\theta})$. Hence%
\begin{equation*}
G(\hat{\theta})^{\prime }A\bar{g}\left( \theta _{0}\right) +G(\hat{\theta}%
)^{\prime }AG\left( \theta _{0}\right) \left( \hat{\theta}-\theta
_{0}\right) =0
\end{equation*}%
or%
\begin{equation*}
\sqrt{n}\left( \hat{\theta}-\theta _{0}\right) =-\left( G(\hat{\theta}%
)^{\prime }AG\left( \theta _{0}\right) \right) ^{-1}G(\hat{\theta})^{\prime
}A\sqrt{n}\bar{g}\left( \theta _{0}\right) .
\end{equation*}%
By assumption, $\sqrt{n}\bar{g}\left( \theta _{0}\right) $ is asymptotically
normal with expectation 0 and covariance matrix $V$. Asymptotically, $G(\hat{%
\theta})$ and $G(\theta _{0})$ are the same since $\hat{\theta}$ is
consistent. We may denote both simply by $G$. Hence, $\sqrt{n}\left( \hat{%
\theta}-\theta \right) $ is also asymptotically normal with expectation 0.
Its covariance matrix is%
\begin{equation*}
\left( G^{\prime }AG\right) ^{-1}G^{\prime }AVA^{\prime }G\left( G^{\prime
}AG\right) ^{-1}.
\end{equation*}%
Hence, the weighting matrix $A$ has an impact on the asymptotic covariance
matrix and should be chosen optimally to minimize the covariance matrix.

\section{Optimal weighting matrix in the alternative notation}

It can be shown that%
\begin{equation*}
(G^{\prime }AG)^{-1}G^{\prime }AVAG(G^{\prime }A^{\prime }G)^{-1}-(G^{\prime
}V^{-1}G)^{-1}
\end{equation*}%
is positive semidefinite for all $A.$ Hence, the best possible covariance
matrix is $(G^{\prime }V^{-1}G)^{-1}$, and the optimal weighting matrix is
\begin{equation*}
A=V^{-1}.
\end{equation*}%
Since $V$ is usually unknown, use the sequence of weight matrices $A_{n}=%
\hat{V}_{n}^{-1}$.

\section{List of equivalences}

First a brief review of the notation used e.g by Davidson and MacKinnon.

\begin{center}
\begin{tabular}{ll}
elementary zero functions & $f_{t}\left( \theta ,y_{t}\right) $ \\
& $E(f)=0$ \\
& $E(ff^{\prime })=\Omega $ \\
instrumental variables & $W$ \\
moment conditions & $E(W^{\prime }f)=0$ \\
estimation equations & $J^{\prime }W^{\prime }f=0$ \\
asymptotic covariance of $\hat{\theta}$ & $\left( J^{\prime }W^{\prime
}F\right) ^{-1}\left( J^{\prime }W^{\prime }\Omega WJ\right) \left(
F^{\prime }WJ\right) ^{-1}$ \\
to be estimated & $W^{\prime }\Omega W\quad $(or $\Sigma $) \\
optimal $J$ matrix & $\left( W^{\prime }\Omega W\right) ^{-1}W^{\prime }F$
\\
optimal asympt. covariance & $\left( F^{\prime }W\left( W^{\prime }\Omega
W\right) ^{-1}W^{\prime }F\right) ^{-1}$%
\end{tabular}
\end{center}

And a brief review of the notation used by many others.

\begin{center}
\begin{tabular}{ll}
moment conditions & $E(g)=0$ \\
& $\bar{g}=0$ \\
estimation by minimizing & $\bar{g}^{\prime }A\bar{g}$ \\
or estimation equations & $G^{\prime }A\bar{g}=0$ \\
asymptotic covariance of $\hat{\theta}$ & $\left( G^{\prime }AG\right)
^{-1}G^{\prime }AVA^{\prime }G\left( G^{\prime }AG\right) ^{-1}$ \\
to be estimated & $V$ \\
optimal $A$ matrix & $V^{-1}$ \\
optimal asympt. covariance & $\left( G^{\prime }V^{-1}G\right) ^{-1}$%
\end{tabular}
\end{center}

List of equivalences between both notations.

\begin{center}
\begin{tabular}{ll}
Davidson \& MacKinnon & Other notation \\ \hline
$E(W^{\prime }f)=0$ & $E(g)=0$ \\
$W^{\prime }f$ & $\bar{g}$ \\
$J^{\prime }W^{\prime }f=0$ & $G^{\prime }A\bar{g}=0$ \\
$J$ & $A^{\prime }G$ \\
$W^{\prime }F$ & $G$ \\
$W^{\prime }\Omega W$ or $\Sigma $ & $V$ \\
$\left( J^{\prime }W^{\prime }F\right) ^{-1}\left( J^{\prime }W^{\prime
}\Omega WJ\right) \left( F{}^{\prime }WJ\right) ^{-1}$ & $\left( G^{\prime
}AG\right) ^{-1}G^{\prime }AVA^{\prime }G\left( G^{\prime }AG\right) ^{-1}$
\\
$J=\left( W^{\prime }\Omega W\right) ^{-1}W^{\prime }F$ & $A^{\prime
}G=V^{-1}G$ \\
$\left( F^{\prime }W\left( W^{\prime }\Omega W\right) ^{-1}W^{\prime
}F\right) ^{-1}$ & $\left( G^{\prime }V^{-1}G\right) ^{-1}$%
\end{tabular}
\end{center}

\end{document}
